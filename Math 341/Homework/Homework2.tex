\documentclass[12pt]{article}
\usepackage[utf8]{inputenc}
\usepackage{mathtools}
\usepackage{amsthm}
\usepackage{amsmath}
\usepackage{amsfonts}
\usepackage{amssymb}
\usepackage{centernot}
\renewcommand\qedsymbol{QED}
\newcommand{\N}{\mathbb{N}}
\newcommand{\Z}{\mathbb{Z}}
\newcommand{\divby}{%
  \mathrel{\text{\vbox{\baselineskip.65ex\lineskiplimit0pt\hbox{.}\hbox{.}\hbox{.}}}}%
  }
\newcommand{\notdivby}{\centernot\divby}
\title{\scalebox{2}{Math 341 Homework 2}}
\author{\scalebox{1.5}{Theo Koss}}
\date{September 2020}
\begin{document}
\maketitle
\section{Practice problems}
\subsection{Problem 2}
Prove: if $(a-b)\divby c$, then $a\divby c$ iff $b\divby c$.
\begin{proof}
To prove iff, we must first prove ``If A, then B'' (forwards), then prove``if B, then A'' (backwards). \newline $(\Longrightarrow)$: Assume that $a\divby c$ is true, that is, $a=cn$. $(a-b)\divby c$ implies that $(a-b)=cl, l\in \Z$. Rearranging for $b$ we get $b=a-cl=cn-cl=c\cdot(n-l)\therefore$ if $a\divby c, b\divby c$.\newline $(\Longleftarrow)$: Assume that $b\divby c$ is true, that is, $b=cm$. We know from above that $(a-b)=cl$. Rearranging for $a$ we get $a=cl+b=cl+cm=c\cdot(l+m) \therefore$ if $b\divby c, a\divby c$.
\end{proof}
\subsection{Problem 3}
Prove: If $a\divby c$, then for any $b,(ab)\divby c$.
\begin{proof}
$a\divby c$ implies that $\exists n \in \Z$ s.t. $a=cn$. Thus, $a\cdot b=b(cn)$, and by the commutative property, $b(cn)=c(bn)$. Let $(bn)=m$, so $a\cdot b=cm, m\in \Z$, therefore $a\cdot b$ is divisible by $c$.
\end{proof}
\newpage\subsection{Problem 4}
Prove: if $a\divby b$ and $b\divby c$, then $a\divby c$.
\begin{proof}
$a\divby b \Longrightarrow a=bn, n\in \Z$. Also, $b\divby c \Longrightarrow b=cm$. Solving the first equation for $b$: $b=\frac{a}{n}$. Plugging it into the second: $\frac{a}{n}=cm$. So $a=cmn=c\cdot (mn)$. Therefore $a\divby c$.
\end{proof}
\end{document}
