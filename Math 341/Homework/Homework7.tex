\documentclass[12pt]{article}
\usepackage[utf8]{inputenc}
\usepackage{mathtools}
\usepackage{amsthm}
\usepackage{amsmath}
\usepackage{amsfonts}
\usepackage{amssymb}
\usepackage{centernot}
\newtheorem{theorem}{Theorem}
\newtheorem{corollary}{Corollary}[theorem]
\newtheorem*{remark}{Remark}
\renewcommand\qedsymbol{QED}
\newcommand{\N}{\mathbb{N}}
\newcommand{\Z}{\mathbb{Z}}
\newcommand{\divby}{%
  \mathrel{\text{\vbox{\baselineskip.65ex\lineskiplimit0pt\hbox{.}\hbox{.}\hbox{.}}}}%
  }
\newcommand{\notdivby}{\centernot\divby}
\title{\scalebox{2}{Math 341 Homework 7}}
\author{\scalebox{1.5}{Theo Koss}}
\date{October 2020}
\begin{document}
\maketitle
\section{Practice problems}
\subsection{Problem 4}
True or False: There are infinitely many primes.
\newline True.\begin{proof}
Suppose $p_1=2<p_2=3<...<p_n$ are all the primes. Let $P$ be the product of all the primes plus one $(P=p_1p_2...p_n+1)$. Then $P$ is either prime or it is not. If $P$ is prime, then it is a prime that wasn't in our list. If $P$ is composite, then there exists some $p$ such that $P\divby p$. Notice $p$ can not be any of $p_1=2>p_2=3>...>p_n$, otherwise $p$ would divide 1, which is impossible. So then $p$ is some prime which is not on our list. So there must exist at least one prime not in our list, for any list of primes, so there are infinitely many primes.
\end{proof}
\subsection{Problem 5}
Describe $\gcd(a,b)$ in terms of the prime factorizations of $a$ and $b$.
\begin{proof}
\begin{remark}Let the infinite product $n=2^{n_2}3^{n_3}5^{n_5}...p^{n_p}...$ be the prime factorization of $n$. For example $n=18=2^1\cdot3^2\cdot5^0\cdot...p^0...$ \end{remark}
\newline Let $a=2^{a_2}3^{a_3}5^{a_5}...p^{a_p}...$ and $b=2^{b_2}3^{b_3}5^{b_5}...p^{b_p}...$, then $\gcd(a,b)=$ some $c=2^{c_2}3^{c_3}5^{c_5}...p^{c_p}...$, where $c_p=\min(a_p,b_p)$. Since the minimum power of each prime which is shared between $a_p,b_p$ is $c_p$, and when $a,b$ share prime factors, the product of those shared factors is the gcd.
\end{proof}
\subsection{Problem 6}
Find lcm$(1,2,3,...,20)$.
Prime factorizations:
\begin{enumerate}
    \item $1=1$
    \item $2=2$
    \item $3=3$
    \item $4=2*2$
    \item $5=5$
    \item $6=2*3$
    \item $7=7$
    \item $8=2*2*2$
    \item $9=3*3$
    \item $10=2*5$
    \item $11=11$
    \item $12=2*2*3$
    \item $13=13$
    \item $14=2*7$
    \item $15=3*5$
    \item $16=2*2*2*2$
    \item $17=17$
    \item $18=2*3*3$
    \item $19=19$
    \item $20=2*2*5$
\end{enumerate}
Highest amount that each prime occurs: $19\cdot17\cdot13\cdot11\cdot7\cdot5\cdot3^2\cdot2^4=232,792,560$
\end{document}