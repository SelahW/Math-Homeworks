\documentclass[12pt]{article}
\usepackage[utf8]{inputenc}
\usepackage{mathtools}
\usepackage{amsthm}
\usepackage{amsmath}
\usepackage{amsfonts}
\usepackage{amssymb}
\usepackage{centernot}
\usepackage{marvosym}
\let\marvosymLightning\Lightning
\newtheorem{theorem}{Theorem}
\newtheorem{corollary}{Corollary}[theorem]
\newtheorem*{remark}{Remark}
\renewcommand\qedsymbol{QED}
\newcommand{\N}{\mathbb{N}}
\newcommand{\Z}{\mathbb{Z}}
\newcommand{\divby}{%
  \mathrel{\text{\vbox{\baselineskip.65ex\lineskiplimit0pt\hbox{.}\hbox{.}\hbox{.}}}}%
  }
\newcommand{\notdivby}{\centernot\divby}
\title{\scalebox{2}{Math 341 Homework 10}}
\author{\scalebox{1.5}{Theo Koss}}
\date{October 2020}
\begin{document}
\maketitle
\section{Practice problems}
\subsection{Problem 10.1}
Let $a,n\in\N$, $n>1$ be relatively prime. Prove that $[a]_n$ has a multiplicative inverse in $Z_n$.
\begin{proof}
Since $\gcd$ can be written as a linear combination, and $a,n$ being relatively prime implies $\gcd{(a,n)=1}$, then $$ax+ny=1$$ Is true. We can take both sides modulo $n$:$$(ax+by)\mod{n}=1\mod{n}$$This can be rewritten as:$$[a]_n[x]_n+[n]_n[y]_n=[1]_n$$Since $n\equiv0\mod{n}$, $$[a]_n[x]_n+[n]_n[y]_n=[a]_n[x]_n+0\cdot[y]_n=[1]_n$$Then, by definition, $$[a]_n[x]_n=[1]_n$$As required, $x\in\Z_n$ is the inverse of $a$.
\end{proof}
\subsection{Problem 10.3}
Find all the invertible elements of $\Z_{10}$.
\newline \begin{remark}
By problem 10.1, an element $a\in\Z_{x}$ is invertible iff $\gcd{(a,x)}=1$, that is to say that the only invertible elements are relatively prime to the modulus $x$.\end{remark}
Therefore the invertible elements in $\Z_{10}$ are $\{1,3,7,9\}$. We can also check that this is correct because Euler's Totient function, $\phi(10)=4$, which is the number of elements relatively prime to 10. :)
\subsection{Problem 10.4}
Let $p$ be a prime, find all invertible elements of $\Z_{p}$.
\newline As described above, invertible elements are all of the elements $a\in\Z_{p}$ such that $\gcd{(a,p)=1}$. Since $p$ is prime, it is relatively prime to every number smaller than it, so the invertible elements of $\Z_{p}$ are: $\{1,2,...,p-1\}$.
\end{document}