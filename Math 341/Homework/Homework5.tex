\documentclass[12pt]{article}
\usepackage[utf8]{inputenc}
\usepackage{mathtools}
\usepackage{amsthm}
\usepackage{amsmath}
\usepackage{amsfonts}
\usepackage{amssymb}
\usepackage{centernot}
\renewcommand\qedsymbol{QED}
\newcommand{\N}{\mathbb{N}}
\newcommand{\Z}{\mathbb{Z}}
\newcommand{\divby}{%
  \mathrel{\text{\vbox{\baselineskip.65ex\lineskiplimit0pt\hbox{.}\hbox{.}\hbox{.}}}}%
  }
\newcommand{\notdivby}{\centernot\divby}
\title{\scalebox{2}{Math 341 Homework 5}}
\author{\scalebox{1.5}{Theo Koss}}
\date{September 2020}
\begin{document}
\maketitle
\section{Practice problems}
\subsection{Problem 2}
Suppose $a\divby b$. Show that $\gcd (a,b)=b$.
\begin{proof}
If $a\divby b$, then by definition, $a=bn$, for some $n\in\Z$. This also, by definition, implies that $b$ is a \emph{factor} of $a$. The common denominators, $\text{cd}(a,b)=\{1,\dots,b\}$, at the very least, includes 1 and $b$. Also, $b$ must be the largest element, because the factors of $b$ are $b=\{1,...,b\}$, meaning the largest factor is $b$ itself. And the factors of $a$ are $a=\{1,...,b,...,a\}$. The largest element that both of these share is $b$, since $b$ is the largest factor of $b$ and is also included in $a$. Therefore $\gcd (a,b)=b$ if $a\divby b$.
\end{proof}
\subsection{Problem 3}
Suppose $a\notdivby b$, divide $a$ by $b$ with remainder $r$. Show that $\gcd(a,b)=\gcd(b,r)$.
\begin{proof}
If $a\notdivby b$, then by definition, $\exists q,r\in\Z, \text{ such that }a=bq+r,\text{ and }\newline0\leq r<b$. Denote $X=\gcd(a,b)$ and $Y=\gcd(b,r)$.
By definition, $a\divby X$ must be true, as must $b\divby X$. And, by example 2.2, since $a\divby X$, $b\divby X$, and $(a-bq)\divby X$. Then $r\divby X$. Also since both $b,r\divby X$, then $X\leq \gcd(b,r)$. This means that $X$ is in $Y$, or $X\subset Y$. Similarly, since $b\divby Y$, $r\divby Y$, and $(bq+r)\divby Y$, that means $a\divby Y$. And, since $a\divby Y$ and $b\divby Y$, $Y\leq \gcd(a,b)$. This means that $Y\subset X$. Since $X\subset Y$ and $Y\subset X$, $Y=X$, and therefore if $a\notdivby b, \gcd(a,b)=\gcd(b,r)$.
\end{proof}
\subsection{Problem 7}
Prove that Euclid’s algorithm works, i.e. it always stops and produces $\gcd(a,b).$
\begin{proof}
By definition of division, for any $a,b\in\N$, such that $a>b$, $\exists q,r\in\N, \text{ s.t. } a=bq+r$. Due to the iterative nature of Euclid's algorithm, I'll denote the first ``step'' as $a=bq_1+r_1$, second, $b=r_1q_2+r_2$, all of the form $r_{n-1}=r_nq_{n+1}+r_{n+1}$. Since you take a smaller value every time, it follows that $0\leq r_{n}<r_{n-1}<...<r_1<b$. And, due to the fact that it is a strictly decreasing sequence of positive integers, you can't keep getting smaller indefinitely, and so eventually $r_{n+1}=0$. In other words, it always terminates. As for why Euclid's Alg. always produces $\gcd(a,b)$, by problem 5.3, $\gcd(a,b)=\gcd(b,r_1)=\gcd(r_1,r_2)=...=\gcd(r_{n-1},r_n)$, and since $r_{n+1}=0$, then $\gcd(a,b)=\gcd(r_n,0)=r_n$. So Euclid's algorithm always terminates, and always produces $\gcd(a,b)$.
\end{proof}
\end{document}