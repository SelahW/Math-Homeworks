\documentclass[12pt]{article}
\usepackage[utf8]{inputenc}
\usepackage{mathtools}
\usepackage{amsthm}
\usepackage{amsmath}
\usepackage{amsfonts}
\usepackage{amssymb}
\usepackage{centernot}
\renewcommand\qedsymbol{QED}
\newcommand{\N}{\mathbb{N}}
\newcommand{\Z}{\mathbb{Z}}
\newcommand{\divby}{%
  \mathrel{\text{\vbox{\baselineskip.65ex\lineskiplimit0pt\hbox{.}\hbox{.}\hbox{.}}}}%
  }
\newcommand{\notdivby}{\centernot\divby}
\title{\scalebox{2}{Math 341 Homework 3}}
\author{\scalebox{1.5}{Theo Koss}}
\date{September 2019}
\begin{document}
\maketitle
\section{Practice problems}
\subsection{Problem 1}
Prove by induction: $n(n+7)\divby2$ for any $n\in\N$.
\begin{proof}
\emph{Basis:}If $n=1$, check: $1(1+7)\divby2$.\newline \emph{Inductive step:} Assume the statement is true for $n=k$. Then $k(k+7)\divby2$. Need to show that it is true for $n=k+1: (k+1)(k+8)\divby2$:$$k^{2}+9k+8=k(k+7)+(2k+8)=k(k+7)+2(k+4)$$ We know by assumption that $k(k+7)\divby2$, and by definition of divisibility, $2(k+4)$ is divisible by 2 for any $k\in\N$. Therefore, the statement is true for all $n\in\N$.
\end{proof}
\subsection{Problem 3}
Prove by induction: $7^{n}+3^{n}\divby2$ for all $n\in\N$.
\begin{proof}
\emph{Basis}: If $n=1$, check: $7^{1}+3^{1}=10\divby2$.\newline \emph{Inductive step}. Assume the statement holds for $n=k$, $7^{k}+3^{k}\divby2$, or $7^{k}+3^{k}=2d$. We must show that the statement is true for $n=k+1$. So $7^{k+1}+3^{k+1}\divby2$, or $7^{k+1}+3^{k+1}=2c$. $$7^{k+1}+3^{k+1}=7\cdot7^{k}+3\cdot3^{k}$$ Since $7^{k}+3^{k}\divby2$, then $7^{k}+3^{k}=2d$, and $3^{k}=2d-7^{k}$. So:$$=7\cdot7^{k}+3\cdot(2d-7^{k})=4\cdot7^{k}+6d$$ We have 2 terms, $4\cdot7^{k}$ and $6d$. Since both are divisible by 2, the sum of them will always be divisible by 2, and since that equation is the same as $7^{k+1}+3^{k+1}$, that means $7^{k+1}+3^{k+1}\divby2$. Therefore, by induction, $7^{n}+3^{n}\divby2$ for all $n\in\N$.
\end{proof}
\subsection{Problem 7}
Prove: a number n is divisible by 5 iff its last digit is either 0 or 5.
\begin{proof}
($\Longrightarrow$): All numbers $n$ which end in 0 can be written as $n=10k, k\in\Z$, since $10\divby5$ always, $10k\divby5$, and therefore $n\divby5$ if $n$ ends in 0. If a number $n$ ends in 5 it can be expressed as $n=10k+5$, and since $10k+5=5(k+1)$, $10k+5\divby5$. Which means $n\divby5$ if $n$ ends in 5.
\newline ($\Longleftarrow$): If $n$ ends in any other number, then $n=10k+a$, where $a\in S, S=\{1,2,3,4,6,7,8,9\}$ since $10k\divby5$, we can remove it, so in essence, $n=a$ (mod 5). Since $a\neq5l$ for $l\in\Z$, $10k+a\notdivby5$ and so $n\notdivby5$ if $n$ ends in any a.
\newline $\therefore$ $n\divby5 \Longleftrightarrow$ $n$ ends in 0 or 5.
\end{proof}
\end{document}