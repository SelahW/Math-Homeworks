\documentclass[12pt]{article}
\usepackage[utf8]{inputenc}
\usepackage{mathtools}
\usepackage{amsthm}
\usepackage{amsmath}
\usepackage{amsfonts}
\usepackage{amssymb}
\usepackage{centernot}
\usepackage{marvosym}
\let\marvosymLightning\Lightning
\newtheorem{theorem}{Theorem}
\newtheorem{corollary}{Corollary}[theorem]
\newtheorem*{remark}{Remark}
\renewcommand\qedsymbol{QED}
\newcommand{\N}{\mathbb{N}}
\newcommand{\Z}{\mathbb{Z}}
\newcommand{\divby}{%
  \mathrel{\text{\vbox{\baselineskip.65ex\lineskiplimit0pt\hbox{.}\hbox{.}\hbox{.}}}}%
  }
\newcommand{\notdivby}{\centernot\divby}
\title{\scalebox{2}{Math 341 Homework 11}}
\author{\scalebox{1.5}{Theo Koss}}
\date{December 2020}
\begin{document}
\maketitle
\section{Practice problems}
\subsection{Problem 12.1}
Prove that for every $n\geq0$, $11^{12n+6}+1\divby13$.
\begin{proof}
N2S: $11^{12n+6}\equiv-1\mod{13}$. By definition, $$11^{12n+6}=(11^{12})^n\cdot(11)^6$$ And by FlT, $a^{12}\equiv1\mod{13}$, so $$(11^{12})^n\cdot(11)^6\equiv(1)^n\cdot(11)^6\mod{13}$$ For $n\geq0$, $1^n=1$. Also, $11\equiv-2\mod{13}$ So $$(1)^n\cdot(11)^6=(-2)^6=64\equiv-1\mod{13}$$ As required.
\end{proof}
\subsection{Problem 12.3}
Let $p$ be an odd prime. Prove that $$\sum_{n=1}^{p-1}n\divby p$$
\begin{proof}
Since $p$ is odd, it must be of the form $p=2k+1$, for some $k\in\Z$. By example 1.1, $$\sum_{n=1}^{p-1}n=\frac{(p-1)(p)}{2}$$ So $$2\sum_{n=1}^{p-1}n=p(p-1)$$ Using $p=2k+1$, $$p(p-1)=p(2k)=2kp$$ Since twice the sum is equal to $2kp$, the sum must be equal to $kp$. By definition, $kp\divby p$, As required.
\end{proof}
\subsection{Problem 12.6}
Solve $x^{21}\equiv6\mod{7}$.
\begin{proof} Using FlT, $x^{6}\equiv1\mod{7}$. Since $x^{18}=(x^{6})^3$, $x^{18}\equiv1\mod{7}$. So $x^{21}\equiv x^3\mod{7}$. \newline We N2S that $x^3-6\equiv0\mod{7}$, in other words, $x^3-6=7n$, for some $n\in\N$. $$x^3-6=7n$$ $$x^3=7n+6$$ $$x=\sqrt[3]{7n+6}$$ If $n=3,17,30$, clearly $x\equiv3,5,6\mod{7}$ are solutions, respectively. They are also the only solution because it is clear that $\nexists n\in\N$ such that: 
$$0=7n+6$$ $$1=7n+6$$ $$8=7n+6$$ $$64=7n+6$$
And since these are all of the $x\in\Z_{7}$, we have found all the solutions and proven that there are no others.\end{proof}
\end{document}