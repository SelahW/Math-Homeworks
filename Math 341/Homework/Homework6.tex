\documentclass[12pt]{article}
\usepackage[utf8]{inputenc}
\usepackage{mathtools}
\usepackage{amsthm}
\usepackage{amsmath}
\usepackage{amsfonts}
\usepackage{amssymb}
\usepackage{centernot}
\newtheorem{theorem}{Theorem}
\newtheorem{corollary}{Corollary}[theorem]
\newtheorem*{remark}{Remark}
\renewcommand\qedsymbol{QED}
\newcommand{\N}{\mathbb{N}}
\newcommand{\Z}{\mathbb{Z}}
\newcommand{\divby}{%
  \mathrel{\text{\vbox{\baselineskip.65ex\lineskiplimit0pt\hbox{.}\hbox{.}\hbox{.}}}}%
  }
\newcommand{\notdivby}{\centernot\divby}
\title{\scalebox{2}{Math 341 Homework 6}}
\author{\scalebox{1.5}{Theo Koss}}
\date{September 2020}
\begin{document}
\maketitle
\section{Practice problems}
\subsection{Problem 1}
Find all prime numbers $p$ such that $p+1$ is prime.
\newline $p=2$.
\begin{proof}
\begin{theorem}
For any odd numbers $n,a$, $n+a$ is even. Recall the definition of an odd number is some number $n=2k+1,k\in\Z$, or $a=2k_1+1,k_1\in\Z$. An even number $m=2l,l\in\Z$. So $n+a=2k+2k_1+2=2(k+k_1+1)$, and since $(k+k_1+1)\in\Z$, $n+a$ is even.
\end{theorem}
\begin{theorem}
\label{thm2}
Any positive even integer $n>2$ is composite, since $n=2k$, for some $k>1\in\N$, therefore 2 divides $n$, and since $2\neq1$ and $2\neq n$, by definition $n$ is composite.
\end{theorem}
\begin{corollary}
All prime numbers $p\neq2$ are odd.
\end{corollary}
There are 2 cases for this problem:
\begin{enumerate}
    \item Case 1: $p=2$, if $p=2$, $p+1=3$ is prime. So this case is a solution.
    \item Case 2: $p$ is a prime number greater than 2. Thus, by Theorem 1, $p+1=n$, where $n$ is some some positive even integer $>2$. And by Theorem 2, any positive even integer greater than 2 is composite, thus every prime number greater than 2 does not work.
\end{enumerate}
\end{proof}
\subsection{Problem 5}
Prove that for any $n\in\N$, $n$ and $n+1$ are relatively prime.
\begin{proof}
\begin{remark}Two numbers $a,b\in\N$ are relatively prime if $\gcd(a,b)=1$.\end{remark}
Also recall the Euclidean Algorithm, by definition 5.1 using Euclidean Alg. on $(n+1,n)$, we achieve:$$n+1=n\cdot1+1$$ $$n=1\cdot n+0$$
The Euclidean Algorithm is over, and it states that $\gcd(n+1,n)=1$, therefore $n+1,n$ are relatively prime.
\end{proof}
\subsection{Problem 9}
True or false: for any $n\in\N, n^2+n+41$ is prime.
False.
\begin{proof}
Counterexample: Let $n=40$, $40^2+40+41=1681$, and $1681=41\cdot41 \therefore$ by definition, since $41\in\N$, the number is composite and the proposition is false.
\end{proof}
\end{document}