\documentclass[12pt]{article}
\usepackage[utf8]{inputenc}
\usepackage{mathtools}
\usepackage{amsthm}
\usepackage{amsmath}
\usepackage{amsfonts}
\usepackage{amssymb}
\usepackage{centernot}
\usepackage{marvosym}
\let\marvosymLightning\Lightning
\newtheorem{theorem}{Theorem}
\newtheorem{corollary}{Corollary}[theorem]
\newtheorem*{remark}{Remark}
\renewcommand\qedsymbol{QED}
\newcommand{\N}{\mathbb{N}}
\newcommand{\Z}{\mathbb{Z}}
\newcommand{\divby}{%
  \mathrel{\text{\vbox{\baselineskip.65ex\lineskiplimit0pt\hbox{.}\hbox{.}\hbox{.}}}}%
  }
\newcommand{\notdivby}{\centernot\divby}
\title{\scalebox{2}{Math 341 Homework 8}}
\author{\scalebox{1.5}{Theo Koss}}
\date{October 2020}
\begin{document}
\maketitle
\section{Practice problems}
\subsection{Problem 8.3}
Let $f:X\to Y$ and $g:Y\to Z$ be maps, prove that if the composition $g\circ f$ is injective then $f$ is injective.
\begin{proof}
\newline Consider the injective composition $g\circ f:X\to Z$, and suppose $x_1\neq x_2$ this implies $g(f(x_1))\neq g(f(x_2))$. By the definition of a function, this implies that $f(x_1)\neq f(x_2)$, thus $f$ is an injection.
\end{proof}
\subsection{Problem 8.9}
Construct a bijection from $\N$ to $2\Z^+$ (the set of positive, even integers).
\begin{proof}
Consider a function $f:\N\to2\Z^+$, defined by $f(n)=2n$. This is the mapping, $(1\to2), (2\to4),(3\to6),...$
\newline \textbf{N2S} (need to show): $f$ is bijective, or $f$ is both injective and surjective.
\begin{remark}
Recall that a function $g:X\to Y$, is injective if $\forall x_1\neq x_2\in X$, $g(x_1)\neq g(x_2)$.\end{remark}
\begin{remark}
A function $g:X\to Y$, is surjective if $\forall y\in Y$, $\exists x\in X$ such that $g(x)=y$.\end{remark}
\begin{enumerate}
    \item Injectivity: $\forall n_1\neq n_2\in\N$, then the mapping is $f(n_1)=2n_1$ and $f(n_2)=2n_2$, if $f(n_1)=f(n_2)$ then $2n_1=2n_2\Longrightarrow2(n_1-n_2)=0$. However since the integers have no nontrivial zero divisors, $(n_1-n_2)$ must be equal to zero, therefore $n_1=n_2$. And since this is true, the contrapositive must be true (if $n_1\neq n_2$, then $f(n_1)\neq f(n_2)$), as required. Therefore the function $f$ is injective.
    \item Surjectivity: $\forall z\in 2\Z^+$, $\exists n\in\N$ such that $f(n)=z$. If we choose some arbitrary $z\in2\Z^+$, by definition, $z=2n$, and since $z$ is a positive, even integer, we can write this as $z=f(n)$. Therefore $f$ is surjective.
\end{enumerate}
\newline Since $f(n)=z$ is both injective and surjective, then it is a bijection from $\N\to2\Z^+$, As required.
\end{proof}
\subsection{Problem 8.10}
Prove that there is a bijection $\N\to\Z$.
\begin{proof}
Consider $f:\N\to\Z$, defined by $f(n)=
\left\{
\begin{array}{ll}
      k & n=2k\\
      -k & n=2k+1\\
\end{array}\right.$
\newline This maps $(1\to0),(2\to1),(3\to-2),(4\to2),...$
\begin{remark}
A function $f:X\to Y$ is injective iff $f(x)=f(y)\Longrightarrow x=y$.
\end{remark}
\newline \textbf{N2S}: Bijectivity \begin{enumerate}
    \item Injectivity: For some $n_1,n_2\in\N$, suppose $f(n_1)=f(n_2)$. Then there are 3 cases:
    \begin{enumerate}
        \item If they are both even, then $f(n_1)=k$, where $n_1=2k$. Also $f(n_2)=k$, where $n_2=2k$, it is clear that, $n_1=n_2=2k$.
        \item If they are both odd, then $f(n_1)=-k$, where $n_1=2k+1$. Also $f(n_2)=-k$, where $n_2=2k+1$, again it is clear that $n_1=n_2=2k+1$.
        \item If one is even and one is odd, then $f(n_1)=k$, where $n_1=2k$ and $f(n_2)=-k$, where $n_2=2k+1$. However in this case $n_1\neq n_2$, because one is $2k$, and one is $2k+1$. $2k\neq2k+1$, clearly. \scalebox{1.5}{\Lightning}
    \end{enumerate}
    \newline Therefore $f$ is an injection.
    \item Surjectivity: $\forall z\in\Z$, $\exists n\in\N$ s.t. $f(n)=z$. Take some $z\in\Z$, $z$ has a sign (positive or negative), or it is 0. \begin{enumerate}
        \item If $z$ is positive, then $z$ is produced by $f(2n)$. Ex. $(z=1,n=2$, $(z=2,n=4)$,...
        \item If $z$ is negative, then $z$ is produced by $f(2n+1)$. Ex. $(z=-1,n=3)$, $(z=-2,n=5)$,...
        \item If $z=0$, it is produced by $f(1)$.
    \end{enumerate}
    And since each $z$ has a unique producer in terms of $n$, the function $f$ is surjective.
\end{enumerate}
Since $f$ is both injective and surjective, then it is a bijection from $\N\to\Z$, as required.
\end{proof}
\end{document}