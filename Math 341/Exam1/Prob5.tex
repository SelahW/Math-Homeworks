\documentclass[12pt]{article}
\usepackage[utf8]{inputenc}
\usepackage{mathtools}
\usepackage{amsmath}
\usepackage{amsfonts}
\usepackage{amssymb}
\usepackage{amsthm}
\usepackage{centernot}
\newcommand{\notequiv}{\centernot\equiv}
\newcommand{\N}{\mathbb{N}}
\newcommand{\Z}{\mathbb{Z}}
\newtheorem{theorem}{Theorem}
\newcommand{\divby}{%
  \mathrel{\text{\vbox{\baselineskip.65ex\lineskiplimit0pt\hbox{.}\hbox{.}\hbox{.}}}}%
  }
\newcommand{\notdivby}{\centernot\divby}
\renewcommand\qedsymbol{QED}
\title{\scalebox{2}{Math 341 Exam 1}}
\author{\scalebox{1.5}{Theo Koss}}
\date{September 2020}
\begin{document}
\maketitle
\section{Problem 5}
Let $a,b,c\in\N$ and suppose $a>c$ and $b>c$. True or false: if $ab\divby c$, then $a\divby c$ or $b \divby c$. Prove.
\begin{proof}
This is true, since $ab\divby c \Longrightarrow ab=cn$, for $a\cdot b$ to be divisible by $c$, either $a$ is divisible by $c$, $b$ is divisible by $c$, or $c=\text{one of the prime factors of }ab$. If it's the first 2, obviously what we're trying to show is true. If it is the last result, then $c$ is one of the prime factors of $ab$, thus $a$ or $b$ share those same prime factors and thus are divisible by them, and since $a,b>c$, this must mean that $a$ or $b$ $\divby c$
\end{proof}
\end{document}