\documentclass[12pt]{article}
\usepackage[utf8]{inputenc}
\usepackage{mathtools}
\usepackage{amsmath}
\usepackage{amsfonts}
\usepackage{amssymb}
\usepackage{amsthm}
\usepackage{centernot}
\newcommand{\notequiv}{\centernot\equiv}
\newcommand{\N}{\mathbb{N}}
\newcommand{\Z}{\mathbb{Z}}
\newtheorem{theorem}{Theorem}
\newcommand{\divby}{%
  \mathrel{\text{\vbox{\baselineskip.65ex\lineskiplimit0pt\hbox{.}\hbox{.}\hbox{.}}}}%
  }
\newcommand{\notdivby}{\centernot\divby}
\renewcommand\qedsymbol{QED}
\title{\scalebox{2}{Math 341 Exam 1}}
\author{\scalebox{1.5}{Theo Koss}}
\date{September 2020}
\begin{document}
\maketitle
\section{Problem 3}
Prove if $a\divby c$ and $b\divby c$, then for any $x$ and $y$, $ax+by\divby c$.
\begin{proof}
We must show that $ax+by=cl$, for some $l\in\Z$. $a\divby c$, by definition means $a=cn, n\in\Z$, similarly, $b\divby c$ by definition means $b=cm, m\in\Z$. Then for any $x,y$, $x\cdot cn+y\cdot cm \divby c$, since we can factor out a $c$: $ax+by=c(xn+ym)$ let $l=xn+ym$, then we have: $ax+by=cl, l\in\Z$, which, by definition, means $ax+by\divby c$.
\end{proof}
\end{document}