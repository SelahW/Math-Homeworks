\documentclass[12pt]{article}
\usepackage[utf8]{inputenc}
\usepackage{mathtools}
\usepackage{amsthm}
\usepackage{amsmath}
\usepackage{amsfonts}
\usepackage{amssymb}
\usepackage{centernot}
\usepackage{marvosym}
\usepackage{enumitem}
\let\marvosymLightning\Lightning
\newtheorem{theorem}{Theorem}
\newtheorem{corollary}{Corollary}[theorem]
\newtheorem*{remark}{Remark}
\renewcommand\qedsymbol{QED}
\newcommand{\N}{\mathbb{N}}
\newcommand{\Z}{\mathbb{Z}}
\newcommand{\divby}{%
  \mathrel{\text{\vbox{\baselineskip.65ex\lineskiplimit0pt\hbox{.}\hbox{.}\hbox{.}}}}%
  }
\newcommand{\notdivby}{\centernot\divby}
\title{\scalebox{1.75}{Math 341 Final Project}}
\author{\scalebox{1.5}{Theo Koss}}
\date{December 2020}
\begin{document}
\maketitle
\section{Problem 14.1}
\begin{enumerate}[label=(\roman*)]
    \item Produce a pair of keys and a message (a number $n$).
    \begin{enumerate}[label=(\alph*)]
        \item $p=17$, $q=53$. $N=pq=901$.
        \item $\phi(N)=16\cdot52=832$.
        \item Let $e=3$, check $\gcd(e,p-1)=1$, and $\gcd(e,q-1)=1$. Therefore $\gcd(e,\phi(N))=1$.
        \item Find $d$ s.t. $ed\equiv1\mod{\phi(N)}$. $d=555$.\newline $ed=555\cdot3=1665\equiv1\mod{832}$.
        \item Public key=$(N,e)=(901,3)$. \newline Private key=$(N,d)=(901,555)$.
        \item Message: $n=99$.
    \end{enumerate}
    \item Encrypt the message.\newline $c\equiv n^e\mod{N}$. \newline $c\equiv 99^3=970299\equiv823\mod{901}$. Cyphertext: $c=823$.
    \item Decrypt the message.\newline $c^d\equiv n\mod{N}$. $$c^d=823^{555}=99^{1665}=99^{k(16)(52)+1}$$ In this case, $k=2$ because $2\cdot(16)\cdot(52)=1664$. According to Euler's Theorem, $$n^{\phi(N)}\equiv1\mod{N}$$ Also, in step (b) of part (i), we found $$\phi(N)=16\cdot52=832$$ So, $$n^{(2\cdot16\cdot52)+1}=\underbrace{(n^{1664})}_{\equiv1\mod{N}}\cdot(n^1)\mod{N}=n=99$$
\end{enumerate}
\end{document}
