\documentclass[hidelinks,12pt]{article}
\usepackage[utf8]{inputenc}
\usepackage{mathtools}
\usepackage{amsthm}
\usepackage{amsmath}
\usepackage{amsfonts}
\usepackage{amssymb}
\usepackage{centernot}
\usepackage{marvosym}
\usepackage{enumitem}
\usepackage{hyperref}
\usepackage{graphicx}
\graphicspath{{/home/theo/Documents/GitHub/Math-Homeworks/Images/}}
\setcounter{tocdepth}{1}
\let\marvosymLightning\Lightning
\renewcommand{\geq}{\geqslant}
\renewcommand{\leq}{\leqslant}
\newtheorem{theorem}{Theorem}
\newtheorem{corollary}{Corollary}[theorem]
\newtheorem*{remark}{Remark}
\newcommand{\ch}{\operatorname{ch}}
\newcommand{\R}{\mathbb{R}}
\newcommand{\N}{\mathbb{N}}
\newcommand{\Z}{\mathbb{Z}}
\newcommand{\Q}{\mathbb{Q}}
\newcommand{\F}{\mathbb{F}}
\newcommand{\divby}{%
  \mathrel{\text{\vbox{\baselineskip.65ex\lineskiplimit0pt\hbox{.}\hbox{.}\hbox{.}}}}%
  }
\newcommand{\notdivby}{\centernot\divby}
\title{\scalebox{2}{Math 835 Homework 1}}
\author{\scalebox{1.5}{Theo Koss}}
\date{September 2024}

\begin{document}

\maketitle
\section{Chapter 13}
\subsection{Chapter 2}
\begin{enumerate}
    \item Prove that if $\ch{\F}=p$, then $|\F|=p^n$. \begin{proof}
            Let $\F$ be a field with $\ch{\F}=p$. Consider the prime subfield $K<\F$, generated by $1_F$. From the book, $F\supset\langle1_F\rangle\cong\Z_p$ if $\ch{\F}=p$. Consider the vector space over $K$, and since $K$ has $p$ elements, we have finitely many choices for the basis. WLOG, choose a basis $b_1,b_2,\dots,b_n$. Then $[K:F]=\dim_FK=n$ and so $|\F|=p^n$.
    \end{proof}
\item[14.] Prove that if \([F(\alpha):F]\) is odd, then \(F(\alpha)=F(\alpha^2)\)
    \begin{proof}
        Let \([F(\alpha):F]\) be odd. Then the degree of the minimal polynomial of \(\alpha\) is odd. By way of contradiction, assume \(F(\alpha^2)\neq F(\alpha)\). In particular, \(\alpha\notin F(\alpha^2)\) Then the extension of \(F(\alpha)/F(\alpha^2)\) is quadratic, with minimal polynomial \(x^2-\alpha^2\). But this is a problem because by theorem 14, \[
            [F(\alpha):F]=[F(\alpha):F(\alpha^2)]\cdot[F(\alpha^2):F]
        \]
        But we assumed LHS is odd, and showed that \(F(\alpha)/F(\alpha^2)\) is quadratic. Contradiction. Therefore, \(\alpha\in F(\alpha^2)\). Which implies \(F(\alpha)=F(\alpha^2)\)
    \end{proof}
\item[18.] Let \(k\) be a field and let \(k(x)\) be the field of rational functions in \(x\) with coefficients from \(k\). Let \(t\in k(x)\) be the rational function \(\frac{P(x)}{Q(x)}\) with relatively prime polynomials \(P(x),Q(x)\in k[x]\), with \(Q(x)\neq0\). Then \(k(x)\) is an extension of \(k(t)\) and to compute its degree it is necessary to compute the minimal polynomial with coefficients in \(k(t)\) satisfied by \(x\). 
    \begin{enumerate}
        \item[(a).] Show that the polynomial \(P(X)-tQ(X)\) in the variable \(X\) and coefficients in \(k(t)\) is irreducible over \(k(t)\) and has \(x\) as a root.
            \[
                P(X)-tQ(X)=0\text{ is linear in }(k[X])[t]\text{ so it is irreducible}
            \]
            It also (trivially) has \(x\) as a root. We also have that \((k[X])[t]=(k[t])[X]\) so \(P(X)-tQ(X)\) is irreducible in \((k(t))[X]\).
        \item[(b).] Show that the degree of \(P(X)-tQ(X)\) as a polynomial in \(X\) with coefficients in \(k(t)\) is the maximum of the degrees of \(P(x)\) and \(Q(x)\).\begin{proof}
                Let \(n\in\N\) be the maximum of the degrees of \(P(x)\) and \(Q(x)\). So we may now write them as:
        \begin{align*}
            P(x)&=a_nx^n+\dots+a_1x+a_0\\
            Q(x)&=b_nx^n+\dots+b_1x+b_0
        \end{align*}
        Where \(a_n\) and \(b_n\) are not \emph{both} 0. Now we can analyse the leading term in \(P(X)-tQ(x)\), which is \(a_n-tb_n\). We must show \(a_n\neq tb_n\). This is true because \(t\in k(x)\) but \(t\notin k\), so \(t\) is some rational polynomial in \(x\), not a constant, and \(a_n\) is just a constant. So \(a_n-tb_n\neq0\) and therefore \(deg(P(X)-tQ(X))=n\) which we defined to be the maximum of the degrees.
        \end{proof}
        
    \item[(c).] Show that
        \[
            \left[k(x):k(t)\right]=\left[k(x):k\left(\frac{P(x)}{Q(x)}\right)\right]=\max(\deg P(x),\deg Q(x))
        \]
        \begin{proof}
            We have from part (a). that \(P(X)-tQ(X)\) is irreducible in \((k(t))[X]\) and has \(x\) as a root, so we can mod out by the polynomial to get \[
                k(x)\cong(k(t)[X])/\langle P(X)-tQ(X)\rangle
            \]
             By part (b), we have that the degree of this extension is the maximum of the degrees of the polynomials. So \[
                 [k(x):k(t)]=\max(\deg P(x),\deg Q(x))
             \]
              As required. 
        \end{proof}
    \end{enumerate}
\end{enumerate}
\end{document}
