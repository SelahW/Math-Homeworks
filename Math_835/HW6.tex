\documentclass[hidelinks,12pt]{article}
\usepackage[utf8]{inputenc}
\usepackage{mathtools}
\usepackage{amsthm}
\usepackage{amsmath}
\usepackage{amsfonts}
\usepackage{amssymb}
\usepackage{centernot}
\usepackage{marvosym}
\usepackage{enumitem}
\usepackage{hyperref}
\usepackage{graphicx}
\graphicspath{{/home/theo/Documents/GitHub/Math-Homeworks/Images/}}
\setcounter{tocdepth}{1}
\let\marvosymLightning\Lightning
\renewcommand{\geq}{\geqslant}
\renewcommand{\leq}{\leqslant}
\newtheorem{theorem}{Theorem}
\newtheorem{corollary}{Corollary}[theorem]
\newtheorem*{remark}{Remark}
\newcommand{\R}{\mathbb{R}}
\newcommand{\N}{\mathbb{N}}
\newcommand{\Z}{\mathbb{Z}}
\newcommand{\Q}{\mathbb{Q}}
\newcommand{\F}{\mathbb{F}}
\newcommand{\divby}{%
  \mathrel{\text{\vbox{\baselineskip.65ex\lineskiplimit0pt\hbox{.}\hbox{.}\hbox{.}}}}%
  }
\newcommand{\notdivby}{\centernot\divby}
\title{\scalebox{1.5}{Math 835 Homework 6}}
\author{\scalebox{1.5}{Theo Koss}}
\date{October 2024}

\begin{document}

\maketitle
\section*{Chapter 13}
\subsection*{Section 6}
\begin{enumerate}
    \item Suppose \(m\) and \(n\) are coprime positive integers. Let \(\zeta_{m}\) be a primitive \(m\)-th root of unity and let \(\zeta_{n}\) be a primitive \(n\)-th root of unity. Prove that \(\zeta_{m}\zeta_{n}\) is a primitive \(mn\)-th root of unity.
        \begin{proof}
        \(\zeta_{m}\) and \(\zeta_{n}\) are primitive resp. \(m\) and \(n\)-th roots of unity, so we have \(\zeta_{m}^{m}=1\) and \(\zeta^{n}_{n}=1\) and \(\zeta_{m}^{k}\neq1\), and \(\zeta_{n}^{l}\neq1\) for any \(1\leq k\leq m-1\) and \(1\leq l\leq n-1\). Then \[
            (\zeta_{m}\zeta_{n})^{mn}=1
        \]
        And because \(m\) and \(n\) are coprime \(\operatorname{lcm}(m,n)=mn\), so we do not have a smaller \(i\) for which \((\zeta_{m}\zeta_{n})^{i}=1\). Therefore \(\zeta_{m}\zeta_{n}\) is a primitive \(mn\)-th root of unity.
        \end{proof}
    \item Let \(\zeta_{n}\) be a primitive \(n-\)th root of unity and let \(d\) be a divisor of \(n\). Prove that \(\zeta_{n}^{d}\) is a primitive \(n/d\)-th root of unity.
        \begin{proof}
            Let \(q=n/d\), we have
            \[
                (\zeta_{n}^{d})^{q}=\zeta_{n}^{dq}=\zeta_{n}^{n}=1
            \]
            So \(\zeta_{n}^{d}\) is clearly a \(q-\)th root of unity. Now it remains to show that \(\zeta_{n}^{d}\) is primitive in \(\mu_{q}\). We have \(\zeta_{n}^{d}=\zeta_{q}\) and since 1 is coprime to \(q\), \(\zeta_{q}^{1}\) is primitive in \(\mu_{q}\).
        \end{proof}
    \item[13.] 
\end{enumerate}
\end{document}
