\documentclass[hidelinks,12pt]{article}
\usepackage[utf8]{inputenc}
\usepackage[table,xcdraw]{xcolor}
\usepackage{mathtools}
\usepackage{amsthm}
\usepackage{amsmath}
\usepackage{amsfonts}
\usepackage{amssymb}
\usepackage{centernot}
\usepackage{marvosym}
\usepackage{enumitem}
\usepackage{hyperref}
\setcounter{tocdepth}{1}
\let\marvosymLightning\Lightning
\newtheorem{theorem}{Theorem}
\newtheorem{corollary}{Corollary}[theorem]
\newtheorem*{remark}{Remark}
\renewcommand\qedsymbol{QED}
\newcommand{\C}{\mathbb{C}}
\newcommand{\R}{\mathbb{R}}
\newcommand{\N}{\mathbb{N}}
\newcommand{\Z}{\mathbb{Z}}
\newcommand{\Q}{\mathbb{Q}}
\newcommand{\divby}{%
  \mathrel{\text{\vbox{\baselineskip.65ex\lineskiplimit0pt\hbox{.}\hbox{.}\hbox{.}}}}%
  }
\newcommand{\notdivby}{\centernot\divby}
\title{\scalebox{2}{Math 531 Homework 5}}
\author{\scalebox{1.5}{Theo Koss}}
\date{February 2021}
\begin{document}
\maketitle
\section{Section 3.3}
\begin{itemize}
\item Problem 10: Construct a group of order 12 that is not abelian.\newline The dihedral group $D_6$, the symmetries of an regular hexagon.
\item Problem 12:\begin{enumerate}[label=(\alph*)]
    \item Let $C_1=\{(a,b)\in\Z\times\Z|a=b\}$. Show that $C_1$ is a subgroup of $\Z\times\Z$.\begin{proof}
    \begin{theorem}[Subgroup Test]\label{sgtest}Let $G$ be a group and let $H$ be a nonempty subset of $G$. If for all $a,b\in H$, $ab^{-1}\in H$, then $H\leqslant G$. \end{theorem}Proof of \ref{sgtest} \href{https://en.wikipedia.org/wiki/Subgroup_test}{\color{cyan}here}.
    Using the above theorem, consider two elements of $C_1$, $(a,b),(c,d)\in C_1$, N2S: $(a,b)-(c,d)\in C_1$. By definition, $a=b,c=d$, thus $a-c=b-d$, therefore $(a,b)-(c,d)=(a-c,b-d)\in C_1$. As required. Thus, $C_1$ is a subgroup.\end{proof}
    \item Show that $C_n=\{(a,b)\in\Z\times\Z|a\equiv b\mod{n}\}$ is a subgroup of $\Z\times\Z$.\begin{proof} Again using Theorem 1, consider 2 elements in $C_n:(a,b),(c,d)$, similarly, we N2S: $(a,b)-(c,d)\in C_n$. Since subtraction of congruence classes is well defined, this is true. ($a\equiv b\mod{n},c\equiv d\mod{n}$, so $a-c\equiv b-d\mod{n}=(a,b)-(c,d)\in C_n$).\end{proof}
    \item Show that every proper subgroup of $\Z\times\Z$ that contains $C_1$ has the form $C_n$, for some pos. int. $n$.
\end{enumerate}
\item Problem 14: Let $G_1$ and $G_2$ be groups, and let $G$ be the direct product $G_1\times G_2$. $H=\{(x_1,x_2)\in G_1\times G_2|x_2=e\}$ and $K=\{(x_1,x_2)\in G_1\times G_2|x_1=e\}$\begin{enumerate}[label=(\alph*)]
    \item Show that $H$ and $K$ are subgroups of $G$.\begin{proof}By the subgroup test, if $ab^{-1}\in H,K$, then $H\leqslant G$ and $K\leqslant G$.\newline $ab^{-1}$ in $H$ is the direct product of $(a,e)$ and $(b^{-1},e)$, which is equal to $(ab^{-1},e)\in H$. Therefore $H\leqslant G$. Similarly, $ab^{-1}$ in $K$ is the direct product of $(e,a)$ and $(e,b^{-1})$, which is $(e,ab^{-1})\in K$. Therefore $K\leqslant G$.
    \end{proof}
    \item Show that $HK=KH=G$.\begin{proof}The direct product $HK$ is defined as the direct product $(x_1,e)\times(e,x_2)=(x_1e,ex_2)=(x_1,x_2)=G$. Similarly, $KH$ is defined as the direct product $(e,x_2)\times(x_1,e)=(ex_1,x_2e)=(x_1,x_2)=G$.
    \end{proof}
    \item Show that $H\cap K=(e,e)$.\begin{proof}The intersection of two groups is the elements they share. In this case, the direct products $(x_1,e)$ and $(e,x_2)$ share one and only one element, namely when both $x_1=x_2=e$. This is the element $(e,e)$, and it occurs in both $H$ and $K$, and is the only element for which this is true.
    \end{proof}
\end{enumerate}
\end{itemize}
\section{Section 3.4}
\begin{itemize}
    \item Problem 4: Show that $\Z^{\times}_{10}$ is isomorphic to the additive group $\Z_4$.\begin{proof}We have an isomorphism $\phi:\Z_4\to\Z^{\times}_{10}$ via the mapping:\begin{itemize}[label=\textbullet]
        \item $0\to1$
        \item $1\to3$
        \item $2\to5$
        \item $3\to7$
    \end{itemize}
    Since $3$ is a generator of $\Z_{10}^{\times}$, $\phi$ is a surjection. and since we laid out the mapping, it is easy to see that $\phi$ is an injection as well. (I just realized this is problem 1, whoops!)
    \end{proof}
    \item Problem 4: Show that $\Z^{\times}_5$ is not isomorphic to $\Z^{\times}_8$ by showing that the first group has an element of order 4 but the second does not.\begin{proof}$\Z^{\times}_5$:\begin{itemize}[label=\textbullet]
        \item 1, $|1|=1$
        \item 2, $|2|=4$
        \item 3, $|3|=4$
        \item 4, $|4|=2$
    \end{itemize}$\Z^{\times}_8$:\begin{itemize}[label=\textbullet]
        \item 1, $|1|=1$
        \item 2, $|2|=\infty$
        \item 3, $|3|=2$
        \item 4, $|4|=\infty$
        \item 5, $|5|=2$
        \item 6, $|6|=\infty$
        \item 7, $|7|=2$
    \end{itemize}
    \end{proof}
    \item Problem 5: Show that the group $(\Q,+)$ is not isomorphic to the group $(\Q^+,\cdot)$.\begin{proof}The group $(\Q,+)$ has one, and only one, element of finite order, namely 0. $\Q^+,\cdot$ on the other hand, has two elements of finite order, $-1,1$. Therefore, by proposition 3.4.3 in the book, there exists no isomorphism between these two.
    \end{proof}
    \item Problem 9: Prove that any group with three elements must be isomorphic to $\Z_3$.\begin{proof}
    Consider some arbitrary group $(G,*)$ with 3 distinct elements. Then it is safe to say that the elements are $G=\{e,a,b\}$. Since $$ab=a\implies b=e\text{ and }ab=b\implies a=e$$ we can conclude that $$ab=e$$ Also, since $$a^2=a\implies a=e\text{ and }a^2=e\implies a=e$$ we can conclude that $$a^2=b$$ and thus $$a^3=a^2a=ba=e$$ Therefore, $a$ generates the whole set, so $G$ is cyclic, and since $|G|=3$, $(G,+)$ must be isomorphic to $\Z_3$.
    \end{proof}
    \item Problem 14: Let $G$ be the following matrices over $\R$.\newline $e=\begin{bmatrix}
1 & 0\\
0 & 1
\end{bmatrix}$,$a=\begin{bmatrix}
-1 & 0\\
0 & 1
\end{bmatrix}$,$b=\begin{bmatrix}
1 & 0\\
0 & -1
\end{bmatrix}$,$c=\begin{bmatrix}
-1 & 0\\
0 & -1
\end{bmatrix}$.\newline Show that $G$ is isomorphic to $\Z_2\times\Z_2$.\begin{proof}It is easy to see that $|e|=1$, $|b|=2$, $|c|=2$, $|d|=2$. The group $\Z_2\times\Z_2$ has $4$ elements, if their orders are $1,2,2,2$, then there exists an isomorphism, by proposition 3.4.3 in the book. The elements of $\Z_2\times\Z_2$ are as follows:$$\Z_2\times\Z_2=\{(0,0),(1,0),(0,1),(1,1)\}$$We check: \newline $|e=(0,0)|=1$. \newline $|(1,0)|=2,$ since $(1,0)+(1,0)=(2,0)=(0,0)\mod2=e$.\newline $|(0,1)|=2,$ since $(1,0)+(1,0)=(2,0)=(0,0)\mod2=e$.\newline $|(1,1)|=2$, since $(1,1)+(1,1)=(2,2)=(0,0)\mod2=e$.\newline Thus, by proposition 3.4.3 in the book, $G$ is isomorphic is $\Z_2\times\Z_2$.\end{proof}
\end{itemize}
\end{document}