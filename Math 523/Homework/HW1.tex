\documentclass[hidelinks,12pt]{article}
\usepackage[utf8]{inputenc}
\usepackage{mathtools}
\usepackage{amsthm}
\usepackage{amsmath}
\usepackage{amsfonts}
\usepackage{amssymb}
\usepackage{centernot}
\usepackage{marvosym}
\usepackage{enumitem}
\usepackage{hyperref}
\setcounter{tocdepth}{1}
\let\marvosymLightning\Lightning
\renewcommand{\geq}{\geqslant}
\renewcommand{\leq}{\leqslant}
\newtheorem{theorem}{Theorem}
\newtheorem{corollary}{Corollary}[theorem]
\newtheorem*{remark}{Remark}
\renewcommand\qedsymbol{QED}
\newcommand{\R}{\mathbb{R}}
\newcommand{\N}{\mathbb{N}}
\newcommand{\Z}{\mathbb{Z}}
\newcommand{\Q}{\mathbb{Q}}
\newcommand{\divby}{%
  \mathrel{\text{\vbox{\baselineskip.65ex\lineskiplimit0pt\hbox{.}\hbox{.}\hbox{.}}}}%
  }
\newcommand{\notdivby}{\centernot\divby}
\title{\scalebox{2}{Math 523 Homework 1}}
\author{\scalebox{1.5}{Theo Koss}}
\date{September 2023}

\begin{document}

\maketitle

\section*{Section 1.3}
\begin{itemize}
    \item[2.] Use induction to prove the following:\begin{enumerate}
        \item[\bf{f.}] If $x\in(0, 1)$ is a fixed real number, then $0 <x^n < 1$ for all $n\in\N$.\begin{proof}
            Let $x\in(0,1)$ and $n\in\N$ \newline Base case: $n=1$, $x^1=x\in(0,1)\implies 0<x<1$ by definition.\newline Inductive step: Suppose $0<x^k<1$ for some $k\in\N$. Consider $n=k+1$, then $$x^{k+1}=x^k\cdot x$$ By induction hypothesis, $$0<x^k<1$$ multiply by $x$: $$0<x^{k+1}<x$$ Since $x<1$ (by assumption) $$0<x^{k+1}<1$$
        \end{proof}
        \item[\bf{h.}]$2^n<n!$ for all natural numbers $n\geq4$.\begin{proof}
            Let $n\geq4\in\N$.\newline Base case: $n=4$, $$2^4=16,\  4!=24,\  16<24 \checkmark$$
            Inductive step: Suppose $2^k<k!$ for some $k\geq4$. Consider $k+1$. Then $$2^{k+1}=2^k+2^K$$\begin{align*}
                (k+1)!&=k\cdot k!+k!\\
                &<k\cdot2^k+2^k\tag{By hypothesis}\\
                &<2^k+2^k\\
                &=2^{k+1}\\
            \end{align*}So $$(k+1)!>2^{k+1}$$
        \end{proof}
        \item[\bf{j.}] $\frac{n^5}{5}+\frac{n^3}{3}+\frac{7n}{15}$ is an integer for every $n\in\N$.\begin{proof}
            Let $n\in\N$.\newline Base case: $n=1$ $$\frac{1}{5}+\frac{1}{3}+\frac{7}{15}=\frac{3}{15}+\frac{5}{15}+\frac{7}{15}=1$$ Inductive step: Suppose the expression is an integer for some $k\in\N$ consider $k+1$. Then we seek $$\frac{(k+1)^5}{5}+\frac{(k+1)^3}{3}+\frac{7(k+1)}{15}$$ to be an integer. i.e. we want $$3(k+1)^5+5(k+1)^3+7(n+1)=15j$$ for some $j\in\N$. (Divisible by 15) \newline Expanding: $$(3k^5+5k^3+7k)+15(k^4+2k^3+3k^2+2k+1)$$ By induction assumption the first part is divisible by $15$, and we factored out a 15 from second part so that part is clearly divisible by 15. (Sum of 2 things that are divisible by $n$ is divisible by $n$ trivially, but just in case: Suppose $a,b$ are both divisible by $n$ $$a=x\cdot n,\ b=y\cdot n$$ so $$a+b=(x+y)\cdot n$$)
        \end{proof}
    \end{enumerate}
    \item[5.]\begin{enumerate}
        \item[\bf{a.}]Prove $$1+nx\leq(1+x)^n$$ for all $n\in\N$ with $x\geq-1$ a fixed real.\begin{proof}
            Let $n\in\N$ and $x\geq-1\in\R$.\newline Base case: $1+x=1+x\checkmark$\newline Inductive step: Assume $1+kx\leq(1+x)^k$
            \begin{align*}
                (1+x)^{k+1}&=(1+x)(1+x)^k\\
                &\geq(1+x)(1+kx)\\
                &=1+(k+1)x+kx^2\\
                &\geq1+(k+1)x\\
            \end{align*}
        \end{proof}
        \item[\bf{b.}] Use binomial theorem to prove $$1+nx\leq(1+x)^n$$ for all $n\in\N$ with $x\geq0$ a fixed real.\begin{proof}
            By binomial theorem, RHS becomes: $$(1+x)^n=\sum_{k=0}^n\binom{n}{k}x^{n-k}$$ Simplifying: $$=x^n+\binom{n}{n-1}x^{n-1}+\dots+\binom{n}{2}x^2+nx+1$$ LHS becomes: $$1+nx$$ Since $x\geq0$, every term in RHS is non-negative. Therefore:\begin{align*}
                (1+x)^n&=x^n+\binom{n}{n-1}x^{n-1}+\dots+\binom{n}{2}x^2+nx+1\\
                &=1+nx+\ell\\
                &\geq1+nx\\
            \end{align*} So $1+nx\leq(1+x)^n$
        \end{proof}
    \end{enumerate}
\end{itemize}
\section*{Section 1.4}
\begin{itemize}
    \item[2.] Prove that if $q^2$ is divisible by 3, then so is $q$.\begin{proof}
        We will prove the contrapositive, assume $3\centernot|q$. That is, $q\equiv1$ or $q\equiv2 \mod{3}$.\begin{itemize}
            \item If $q\equiv1\mod{3}$, then $q^2\equiv1\mod{3}\implies3\centernot|q^2$.
            \item If $q\equiv2\mod{3}$, then $q^2\equiv1\mod{3}\implies3\centernot|q^2$.
        \end{itemize}
    \end{proof}
    \item[4.]\begin{enumerate}
        \item[\bf{a.}] $\sqrt{3}$\begin{proof}
            By way of contradiction, assume $\sqrt{3}=\frac{a}{b}$ for coprime $a,b\in\Z$. Then $3=\frac{a^2}{b^2}\implies 3b^2=a^2$, thus we have shown $a^2$ is divisible by 3, therefore so is $a$. We also have $3b^2=(3k)^2=9k^2$ (subbing in $3k$ for $a$) so $b$ is also divisible by $3$. \marvosymLightning
        \end{proof}
        \item[\bf{b.}] $\sqrt{6}$\begin{proof}
        By way of contradiction, assume $\sqrt{6}=\frac{a}{b}$ for coprime $a,b\in\Z$. Then $6=\frac{a^2}{b^2}\implies 6b^2=a^2$, thus we have shown $a^2$ is divisible by 6, therefore so is $a$. We also have $6b^2=(6k)^2=36k^2$ so $b$ is also divisible by 6. \marvosymLightning\end{proof}
        \item[\bf{c.}] $\sqrt[3]{2}$\begin{proof}
            By way of contradiction, assume $\sqrt[3]{2}=\frac{a}{b}$ for coprime $a,b\in\Z$. Then $2b^3=a^3$, so $a^3$ is even, therefore $a$ is even ($0^3\mod2=0$). So there is some $k\in\Z$ such that $2b^3=(2k)^3\implies b^3=4k^3$ so $b$ is even. \marvosymLightning
        \end{proof}
        \item[\bf{d.}] $\sqrt{2}+\sqrt{3}$\begin{proof}
            By way of contradiction, assume $\sqrt{2}+\sqrt{3}$ is rational, then so is $(\sqrt{2}+\sqrt{3})^2=5+2\sqrt{6}$. That would imply $\sqrt{6}\in\Q$, which we have disproven above.
        \end{proof}
    \end{enumerate}
    \item[5.] Consider the statement $P$: the sum of two irrational numbers is irrational.\begin{enumerate}
        \item[\bf{a.}] Give example where $P$ is true. \newline As seen above: $$\sqrt{2}+\sqrt{3}\notin\Q$$
        \item[\bf{b.}] Prove or disprove $P$ by giving counterexample. $$\sqrt{2}+(-\sqrt{2})=0\in\Q$$
    \end{enumerate}
\end{itemize}
\section*{Additional Problems}
\begin{enumerate}
    \item Let $|X|=n$ for some $n\geq0$, and choose an integer $k$ with $0\leq k\leq n$. Let $A$ be the collection of all subsets of $X$ with $k$ elements. Let $B$ be the collection of all subsets of $X$ with $n-k$ elements. Find a one to one correspondence $f:A\to B$. Conclude that $A$ and $B$ have the same number of elements.\begin{proof}
        Define $$f:A\to B$$ as $$Y_{i}\mapsto X\setminus Y_i$$ Where $A=\{Y_1,Y_2,\dots,Y_r\}$, the $Y$'s are subsets of $X$ with $k$ elements. As required, the subsets $X\setminus Y_i$ have order $n-k$. Since $f$ is a bijection, $|A|=|B|\implies\binom{n}{k}=\binom{n}{n-k}$, i.e. the number of ways to choose $k$ things from $n$ is equal to the number of ways to choose $n-k$ things from $n$.\par\null\par Intuitively, when you choose $k$ things from $n$, you're also \emph{not} choosing $n-k$ things from $n$.
    \end{proof}
    \item Prove following with Binomial Theorem $$(x+y)^n=\sum_{k=0}^n\binom{n}{k}x^{n-k}y^k$$\begin{enumerate}
        \item[\bf{a.}] $$2^n=\sum_{k=0}^n\binom{n}{k}=\binom{n}{0}+\binom{n}{1}+\dots+\binom{n}{n}$$ for any integer $n\geq0$.\begin{proof}
            Let $n\geq0\in\Z$, using the binomial theorem, let $x=1$ and $y=1$. Then $$(1+1)^n=2^n=\sum_{k=0}^n\binom{n}{k}1^{n-k}1^k=\sum_{k=0}^n\binom{n}{k}$$
        \end{proof}
        \item[\bf{b.}]$$\sum_{k=0}^n(-1)^k\binom{n}{k}=0$$ for any integer $n\geq0$.\begin{proof}
            Let $n\geq0\in\Z$, using the binomial theorem, let $x=1$ and $y=-1$. Then $$(1-1)^n=0=\sum_{k=0}^n\binom{n}{k}1^{n-k}(-1)^k=\sum_{k=0}^n(-1)^k\binom{n}{k}$$
        \end{proof}
    \end{enumerate}
\end{enumerate}
\end{document}