\documentclass[hidelinks,12pt]{article}
\usepackage[utf8]{inputenc}
\usepackage{mathtools}
\usepackage{amsthm}
\usepackage{amsmath}
\usepackage{amsfonts}
\usepackage{amssymb}
\usepackage{centernot}
\usepackage{marvosym}
\usepackage{enumitem}
\usepackage{hyperref}
\setcounter{tocdepth}{1}
\let\marvosymLightning\Lightning
\renewcommand{\geq}{\geqslant}
\renewcommand{\leq}{\leqslant}
\newtheorem{theorem}{Theorem}
\newtheorem{corollary}{Corollary}[theorem]
\newtheorem*{remark}{Remark}
\newcommand{\R}{\mathbb{R}}
\newcommand{\N}{\mathbb{N}}
\newcommand{\Z}{\mathbb{Z}}
\newcommand{\Q}{\mathbb{Q}}
\newcommand{\C}{\mathbb{C}}
\newcommand{\divby}{%
    \mathrel{\text{\vbox{\baselineskip.65ex\lineskiplimit0pt\hbox{.}\hbox{.}\hbox{.}}}}%
}
\newcommand{\notdivby}{\centernot\divby}
\title{\scalebox{1.5}{Math 723 Homework 3}}
\author{\scalebox{1.5}{Theo Koss}}
\date{September 2024}

\begin{document}
\maketitle
\section*{Assignment 3}
\begin{enumerate}
    \item Show that \[
            |\frac{f_{n+1}}{f_{n}}-\varphi|=\frac{1}{f_{n}}\cdot \frac{1}{\varphi^{n+1}}\text{ and } \lim_{n\to\infty}\frac{f_{n+1}}{f_{n}}=\varphi
    \] 
    Where \(\varphi\) is the unique positive root of \(\varphi^{2}-\varphi-1=0\).
    \begin{proof}
        Induction base case:
        \[
            |\frac{f_{1}}{f_{0}}-\varphi|=\frac{1}{f_{0}}\cdot \frac{1}{\varphi^{1}}
        \]
        \[
        \implies |1-\varphi|=\frac{1}{\varphi}\implies \varphi^{2}-\varphi-1=0
        \]
       Which is exactly as we have defined \(\varphi\). Assume
       \[
           |\frac{f_{k+1}}{f_{k}}-\varphi|=\frac{1}{f_{k}}\cdot \frac{1}{\varphi^{k+1}}
       \]
       Then, we have \begin{align*}
       |\frac{f_{k+2}}{f_{k+1}}-\varphi|&=|\frac{f_{k+1}+f_{k}}{f_{k+1}}-\varphi|\\
                                        &=|1+\frac{f_{k}}{f_{k+1}}-\varphi|\\
                                        &=|1+(\varphi+\frac{1}{f_{k}}\cdot \frac{1}{\varphi^{k+1}})-\varphi|\\
                                        &=|1+\frac{1}{f_{k}}\cdot \frac{1}{\varphi^{k+1}}|\\
                                        &=\frac{1}{f_{k+1}}\cdot \frac{1}{\varphi^{k+2}}\tag{\(1=\varphi-\frac{1}{\varphi}\) }\\
    \end{align*}
      Then, since this equality holds and the RHS \(\to0\) as \(n\to\infty\), we have that \(\lim_{n\to\infty}\frac{f_{n+1}}{f_{n}}=\varphi\)  
    \end{proof}
    \item With the Fibonacci numbers, define \(f(z)=\sum_{n=0}^{\infty}f_{n}z^{n}\). Show that this power series has radius of convergence \(R=\frac{1}{\varphi}\), and for \(|z|<R\), \[
            f(z)=\frac{1}{1-z-z^{2}}
    \]
   \begin{proof}
   By the Cauchy-Hadamard Theorem, we have \[
   \frac{1}{R}=\limsup_{n\to\infty}|a_{n}|^{1/n}
   \]
  We have \(a_{n}=f_{n}\), and by above, \(\limsup_{n\to\infty}|f_{n}|^{1/n}=\varphi\). Thus \(R=\frac{1}{\varphi}\).\\
  Let \(|z|<R\), then \[
      f(z)=\sum_{n=0}^{\infty}f_{n}z^{n}=1+f_{1}z+\sum_{n=2}^{\infty}(f_{n-1}+f_{n-2})z^{n}
  \]
 Split the sum:
\[
=1+f_{1}z+\sum_{n=2}^{\infty}a_{n-1}z^{n}+\sum_{n=2}^{\infty}a_{n-2}z^{n}
\]
Pull out \(z\) from the first sum, and \(z^{2}\) from the second:
\[
=1+f_{1}z+z \sum_{n=2}^{\infty}a_{n-1}z^{n-1}+z^{2}\sum_{n=2}^{\infty}a_{n-2}z^{n-2}
\]
Now each sum is exactly \(f(z)\), so \[
=1+zf(z)+z^{2}f(z)\implies f(z)=\frac{1}{1-z-z^{2}}
\]
As required
   \end{proof} 
    \item The generalized binomial coefficient for \(z\in\C\) and \(k=0,1,2,\dots\) is 1 if \(k=0\) and \[
            \binom{z}{k}=\frac{z(z-1)\cdots (z-k+1)}{k!}
    \]
    Find the radius of convergence of \(B_{s}(z)=\sum_{n=0}^{\infty}\binom{s}{n}z^{n}\). 
    \begin{proof}
   Apply the ratio test: \[
       \frac{a_{n+1}}{a_{n}}=\frac{\binom{s}{n+1}z^{n+1}}{\binom{s}{n}z^{n}}=\frac{s-n}{n+1}z
   \]
   Then \[
   \lim_{n\to\infty}|\frac{a_{n+1}}{a_{n}}|=\lim_{n\to\infty}\left| \frac{s-n}{n+1}z\right|=|z|
   \]
   Since \(\frac{s-n}{n+1}\) converges to 1 for fixed \(s\).\\ Therefore to converge we require \(|z|<1\). So \(R=1\).
    \end{proof}
    \item For every sequence \(\mathbf{a}=(a_{n})\in\{0,2\}^{\N}\) define \(x(\mathbf{a})=\sum_{n=1}^{\infty}\frac{a_{n}}{3^{n}}\). Show that the set of all such values is the cantor set. 
        \begin{proof}
            Recall the Cantor set is defined by first taking \([0,1]\), then removing the middle third, \((\frac{1}{3},\frac{2}{3})\). Then we get two parts: \([0,\frac{1}{3}]\cup[\frac{2}{3},1]\).\\
            At \(n=1\), the sum can take two values, \(0\text{ or }\frac{2}{3}\). The value 0 is in the first part, and 2/3 is in the second part. Then at level 2, we have 4 options: 
            \begin{itemize}
                \item \( (0,0)\mapsto0\) 
                \item \((0,2)\mapsto \frac{2}{9}\) 
                \item \((2,0)\mapsto \frac{2}{3}\) 
                \item \((2,2)\mapsto \frac{2}{3}+\frac{2}{9}=\frac{8}{9}\) 
            \end{itemize}All of which lie in the four parts in the second layer (in fact, one in each), which are
            \[
                \left[0,\frac{1}{9}\right]\cup\left[\frac{2}{9},\frac{1}{3}\right]\cup\left[\frac{2}{3},\frac{7}{9}\right]\cup\left[\frac{8}{9},1\right]
            \]
            Each segment after \(n\) steps has length \(\frac{1}{3^{n}}\), and there are \(2^{n}\) of them. A value of \(0\) keeps you in the segment you are in, while a value of 2 moves you over to the next segment (of the Cantor set), skipping over the middle third. Therefore there is no way to get outside of the set with a sequence of 0's and 2's. And, each sequence \(\mathbf{a}\) corresponds to a specific value of the cantor set, by choosingleft or right (0 or 2) however many times. 
    \end{proof}
\end{enumerate}
\end{document}
