\documentclass[hidelinks,12pt]{article}
\usepackage[utf8]{inputenc}
\usepackage{mathtools}
\usepackage{amsthm}
\usepackage{amsmath}
\usepackage{amsfonts}
\usepackage{amssymb}
\usepackage{centernot}
\usepackage{marvosym}
\usepackage{enumitem}
\usepackage{hyperref}
\setcounter{tocdepth}{1}
\let\marvosymLightning\Lightning
\renewcommand{\geq}{\geqslant}
\renewcommand{\leq}{\leqslant}
\newtheorem{theorem}{Theorem}
\newtheorem{corollary}{Corollary}[theorem]
\newtheorem*{remark}{Remark}
\newcommand{\R}{\mathbb{R}}
\newcommand{\N}{\mathbb{N}}
\newcommand{\Z}{\mathbb{Z}}
\newcommand{\Q}{\mathbb{Q}}
\newcommand{\divby}{%
    \mathrel{\text{\vbox{\baselineskip.65ex\lineskiplimit0pt\hbox{.}\hbox{.}\hbox{.}}}}%
}
\newcommand{\notdivby}{\centernot\divby}
\title{\scalebox{2}{Math 724 Homework 1}}
\author{\scalebox{1.5}{Theo Koss}}
\date{September 2024}

\begin{document}

\maketitle
\section{Chapter 1}
\begin{enumerate}
    \item[1.] If $r\neq0$ is rational, and $x$ is irrational, prove that $r+x$ and $rx$ are irrational.
        \begin{proof}
            Let $r\neq0$ be rational and $x$ be irrational.
            \begin{enumerate}[label=(\alph*)]
                \item Assume, by way of contradiction, that $r+x$ is rational. Then $r+x=\frac{p}{q}$ for $p,q\in\Z$ and $q\neq0$. We have that $r$ is rational, so $r=\frac{a}{b}$ for some $a,b\in\Z$ and $b\neq0$. Then we have \[x=\frac{p}{q}-\frac{a}{b}=\frac{bp-aq}{bq}\] since the ring $(\Z,+,\cdot)$ is closed under addition and multiplication so numerator and denominator are integers, and since $\Z$ is a domain, $b\neq0$ and $q\neq0\implies bq\neq0$. So then we have $x\in\Q$ which is a contradiction.
                \item Assume, by way of contradiction, that $rx=\frac{p}{q}$ for some $p,q\in\Z$ and $q\neq0$. We have $r=\frac{a}{b}$ for $a,b\in\Z$ and $b\neq0$. Then \[x=\frac{rx}{r}=\frac{\frac{p}{q}}{\frac{a}{b}}=\frac{p}{q}\cdot\frac{b}{a}=\frac{pb}{qa}\] Again since $\Z$ is closed under multiplication and $r\neq0$ by assumption, we have $a\neq0$ and therefore $qa\neq0$, so we are not dividing by 0 and $x\in\Q$ contradiction.
            \end{enumerate} 
        \end{proof}
    \item[17.] Prove that \[
        |x+y|^2+|x-y|^2=2|x|^2+2|y|^2 
    \]
    if $x\in\R^k$ and $y\in\R^k$.
    \begin{proof}
    \[
        |x+y|^2=|x|^2+2|xy|+|y|^2,\ |x-y|^2=|x|^2-2|xy|+|y|^2 \tag{By FOIL}
    \]
    So
    \[
        |x+y|^2+|x-y|^2=2|x|^2+2|y|^2
    \]
    As required. Geometrically, the LHS is the sum of squares of the diagonals ($x+y$ is the long diag, $x-y$ is the short). RHS is 2 times sum of squares of the side lengths. The parallelogram has corners $0,x,y,\text{ and }x+y$.
    \end{proof}
\end{enumerate}
\section{Chapter 2}
\begin{enumerate}
    \item[2.] Prove that the set of all algebraic numbers is countable.
        \begin{proof}
            For every positive $N\in\Z$, we have that the set of
            \[
                n+|a_0|+|a_1|+\dots+|a_n|=N
            \]
            is finite. (compositions of $N$?) By theorem 2.12 in the text,
            \[
                S=\bigcup_{n=1}^{\infty}E_n
            \]
            is countable if $E_n$ is a sequence of countable sets. For every positive integer $N$, we have a countable (in fact finite) set of terms which satisfy the above sum. Therefore, since the index set $\Z$ is countable, the union of these sets is countable. Then we biject each set of terms to an algebraic number, so the set of algebraic numbers is at most countable.
        \end{proof}
    \item[12.] Let \(K\subset R^1\) consist of \(0\) and the numbers \(1/n\), for \(n=1,2,3,\dots\). Prove that \(K\) is compact directly from the definition. (Without the Heine-Borel theorem).  
        \begin{proof}
            We must show that every open cover of \(K\) contains a finite subcover. Any open cover \(\{G_{\alpha}\}\) must contain 0, in particular one of the sets of the cover contains 0. Then, we consider a neighboorhood around 0, say \(G_0=(-\varepsilon,\varepsilon)\) for some \(\varepsilon\in\R\). This neighboorhood contains infinitely many points, and outside of the neighboorhood there are only finitely many (the maximum is \(1/1=1\)). So we may enumerate them, say \(x_1,x_2,\dots,x_m\) for some \(m\in\Z\). Then, consider the open set containing \(x_1\), then another open set containing \(x_2\), do this for all \(x_m\). Since there are finitely many, there are finitely many open sets, and clearly their union is \(K\), so \(K\) is compact.  
        \end{proof}
\end{enumerate}
\end{document}
