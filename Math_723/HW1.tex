\documentclass[hidelinks,12pt]{article}
\usepackage[utf8]{inputenc}
\usepackage{mathtools}
\usepackage{amsthm}
\usepackage{amsmath}
\usepackage{amsfonts}
\usepackage{amssymb}
\usepackage{centernot}
\usepackage{marvosym}
\usepackage{enumitem}
\usepackage{hyperref}
\setcounter{tocdepth}{1}
\let\marvosymLightning\Lightning
\renewcommand{\geq}{\geqslant}
\renewcommand{\leq}{\leqslant}
\newtheorem{theorem}{Theorem}
\newtheorem{corollary}{Corollary}[theorem]
\newtheorem*{remark}{Remark}
\renewcommand\qedsymbol{QED}
\newcommand{\R}{\mathbb{R}}
\newcommand{\N}{\mathbb{N}}
\newcommand{\Z}{\mathbb{Z}}
\newcommand{\Q}{\mathbb{Q}}
\newcommand{\divby}{%
  \mathrel{\text{\vbox{\baselineskip.65ex\lineskiplimit0pt\hbox{.}\hbox{.}\hbox{.}}}}%
  }
\newcommand{\notdivby}{\centernot\divby}
\title{\scalebox{2}{Math 724 Homework 1}}
\author{\scalebox{1.5}{Theo Koss}}
\date{September 2024}

\begin{document}

\maketitle
\section{Chapter 1}
\begin{enumerate}
    \item[1.] If $r\neq0$ is rational, and $x$ is irrational, prove that $r+x$ and $rx$ are irrational.
        \begin{proof}
            Let $r\neq0$ be rational and $x$ be irrational.
            \begin{enumerate}[label=(\alph*)]
                \item Assume, by way of contradiction, that $r+x$ is rational. Then $r+x=\frac{p}{q}$ for $p,q\in\Z$ and $q\neq0$. We have that $r$ is rational, so $r=\frac{a}{b}$ for some $a,b\in\Z$ and $b\neq0$. Then we have \[x=\frac{p}{q}-\frac{a}{b}=\frac{bp-aq}{bq}\] since the ring $(\Z,+,\cdot)$ is closed under addition and multiplication so numerator and denominator are integers, and since $\Z$ is a domain, $b\neq0$ and $q\neq0\implies bq\neq0$. So then we have $x\in\Q$ which is a contradiction.
    \item Assume, by way of contradiction, that $rx=\frac{p}{q}$ for some $p,q\in\Z$ and $q\neq0$. We have $r=\frac{a}{b}$ for $a,b\in\Z$ and $b\neq0$. Then \[x=\frac{rx}{r}=\frac{\frac{p}{q}}{\frac{a}{b}}=\frac{p}{q}\cdot\frac{b}{a}=\frac{pb}{qa}\] Again since $\Z$ is closed under multiplication and $r\neq0$ by assumption, we have $a\neq0$ and therefore $qa\neq0$, so we are not dividing by 0 and $x\in\Q$ contradiction.
           \end{enumerate} 
        \end{proof}
\end{enumerate}
\end{document}
