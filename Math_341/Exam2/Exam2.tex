\documentclass[12pt]{article}
\usepackage[utf8]{inputenc}
\usepackage{mathtools}
\usepackage{amsthm}
\usepackage{amsmath}
\usepackage{amsfonts}
\usepackage{amssymb}
\usepackage{centernot}
\newtheorem{theorem}{Theorem}
\newtheorem{corollary}{Corollary}[theorem]
\newtheorem*{remark}{Remark}
\renewcommand\qedsymbol{QED}
\newcommand{\N}{\mathbb{N}}
\newcommand{\Z}{\mathbb{Z}}
\newcommand{\divby}{%
  \mathrel{\text{\vbox{\baselineskip.65ex\lineskiplimit0pt\hbox{.}\hbox{.}\hbox{.}}}}%
  }
\newcommand{\notdivby}{\centernot\divby}
\title{\scalebox{2}{Math 341 Exam 2}}
\author{\scalebox{1.5}{Theo Koss}}
\date{October 2020}
\begin{document}
\maketitle
\section{Problem 1}
Find $\gcd(2322,654)$ using Euclid's Algorithm.
$$2322=654\cdot3+360$$ $$654=360\cdot1+294$$ $$360=294\cdot1+66$$ $$294=66\cdot4+30$$ $$66=30\cdot+6$$ $$30=6\cdot5+0$$ Therefore $\gcd(2322,654)=6$.
\section{Problem 2}
Prove that for any $n\geq2$, the numbers $n!+2,n!+3,...,n!+n$ are composite.
\begin{proof}
\begin{remark} Any number $n!\in\N$ can be written as $n!=2*3*4*...*n$.\end{remark}
\begin{remark} Any number $n!$ where $n\geq2$ is even, since it will always be something times 2. \end{remark}
For any number $n\geq2$, we may write $n!=2*3*4*...*n$. $n$ is either even or odd.
\begin{enumerate}
    \item $\forall n$, $n!+x$, where $x\in\{2,4,...,n\}$, is always even, and therefore composite, since $n!$ is even, by the remark above, so you are always able to factor out a 2 from $n!+x$, thus $n!+x$ is composite.
    \item $\forall n$, $n!+y$, where $y\in\{3,5,...,n-1\}$, is always composite, because if you rearrange $n!$, and factor out $y$, you get $n!=y(z)$. Where $z=\frac{n!}{y}\in\N$. and we can factor out $y$ from $y$, of course. Thus, we are able to rewrite $n!+y=y(z+1)$. For example if $n=4$, then $n!+3=3(8+1)$ is composite. Therefore $y$ is a factor, so $n!+y$ is composite.
\end{enumerate}
Since $n$ plus any even or odd number up to $n$ is composite, then the statement is true.
\end{proof}
\section{Problem 3}
Prove by strong induction that for any $n\in\N$, $n>1$, there exists a prime factorization of $n$.
\begin{proof}
\emph{Basis}: $n=2$, the prime factorization is $2=2^1$.
\newline \emph{Inductive step}: Assume it is true for all $n>k$. \newline \textbf{N2S}: It is true for $n=k$. There are 2 cases:\begin{enumerate}
    \item If $n=k$ is prime, then the prime factorization is trivial, $k=k\cdot1$.
    \item If $n=k$ is composite, then $\exists m,n\in\N$, s.t. $k=m\cdot n$, and $1<m,n\leq (k-1)$. By our inductive assumption, $m,n$ both have prime factorizations, so we may write them as $m=2^{m_2}*3^{m_3}*5^{m_5}*...*p^{m_p}$, and $n=2^{n_2}*3^{n_3}*5^{n_5}*...*p^{n_p}$. Then $k$ must have a prime factorization, namely $k=2^{m_2+n_2}*3^{m_3+n_3}*5^{m_5+n_5}*...*p^{m_p+n_p}$.
\end{enumerate}
Therefore, by strong induction, for any $n\in\N$, $n>1$, there exists a prime factorization of $n$.
\end{proof}
\section{Problem 4}
Find a solution to the equation $5x+8y=1$.
\newline Since this is a Linear Diophantine equation, it can be solved via the reverse Euclidean Algorithm.
$$8=5\cdot1+3$$ $$5=3\cdot1+2$$ $$3=2\cdot1+1$$ $$2=1\cdot2+0$$ In reverse: $$1=(1\cdot3)+(-1\cdot2)$$ $$=(-1\cdot5)+(2\cdot3)$$ $$=(2\cdot8)+(-3\cdot5)$$ Therefore $x=-3,y=2$ is a solution to this equation.
\section{Problem 5}
Let $f(x)=x^2$. For each pair of sets $X$ and $Y$ below, determine if $f$ defines a function from $X\to Y$, and if yes, whether this function is injective and/or surjective.
\begin{enumerate}
    \item $X=\N$, $Y=\N$. Yes, $f$ defines an injective function. \begin{proof}
    $f$ is injective because $\forall x_1\neq x_2\in\N$, their mapping is $f(x_1)=x_{1}^2$, and $f(x_2)=x_2^2$. If $f(x_1)=f(x_2)$, then $x_1^2=x_2^2\Longrightarrow (x_1-x_2)^2=0$, and since the natural numbers have no nontrivial zero divisors, $x_1$ must equal $x_2$. Since this is true, the contrapositive, (if $f(x_1)\neq f(x_2)$, then $x_1\neq x_2$) must be true. Therefore $f$ is injective.\newline $f$ is not surjective because any number $y\in\N$ which is not a perfect square, is not hit.\end{proof}
    \item $X=\Z$, $Y=\Z$. Yes, $f$ defines a function, however it is neither injective nor surjective. \begin{proof}
    It is not injective because $x_1=2,x_2=-2$ both map to 4. It is not surjective because any negative number, as well as any number that is not a perfect square, is not hit.\end{proof}
    \item $X=\Z$, $Y=\N$. Yes, $f$ defines a function, however it is neither injective nor surjective. \begin{proof}
    It is not injective because again, both $x_1=2,x_2=-2$ map to 4. It is not surjective because any number that is not a perfect square is not hit.\end{proof}
    \item $X=\N$, $Y=\Z$. Yes, $f$ defines an injective function. \begin{proof}
    Similarly to the first part, $f$ is injective because $\forall x_1\neq x_2\in\N$, their mapping is $f(x_1)=x_{1}^2$, and $f(x_2)=x_2^2$. If $f(x_1)=f(x_2)$, then, by the first part of this problem, $x_1=x_2$. Therefore the contrapositive is true, so $f$ is an injection. \newline $f$ is not surjective because any negative integer, or integer that is not a perfect square, is not hit.\end{proof}
    \item $X=\{0,1\}$, $Y=\{0,1\}$. Yes, $f$ defines a bijective function. \begin{proof}
    $f$ is injective because we can check all $x\in X$, to see if we get different values of $y\in Y$. $f(0)=0^2=0\in Y$, and $f(1)=1^2=1\in Y$. We also just checked that every $y\in Y$ has a unique $x\in X$, so the function is both injective and surjective.\end{proof}
\end{enumerate}
\end{document}