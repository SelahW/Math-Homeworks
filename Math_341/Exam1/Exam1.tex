\documentclass[12pt]{article}
\usepackage[utf8]{inputenc}
\usepackage{mathtools}
\usepackage{amsmath}
\usepackage{amsfonts}
\usepackage{amssymb}
\usepackage{amsthm}
\usepackage{centernot}
\newcommand{\notequiv}{\centernot\equiv}
\newcommand{\N}{\mathbb{N}}
\newcommand{\Z}{\mathbb{Z}}
\newtheorem{theorem}{Theorem}
\newcommand{\divby}{%
  \mathrel{\text{\vbox{\baselineskip.65ex\lineskiplimit0pt\hbox{.}\hbox{.}\hbox{.}}}}%
  }
\newcommand{\notdivby}{\centernot\divby}
\renewcommand\qedsymbol{QED}
\title{\scalebox{2}{Math 341 Exam 1}}
\author{\scalebox{1.5}{Theo Koss}}
\date{September 2020}
\begin{document}
\maketitle
\section{Problem 1}
Prove by induction: $7^{n+1}+8^{2n-1}\divby57$ for all $n\in\N$.
\begin{proof}
\emph{Basis}: Check for $n=1$: $7^2+8\divby57=57\divby57$.\newline \emph{Inductive step}: Assume that the formula holds for $n=k$. We must show it holds for $n=k+1$. So, by assumption: $7^{k+1}+8^{2k-1}\divby57$. We must show: $$7^{k+2}+8^{2k+1}\divby57$$ Simplifying:$$=7\cdot7^{k+1}+64\cdot8^{2k-1}$$ Rearranging:$$=7\cdot7^{k+1}+7\cdot8^{2k-1}+57\cdot8^{2k-1}=7(7^{k+1}+8^{2k-1})+57\cdot8^{2k-1}$$By our assumption:$$7^{k+1}+8^{2k-1}\divby57\Longrightarrow 7(7^{k+1}+8^{2k-1})\divby57$$ and by definition:$$57\cdot8^{2k-1}\divby57$$ Finally:$$7(7^{k+1}+8^{2k-1})\divby57\text{ and } 57\cdot8^{2k-1}\divby57$$ $$\Longrightarrow 7^{k+2}+8^{2k+1}\divby57$$ As required.
\end{proof}
\section{Problem 2}
Prove by induction: if $a\geq-1$, then, $(1+a)^n\geq1+na$ for all $n\geq0$.
\begin{proof}
\emph{Basis}: Check for $n=1$: $1+a\geq1+a$.\newline \emph{Inductive step}: \newline Assume the formula holds for $n=k$, so $(1+a)^k\geq(1+ka), a\geq-1$ We must show it holds for $n=k+1$, that is: $$(1+a)^{k+1}\geq(1+(k+1)a)$$ Simplifying:$$(1+a)\cdot(1+a)^k\geq(1+ka+a)$$
\end{proof}
\section{Problem 3}
Prove if $a\divby c$ and $b\divby c$, then for any $x$ and $y$, $ax+by\divby c$.
\begin{proof}
We must show that $ax+by=cl$, for some $l\in\Z$. $a\divby c$, by definition means $a=cn, n\in\Z$, similarly, $b\divby c$ by definition means $b=cm, m\in\Z$. Then for any $x,y$, $x\cdot cn+y\cdot cm \divby c$, since we can factor out a $c$: $ax+by=c(xn+ym)$ let $l=xn+ym$, then we have: $ax+by=cl, l\in\Z$, which, by definition, means $ax+by\divby c$.
\end{proof}
\section{Problem 4}
Prove that $12\notdivby5$.
\begin{proof}
Assume 12 \emph{is} divisible by 5. Then 12=5c, for some \textbf{integer} c. This means that $c=\frac{12}{5}$, which is not an integer, thus we have a contradiction, so $12\notdivby5$.
\end{proof}
\section{Problem 5}
Let $a,b,c\in\N$ and suppose $a>c$ and $b>c$. True or false: if $ab\divby c$, then $a\divby c$ or $b \divby c$. Prove.
\begin{proof}
This is true, since $ab\divby c$, by definition, means $ab=cn$, for some $n\in\N$.
\end{proof}
\end{document}