\documentclass[12pt]{article}
\usepackage[utf8]{inputenc}
\usepackage{mathtools}
\usepackage{amsthm}
\usepackage{amsmath}
\usepackage{amsfonts}
\usepackage{amssymb}
\usepackage{centernot}
\usepackage{marvosym}
\let\marvosymLightning\Lightning
\newtheorem{theorem}{Theorem}
\newtheorem{corollary}{Corollary}[theorem]
\newtheorem*{remark}{Remark}
\renewcommand\qedsymbol{QED}
\newcommand{\N}{\mathbb{N}}
\newcommand{\Z}{\mathbb{Z}}
\newcommand{\divby}{%
  \mathrel{\text{\vbox{\baselineskip.65ex\lineskiplimit0pt\hbox{.}\hbox{.}\hbox{.}}}}%
  }
\newcommand{\notdivby}{\centernot\divby}
\title{\scalebox{2}{Math 341 Homework 11}}
\author{\scalebox{1.5}{Theo Koss}}
\date{November 2020}
\begin{document}
\maketitle
\section{Practice problems}
\subsection{Problem 11.1}
Let $p$ be prime, solve the equation $x^2\equiv1$ in $\Z_p$. (Find all solutions, and prove that there are no other solutions.)
\begin{proof}
$$x^2=1\Longrightarrow x^2-1=0\Longrightarrow(x+1)(x-1)=0$$ Assume $(x+1)$ is nonzero, because we are in $\Z_p$, all nonzero elements are invertible. So multiplying by $(x+1)^{-1}$: $$(x+1)(x+1)^{-1}(x-1)=0(x+1)^{-1}\Longrightarrow(x-1)=0,\text{ so }x=1$$ Now assume $(x-1)$ is nonzero. Again, $(x-1)$ is invertible, so multiplying by $(x-1)^{-1}$ yields a similar result:$$(x-1)(x-1)^{-1}(x+1)=0(x-1)^{-1}\Longrightarrow(x+1)=0,\text{ so }x=-1$$ These are the only 2 cases for which this is true, $x=1,-1$.
\end{proof}
\subsection{Problem 11.4}
Let $p$ be prime, let $S=\Z_p-\{0\}=\{[1]_p,[2]_p,...,[p-1]_p\}$. Prove that for $y\neq0$, $L_y$ restricts to a bijective map $L_y|_S:S\to S$. (Prove that there are no 0 divisors in $\Z_p$)
\begin{proof}
N2S: $\nexists x\in S$, such that $L_y(x)=0$. To the contrary, assume such an $x$ does exist. \newline Suppose $$L_y(x)=0$$ then $$yx=0$$ Since $y\in S$, it is invertible, ($\exists y'\in S$ s.t. $yy'=1$). Multiply by $y'$ on both sides: $$yy'x=0y'$$ This yields: $x=0$ \scalebox{1.5}{\Lightning} Because $0\notin S$. So $L_y$ restricts to a bijective map $S\to S$.
\end{proof}
\subsection{Problem 11.7}
Find $2019^{2020}\mod{43}$.
\newline By FlT, (Fermat's Little Theorem) Since $43$ is prime, any $a^{42}=1\mod{43}$. $$2020=42\cdot48+4,\text{ and }2019=-2\mod{43}$$ $$2019^{2020}=(-2^{42})^{48}(-2^4)$$ As stated above, by FlT, $$(-2^{42})^{48}=1^{48}=1\mod{43}$$ So $$(-2^{42})^{48}(-2^4)=1\cdot-2^4=16\mod{43}$$
\end{document}