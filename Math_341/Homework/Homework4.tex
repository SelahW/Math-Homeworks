\documentclass[12pt]{article}
\usepackage[utf8]{inputenc}
\usepackage{mathtools}
\usepackage{amsthm}
\usepackage{amsmath}
\usepackage{amsfonts}
\usepackage{amssymb}
\usepackage{centernot}
\renewcommand\qedsymbol{QED}
\newcommand{\N}{\mathbb{N}}
\newcommand{\Z}{\mathbb{Z}}
\newcommand{\divby}{%
  \mathrel{\text{\vbox{\baselineskip.65ex\lineskiplimit0pt\hbox{.}\hbox{.}\hbox{.}}}}%
  }
\newcommand{\notdivby}{\centernot\divby}
\title{\scalebox{2}{Math 341 Homework 4}}
\author{\scalebox{1.5}{Theo Koss}}
\date{September 2020}
\begin{document}
\maketitle
\section{Practice problems}
\subsection{Problem 1}
Let $a\in\Z$, $b\in\N$. Suppose $a$ is divided by $b$ with remainder $r$ and quotient $q$. Divide $a+1$ by $b$ with remainder. \begin{proof}
By definition of divisibility with remainder, $$a=bq+r \text{ and } 0\leq r<b$$Also by theorem 4.1, $q,r\in\Z$ are unique $\forall a$. So $$a+1=bq'+r'$$There are 2 cases:\begin{enumerate}
    \item $r\in\{0,1,...b-2\}$: In this case $r'=r+1$, and since $r\leq b-2$ we know $r'\leq b-1\text{ and thus }r'<b$. Also because we haven't ``overflowed*'' so to speak, $q'=q$.\newline(* By this I mean since $r'<b$, $q$ remains the same, whereas if $r'=b$, then $q'=q+1$.)
    \item $r=b-1$: In this case, $r'=b$, since $r'=r+1=(b-1)+1=b$. And since $r'=b$, this is division with remainder 0, (or normal division). Also since $r'=b$, $q'=q+1$.
\end{enumerate}
So $a+1=bq+r+1$ when $r\in\{0,1,...b-2\}$.\newline And $a+1=b(q+1)+0$ when $r=b-1$.
\end{proof}
\subsection{Problem 3}
Let $b\in\N$ and let $a\in\Z^<$. Prove the existence of $q,r\in\Z$, such that $a=bq+r$ and $0\leq r<b$.
\begin{proof}
By definition, $a=bq+r$, and $0\leq r<b$, then $-a=bq'+r'$, $0\leq r<b$. $q'=-q$, however $r'$ can be one of two things, either:\begin{enumerate}
    \item $r'=b-r$: In this case, $-a=bq'+r'=-bq+b-r=-b(q-1)-r$. Therefore $q,r\in\Z$ exist for $-a$.
    \item $r'=b+r$: In this case, $-a=bq'+r'=-bq+b+r=-b(q-1)+r$. Therefore $q,r$ exist, and are unique, for $-a$.
\end{enumerate}
\end{proof}
\subsection{Problem 4}
Let $b\in\N$ and suppose $-b<r<b$. Prove that if $r\divby b$, then $r=0$.
\begin{proof}
Recall that by definition of divisibility, $r\divby b\implies r=bn,\text{ for some }n\in\Z$. Also since $-b<r<b$, then $$-b<bn<b$$ again where $n\in\Z$. We can divide everything by $b$ to get:$$-1<n<1, n\in\Z$$There are no integers $n$ between -1 and 1, except 0. Thus $n=0$ and since $r=bn$, $r=0$.
\end{proof}
\end{document}