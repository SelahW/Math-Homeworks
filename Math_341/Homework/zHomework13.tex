\documentclass[12pt]{article}
\usepackage[utf8]{inputenc}
\usepackage{mathtools}
\usepackage{amsthm}
\usepackage{amsmath}
\usepackage{amsfonts}
\usepackage{amssymb}
\usepackage{centernot}
\usepackage{marvosym}
\let\marvosymLightning\Lightning
\newtheorem{theorem}{Theorem}
\newtheorem{corollary}{Corollary}[theorem]
\newtheorem*{remark}{Remark}
\renewcommand\qedsymbol{QED}
\newcommand{\N}{\mathbb{N}}
\newcommand{\Z}{\mathbb{Z}}
\newcommand{\divby}{%
  \mathrel{\text{\vbox{\baselineskip.65ex\lineskiplimit0pt\hbox{.}\hbox{.}\hbox{.}}}}%
  }
\newcommand{\notdivby}{\centernot\divby}
\title{\scalebox{2}{Math 341 Homework 13}}
\author{\scalebox{1.5}{Theo Koss}}
\date{December 2020}
\begin{document}
\maketitle
\section{Practice problems}
\subsection{Problem 13.1}
Find all $\phi(n)$ for $n\leq12$.
\begin{table}[h]
\begin{tabular}{|l|l|l|l|l|l|l|l|l|l|l|l|l|}
\hline
$n$       & 1 & 2 & 3 & 4 & 5 & 6 & 7 & 8 & 9 & 10 & 11 & 12 \\ \hline
$\phi(n)$ & 1 & 1 & 2 & 2 & 4 & 2 & 6 & 4 & 6 & 4  & 10 & 4  \\ \hline
\end{tabular}
\end{table}
\subsection{Problem 13.2}
Let $p$ be prime, find $\phi(p)$.
\begin{proof}
By definition, because $p$ is prime, there are no numbers smaller than $p$ which evenly divide it. Therefore, since all numbers $\{1,2,\dots,p-1\}$ are relatively prime to $p$, $\phi(p)=p-1$.
\end{proof}
\subsection{Problem 13.6}
Prove that $\phi(n)$ is the number of invertible elements in $\Z_{n}$.
\begin{proof}
By definition, an element $x\in\Z_n$ is invertible iff $\gcd(x,n)=1$, (they are relatively prime). Conveniently, the totient function at $n$, $\phi(n)$, counts how many numbers smaller than $n$ are relatively prime to $n$. Therefore $\phi(n)$ is the number of invertible elements in $\Z_n$.
\end{proof}
\end{document}
