\documentclass[hidelinks,12pt]{article}
\usepackage[utf8]{inputenc}
\usepackage{mathtools}
\usepackage{amsthm}
\usepackage{amsmath}
\usepackage{amsfonts}
\usepackage{amssymb}
\usepackage{centernot}
\usepackage{marvosym}
\usepackage{enumitem}
\usepackage{hyperref}
\setcounter{tocdepth}{1}
\let\marvosymLightning\Lightning
\renewcommand{\geq}{\geqslant}
\renewcommand{\leq}{\leqslant}
\newtheorem{theorem}{Theorem}
\newtheorem{corollary}{Corollary}[theorem]
\newtheorem*{remark}{Remark}
\renewcommand\qedsymbol{QED}
\newcommand{\R}{\mathbb{R}}
\newcommand{\N}{\mathbb{N}}
\newcommand{\Z}{\mathbb{Z}}
\newcommand{\Q}{\mathbb{Q}}
\newcommand{\divby}{%
  \mathrel{\text{\vbox{\baselineskip.65ex\lineskiplimit0pt\hbox{.}\hbox{.}\hbox{.}}}}%
  }
\newcommand{\notdivby}{\centernot\divby}
\title{\scalebox{2}{Math 523 Homework 6}}
\author{\scalebox{1.5}{Theo Koss}}
\date{December 2023}

\begin{document}

\maketitle

\section*{Problems}
\begin{enumerate}
    \item Use definition of derivative to determine whether $f$ is differentiable at $0$, and if so find the value $f'\left(0\right)$, do the same for $g$. $$f'\left(0\right)=\lim_{h\to0}\frac{f\left(0+h\right)-f\left(0\right)}{h}=\lim_{h\to0}\frac{f\left(h\right)}{h}=\lim_{h\to0}\frac{h\sin\left(\frac{1}{h}\right)}{h}=\lim_{h\to0}\sin\left(\frac{1}{h}\right)$$ This does not exist. So $f$ is not differentiable at $0$. $g\left(x\right)=xf\left(x\right)$ so same calculation as above but: $$g'\left(0\right)=\dots=\lim_{h\to0}h\sin\left(\frac{1}{h}\right)=0$$ Does exist and equals 0 via the squeeze theorem. 
    \item Show that if $f$ is even then $f'$ is odd, and if $f$ odd, $f'$ even.\begin{proof}
        Suppose $f$ is even, so $f\left(-x\right)=f\left(x\right)$. Then by definition of derivative, $$f'\left(-x\right)=\lim_{h\to0}\frac{f\left(-x+h\right)-f\left(-x\right)}{h}=\lim_{h\to0}\frac{f\left(-\left(x-h\right)\right)-f\left(x\right)}{h}=$$
        $$=\lim_{h\to0}\frac{f\left(x-h\right)-f\left(x\right)}{h}=-\lim_{h\to0}\frac{f\left(x+h\right)-f\left(x\right)}{h}=-f'\left(x\right)$$\\
        Suppose $f$ is odd, so that $f\left(-x\right)=-f\left(x\right)$. Then $$f'\left(-x\right)=\lim_{h\to0}\frac{f\left(-x+h\right)-f\left(-x\right)}{h}=\lim_{h\to0}\frac{f\left(-\left(x-h\right)\right)+f\left(x\right)}{h}=$$
        $$=\lim_{h\to0}\frac{-f\left(x-h\right)+f\left(x\right)}{h}=-\lim_{h\to0}\frac{f\left(x-h\right)-f\left(x\right)}{h}=$$\\$$=\lim_{h\to0}\frac{f\left(x+h\right)-f\left(x\right)}{h}=f'\left(x\right)$$ So $f'$ is even.
    \end{proof}
    \item Show that the function $f\left(x\right)=x^3+2x+1$ is strictly increasing on $\R$ and thus, has an inverse function $f^{-1}$ on $\R$. Find the value of $\left(f^{-1}\right)'\left(y\right)$ at the points corresponding to $x=0,1,-1$.\begin{proof}
        $f'\left(x\right)=3x^2+2$ which is never negative for $x\in\R$, so $f$ is strictly increasing on $\R$. By the inverse function theorem, $$\left(f^{-1}\right)'\left(y\right)=\frac{1}{f'\left(x\right)}$$ \begin{itemize}
            \item At $x=0,\ \left(f^{-1}\right)'\left(y\right)=\frac{1}{2}$
            \item At $x=1,\ \left(f^{-1}\right)'\left(y\right)=\frac{1}{5}$
            \item At $x=-1,\ \left(f^{-1}\right)'\left(y\right)=\frac{1}{5}$
        \end{itemize}
    \end{proof}
    \item\begin{enumerate}
        \item[(a)] Use definition of derivative to prove: If $f$ is an increasing function that is differentiable on an interval $I$, then $f'\left(x\right)\geq0$ for all $x\in I$.
        \item[(b)] Is the following true or false? If $f$ is strictly increasing and differentiable on $I$, then $f'\left(x\right)>0$ on $I$. Explain
    \end{enumerate}
    \begin{proof}
        $\left(a\right):$ Let $f$ be an increasing function which is differentiable on $I$. Then $$f'\left(x\right)=\lim_{h\to0}\frac{f\left(x+h\right)-f\left(x\right)}{h}$$ But $f\left(x+h\right)\geq f\left(x\right)$ since $f$ is increasing, so numerator is nonnegative $\implies$ $f'\left(x\right)\geq0$ for $x$ in the interval.\\
        $\left(b\right):$ True, by $f$ being strictly increasing, the statement goes from: $[f\left(x+h\right)\geq f\left(x\right)]\to[f\left(x+h\right)>f\left(x\right)]$ so instead of the numerator being nonnegative, it becomes strictly positive.
    \end{proof} (WLOG assuming $h\to 0$ from the right, if from the left, denominator is negative as is numerator so still $f'\left(x\right)\geq0$.)
    \item Show that if $f'\left(c\right)=0$ for some $c\in I$ and $f''\left(x\right)>0$ for all $x\in I$, then $f$ has a minimum value at $c$. Similarly, if $f'\left(c\right)=0$ for some $c\in I$ and $f''\left(x\right)<0$ then $f$ has a maximum value at $c$.\begin{proof}
        Let $f$ be twice differentiable, $f''\left(x\right)>0\ \forall x\in I$ and $f'\left(c\right)=0$ for some $c\in I$. Then $$f''\left(c\right)=\lim_{h\to0}\frac{f'\left(c+h\right)-f'\left(c\right)}{h}=\lim_{h\to0}\frac{f'\left(c+h\right)}{h}>0$$ So for small $h$: $\frac{f'\left(c+h\right)}{h}>0$, which means if $h\to0$ from the left, $f'\left(c+h\right)<0$, so $f$ is decreasing, and if $h\to0$ from the right, $f'\left(c+h\right)>0$, so that $f$ is increasing. By the first derivative test, $f$ has a local minimum at $c$.\\ Similarly, let $f$ be twice diff., $f''\left(x\right)<0$ $\forall x\in I$ and $f'\left(c\right)=0$ for some $c\in I$. Then follow the same argument but: $\frac{f'\left(c+h\right)}{h}<0$ for small $h$. So if $h\to0$ from left, $f'\left(c+h\right)>0$ so $f$ is increasing, and if $h\to0$ from right, $f'\left(c+h\right)<0$ so $f$ is decreasing. Again apply first derivative test, $\implies f$ has a local maximum at $c$.
    \end{proof}
    \item Evaluate the following limits:\begin{itemize}
        \item[(a)] $$\lim_{x\to0^+}\frac{\sin x}{\sqrt{x}}\ \left(0,\infty\right)$$ $$\lim_{x\to0^+}\frac{\sin x}{\sqrt{x}}=\lim_{x\to0^+}\frac{2\sqrt{x}\cos x}{1}=0$$
        \item[(b)] $$\lim_{x\to0^+}\frac{\tan x-x}{x^3}\ \left(0,\frac{\pi}{2}\right)$$ $$\lim_{x\to0^+}\frac{\tan x-x}{x^3}=\lim_{x\to0^+}\frac{\sec^2x-1}{3x^2}=\lim_{x\to0^+}\frac{2\sec^2x\tan x}{6x}=$$ $$=\lim_{x\to0^+}\frac{-4\sec^2x+6\sec^4x}{6}=\frac{1}{3}$$
        \item[(c)] $$\lim_{x\to0^+}\left(\frac{1}{x}-\frac{1}{\sin x}\right)\ \left(0,\frac{\pi}{2}\right)$$ $$\lim_{x\to0^+}\left(\frac{1}{x}-\frac{1}{\sin x}\right)=\lim_{x\to0^+}\left(1-\frac{1}{\cos x}\right)=1-1=0$$
    \end{itemize}
    \item \begin{enumerate} If $\lim_{x\to\infty}f(x)=\lim_{x\to\infty}g(x)=\infty$ and $\lim_{x\to\infty}\frac{f'(x)}{g'(x)}=L\in\R$, then $\lim_{x\to\infty}\frac{f(x)}{g(x)}=L$.
        \item[(a)] Use result above to conclude the following: If $\lim_{x\to0}f(x)=\lim_{x\to0}g(x)=\infty$ and $\lim_{x\to0}\frac{f'(x)}{g'(x)}=L\in\R$ then $\lim_{x\to0}\frac{f(x)}{g(x)}=L$.\begin{proof}
            Consider $\lim_{x\to0^+}\frac{f(x)}{g(x)}$ where $\lim_{x\to0^+}f(x)=\lim_{x\to0^+}g(x)=\infty$, and $\lim_{x\to0^+}\frac{f'(x)}{g'(x)}=L\in\R$. 
        \end{proof}
    \end{enumerate}
\end{enumerate}
\end{document}