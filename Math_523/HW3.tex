\documentclass[hidelinks,12pt]{article}
\usepackage[utf8]{inputenc}
\usepackage{mathtools}
\usepackage{amsthm}
\usepackage{amsmath}
\usepackage{amsfonts}
\usepackage{amssymb}
\usepackage{centernot}
\usepackage{marvosym}
\usepackage{enumitem}
\usepackage{hyperref}
\setcounter{tocdepth}{1}
\let\marvosymLightning\Lightning
\renewcommand{\geq}{\geqslant}
\renewcommand{\leq}{\leqslant}
\newtheorem{theorem}{Theorem}
\newtheorem{corollary}{Corollary}[theorem]
\newtheorem*{remark}{Remark}
\renewcommand\qedsymbol{QED}
\newcommand{\R}{\mathbb{R}}
\newcommand{\N}{\mathbb{N}}
\newcommand{\Z}{\mathbb{Z}}
\newcommand{\Q}{\mathbb{Q}}
\newcommand{\divby}{%
  \mathrel{\text{\vbox{\baselineskip.65ex\lineskiplimit0pt\hbox{.}\hbox{.}\hbox{.}}}}%
  }
\newcommand{\notdivby}{\centernot\divby}
\title{\scalebox{2}{Math 523 Homework 3}}
\author{\scalebox{1.5}{Theo Koss}}
\date{October 2023}

\begin{document}

\maketitle
\section*{Section 2.3}
\begin{enumerate}
    \item[5.] If $\{a_n\}$ and $\{b_n\}$ diverge to $+\infty$, prove that $\{a_n+b_n\}$ and $\{a_nb_n\}$ also diverge to $+\infty$. \begin{proof}
    By limit properties, we have that $\lim_{n\to\infty}\{a_n+b_n\}=\lim_{n\to\infty}\{a_n\}+\lim_{n\to\infty}\{b_n\}$, and $\lim_{n\to\infty}\{a_nb_n\}=\lim_{n\to\infty}\{a_n\}\cdot\lim_{n\to\infty}\{b_n\}$, so $$\lim_{n\to\infty}\{a_n+b_n\}=+\infty+(+\infty)=+\infty$$ $$\lim_{n\to\infty}\{a_nb_n\}=(+\infty)(+\infty)=+\infty$$
    \end{proof}
\end{enumerate}
\section*{Section 2.4}
\begin{enumerate} 
    \item[15(b).] If $a_1=1$ and $a_{n+1}=\sqrt{1+a_n}$, show that $\{a_n\}$ converges to $\alpha$.\begin{proof}
        We will show $\{a_n\}$ is strictly increasing and bounded above, and thus converges.\begin{itemize}
            \item $a_2>a_1$ because $\sqrt{2}>1$, assume $a_{k+1}>a_{k}$ for some $k$. We seek $a_{k+2}>a_{k+1}$, so $a_{k+2}=\sqrt{1+a_{k+1}}>\sqrt{1+a_k}=a_{k+1}$. So $\{a_n\}$ is strictly increasing.
            \item $\{a_n\}$ is bounded above by 3, $\sqrt{1+1}<3$. Assume $a_k<3$ for some $k$, we seek $a_{k+1}<3$, $a_{k+1}=\sqrt{1+a_k}<\sqrt{1+3}<3$.
        \end{itemize}So $\{a_n\}$ converges, to find the value we will take the limit of both sides of the recursion formula. $$\lim_{n\to\infty}a_{n+1}=\lim_{n\to\infty}\sqrt{1+a_n}\implies A=\sqrt{1+A}$$ This gives $$A^2-A-1=0$$ And since $A$ is nonnegative, $A=\frac{1+\sqrt{5}}{2}$
    \end{proof}
\end{enumerate}
\section*{Section 2.5}
\begin{enumerate}
    \item[2(c).] Find the set of accumulation points of $S=\{a_n\mid n\in\N\}$ Where $$a_n=\begin{cases}
        0 & n \text{ is odd}\\
        \frac{n}{n+1} & n \text{ is even}\\
    \end{cases}$$ $\frac{n}{n+1}=1-\frac{1}{n+1}$, so as $n$ increases, $a_n$ tends to $1$. Therefore for any nbd of 1, there are infinitely many $n$ such that $a_n$ is in the nbd. $\{1\}$ is the set of accumulation points.
\end{enumerate}
\section*{Additional Problems}
\begin{enumerate}
    \item The mistake in the argument is treating the limit of the sequence, $L$ as a number. When in fact it is $\infty$, which is not a number. If it were you could also claim $\infty=\infty+1\implies 0=1$, which is clearly untrue.
    \item Using the axiom of completeness, consider a nonempty set of reals $B$ which is bounded below. Let $A=-B$, since $B$ is bounded below, we have some $x$ for which $x\leq b\ \forall b\in B$, then $-x\geq -b\ \forall (-b)\in A$. So lower bounds for $B$ are upper bounds for $A$. Since $A$ is bounded above, it has a lub, call it $y$. Then I claim $-y$ is the greatest lower bound for $B$. It is indeed a lower bound, if not then $b<-y$ for some $b\in B$, but that would mean $-b>y$ for some $-b\in A$, which is false. A similar argument shows nothing greater than $-y$ is a lower bound.
    \item Let $x,\bar{x}$ both be suprema of a set of reals $A$ and $x\neq\bar{x}$. Then since $x\in A$ and $\bar{x}$ is a supremum, $x\leq \bar{x}$. Similarly since $\bar{x}\in A$ and $x$ is a supremum, $\bar{x}\leq x$. Thus $x=\bar{x}$ contradicts $x\neq\bar{x}$. Thus there must only be one.
    \item Let $q\neq0\in\Q^*$ and $x\in\R^*\setminus\Q^*$. By way of contradiction, assume $qx=\frac{a}{b}$ for nonzero integers $a,b$. Since $q\in\Q^*$, one can write $q=\frac{m}{n}$ for nonzero integers $m,n$. So $x=\frac{na}{mb}$, and since all four of $a,b,m,n$ are nonzero integers, this is a fraction of nonzero integers, so $x\in\Q^*$, contradiction.
    \item Prove for $x<y$, $(x,y)$ contains infinitely many rationals.\begin{proof}
    By density of $\Q$ in $\R$, there exists at least one $a\in\Q$ such that $x<a<y$. Let $n$ be an integer such that $\frac{1}{n}<y-x$, then consider $x+\frac{1}{n},x+\frac{1}{n+1},\dots$ Since $\frac{1}{n}<y-x$, and $0<\frac{1}{n+k}<\frac{1}{n}$, this is an infinite sequence of rationals between $x$ and $y$.
    \end{proof}
\end{enumerate}
\end{document}