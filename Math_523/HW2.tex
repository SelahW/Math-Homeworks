\documentclass[hidelinks,12pt]{article}
\usepackage[utf8]{inputenc}
\usepackage{mathtools}
\usepackage{amsthm}
\usepackage{amsmath}
\usepackage{amsfonts}
\usepackage{amssymb}
\usepackage{centernot}
\usepackage{marvosym}
\usepackage{enumitem}
\usepackage{hyperref}
\setcounter{tocdepth}{1}
\let\marvosymLightning\Lightning
\renewcommand{\geq}{\geqslant}
\renewcommand{\leq}{\leqslant}
\newtheorem{theorem}{Theorem}
\newtheorem{corollary}{Corollary}[theorem]
\newtheorem*{remark}{Remark}
\renewcommand\qedsymbol{QED}
\newcommand{\R}{\mathbb{R}}
\newcommand{\N}{\mathbb{N}}
\newcommand{\Z}{\mathbb{Z}}
\newcommand{\Q}{\mathbb{Q}}
\newcommand{\divby}{%
  \mathrel{\text{\vbox{\baselineskip.65ex\lineskiplimit0pt\hbox{.}\hbox{.}\hbox{.}}}}%
  }
\newcommand{\notdivby}{\centernot\divby}
\title{\scalebox{2}{Math 523 Homework 2}}
\author{\scalebox{1.5}{Theo Koss}}
\date{September 2023}

\begin{document}
\maketitle
\section*{Section 2.1}
\begin{enumerate}
    \item[2.]Determine whether the given sequence $\{a_n\}$ converges or diverges with $a_n$ as given. In
 each case, prove your conclusion.
    \begin{enumerate}
        \item[b.]$a_n=\frac{n}{n^2-2}$. Converges to $\lim_{n\to\infty}\frac{n}{n^2-2}=0$.\begin{proof}
            Choose an $\epsilon>0$, and choose $N$ so that $|\frac{n}{n^2-2}|<\epsilon$ when $n\geq N$. Whenever $n\geq N$, we get $|\frac{n}{n^2-2}|=\frac{|n|}{|n^2-2|}$ $|n|=n$ if $n\geq0$, and $|n^2-2|=n^2-2$ if $n>1$. So $$|\frac{n}{n^2-2}|=\frac{n}{n^2-2}$$ if $n>1$. $$\frac{n}{n^2-2}<\frac{2n}{n^2}=\frac{2}{n}$$ If $n>2$. This is less than $\epsilon$ when $n>\frac{2}{\epsilon}$. So choose $N>\max\{2,\frac{2}{\epsilon}\}$. Then $|\frac{n}{n^2-2}|<\epsilon$ for $n\geq N$.
        \end{proof}
        \item[c.]$a_n=\frac{1}{n^p}$ where $p$ positive constant. Converges to $0$ for $p\geq1$.\begin{proof}
            By Theorem 2.1.13 in the text, if $|r|<1$, then $\lim_{n\to\infty}r^n=0$. Then note that $\frac{1}{n^p}=(\frac{1}{n})^p$, and furthermore, $|(\frac{1}{n})^p|<1$ for all $p\geq1$ and $n>1$. So choose $N>1$, then if $n>N$, $$|\frac{1}{n^p}|<\epsilon$$.
        \end{proof}
    \end{enumerate}
    \item[11.]Give an example of a sequence that is bounded but not convergent.$$a_n=(-1)^n$$ Series is bounded, as it fluctuates between $-1$ and $1$ for $n$ odd and even, respectively, so it is bounded by $1$, however it does not converge.
    \item[17.]Use the binomial theorem to prove that $\lim_{n\to\infty}nr^n=0$ if $|r|<1$.\begin{proof}
       Let $|r|=\frac{1}{1+x}$, by binomial thm, $(1+x)^n\geq1+nx+...>nx+n(n-1)\frac{x^2}{2}$ for $n>2$.\par\null\par Let $\epsilon>0$ be given, then let $N>\ell$ $$|nr^n|=\frac{|n|}{(1+x)^n}<\frac{2n}{2nx+n(n-1)x^2}<\frac{2}{2x+(n-1)x^2}=\frac{2}{nx^2-x^2+2x}<\epsilon$$ Solving for $n$: $$n>\frac{-x^2\epsilon+2x\epsilon-2}{-x^2\epsilon}$$ Call $\ell=\frac{-x^2\epsilon+2x\epsilon-2}{-x^2\epsilon}$. Then for $n\geq\ell$, we have that $nr^n$ converges to $0$.
    \end{proof}
\end{enumerate}
\section*{Section 2.2}
\begin{enumerate}
    \item[3.]Part $d$, prove that $\lim_{n\to\infty}a_n=A$ implies  $\lim_{n\to\infty}a_n^p=A^p$ for $p\in\N$.\begin{proof}
        We will use induction, base case: $p=2$, then by part (b) of this theorem, $\lim_{n\to\infty}a_n=A$ and $\lim_{n\to\infty}a_n=A$ implies $\lim_{n\to\infty}a_na_n=AA$.
        \newline Assume it holds for $p$, then $$\lim_{n\to\infty}a_n^{p+1}=A^{p+1}$$ $$\iff\lim_{n\to\infty}a_n^{p-1}\cdot a_n=A^{p-1}\cdot A$$ $$\implies(\text{By assumption})\ a_n=A$$ Which is true by assumption.
    \end{proof}Negative case of part (a):\begin{proof}
        Let $\epsilon$ be given, since $a_n$ and $b_n$ converge, let $|a_n-A|<3\epsilon$ and $|b_n-A|<2\epsilon$. $$|a_n-b_n-A-B|=|(a_n-A)-(b_n-B)|\leq|a_n-A|-|b_n-B|<3\epsilon-2\epsilon=\epsilon$$
    \end{proof}
    \item[12.]Consider the sequences $\left\{a_n\right\}$ and $\left\{b_n\right\}$, where sequence $\left\{a_n\right\}$ converges to zero. Is it true that the sequence $\left\{a_n b_n\right\}$ converges to zero? Explain. (See Theorem 2.2.7.)\par\null\par It is true when $b_n$ is a bounded sequence, by theorem, however when it is unbounded, the product may not converge to 0, if $b_n$ ``grows faster'' than $a_n$.
\end{enumerate}
\section*{Additional Questions}
\begin{enumerate}
    \item[1.] \begin{proof}
        By way of contradiction, assume $|-1^n-A|<\epsilon=\frac{1}{2}$ for $n\geq N$. If $n$ is even: $$|a_n-A|<\frac{1}{2}\implies |-1-A|<\frac{1}{2}\implies-\frac{1}{2}<-1-A<\frac{1}{2}$$ Then consider $n+1$, which is odd: $$|a_n-A|<\frac{1}{2}\implies |1-A|<\frac{1}{2}\implies-\frac{1}{2}<1-A<\frac{1}{2}$$ Therefore we have that $A$ is between $-\frac{1}{2}$ and $-\frac{3}{2}$, and it is between $\frac{1}{2}$ and $\frac{3}{2}$. This is true of no real number, so $a_n$ must not converge to a real.
    \end{proof}
    \item[2.] Show that $\lim_{n\to\infty}(\frac{1}{n}-\frac{1}{n+1})=0$.\begin{proof}
    Remark 2.1.8(b) states: If $\{a_n\}$ and $\{b_n\}$ differ from each other in only a finite number of terms, then both sequences converge to the same value or they both diverge. Given that $a_n=\{\frac{1}{n}\}$ converges to 0 (problem 2(b) above), notice that $a_n=b_{n-1}$, where $b_n=\{\frac{1}{n+1}\}$ so they have all the same terms, offset by 1, except the starting point. Therefore $b_n$ also converges to 0. By theorem 2.2.7(a), their difference diverges to the difference $0-0=0$.
    \end{proof}
    \item[3.] Show that if $\{a_n\}$ converges to $A$, then $\{|a_n|\}$ converges to $|A|$.\begin{proof}
        By the triangle inequality, we have that $$||a_n|-|A||\leq|a_n-A|$$ Since $a_n$ converges to $A$, we have an $N$ s.t. for any $\epsilon>0$, $n\geq N$ implies $|a_n-A|<\epsilon$. Then, by our inequality above, $$||a_n|-|A||\leq|a_n-A|<\epsilon$$ For $n\geq N$, therefore we have shown $\{|a_n|\}$ converges to $|A|$ for $n\geq N$.
    \end{proof}
\end{enumerate}
\end{document}