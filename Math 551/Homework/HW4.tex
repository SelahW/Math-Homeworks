\documentclass[hidelinks,12pt]{article}
\usepackage[utf8]{inputenc}
\usepackage[table,xcdraw]{xcolor}
\usepackage{mathtools}
\usepackage{amsthm}
\usepackage{amsmath}
\usepackage{amsfonts}
\usepackage{amssymb}
\usepackage{centernot}
\usepackage{marvosym}
\usepackage{enumitem}
\usepackage{hyperref}
\setcounter{tocdepth}{1}
\theoremstyle{definition}
\newtheorem{definition}{Definition}
\let\marvosymLightning\Lightning
\newcommand{\T}{\mathcal T}
\renewcommand{\O}{\mathcal{O}}
\renewcommand{\Lightning}{\scalebox{1.5}{\marvosymLightning}}
\newtheorem{theorem}{Theorem}
\newtheorem{corollary}{Corollary}[theorem]
\newtheorem*{remark}{Remark}
\renewcommand\qedsymbol{QED}
\newcommand{\R}{\mathbb{R}}
\newcommand{\N}{\mathbb{N}}
\newcommand{\Z}{\mathbb{Z}}
\newcommand{\divby}{%
  \mathrel{\text{\vbox{\baselineskip.65ex\lineskiplimit0pt\hbox{.}\hbox{.}\hbox{.}}}}%
  }
\newcommand{\notdivby}{\centernot\divby}
\title{\scalebox{2}{Math 551 Homework 4}}
\author{\scalebox{1.5}{Theo Koss}}
\date{November 2021}
\begin{document}
\maketitle
\section{Section 3.4}
\begin{itemize}
    \item Problem 6: Let $X$ be a metric space with metric $d$ and $A$ a nonempty subset of $X$. Define $f:X\to\R$ by $$f(x)=d(x,A),\quad x\in X$$ Show that $f$ is continuous.\begin{proof}We wish to show that $\forall x,y\in X,\quad |f(x)-f(y)|\leq d(x,y)$.\newline Consider the distance from $x,y$ to a third point $z\in A$. There are 2 cases:\begin{enumerate}[label=\roman*.]
        \item $d(x,z)\geq d(y,z)$.
        \item $d(x,z)<d(y,z)$.
    \end{enumerate}
    i.\begin{align*}
        &\forall z\in A,\quad f(x)\leq d(x,z)\qquad\text{(By definition of infimum)}\\
        &\exists\epsilon>0\text{ s.t. }f(x)+\epsilon>d(x,z)
    \end{align*}
    \begin{theorem}[Inverse Triangle Inequality]$$\forall x,y\in\R,\quad |x-y|\geq||x|-|y||$$ This follows directly from the Triangle Inequality.
    \end{theorem}
    Using Thm. 1, \begin{align*}
        &d(x,z)-d(y,z)\leq d(x,y)\\
        \implies &f(x)-d(y,z)\leq d(x,y) \\
        \implies &f(x)-f(y)-\epsilon\leq d(x,y)\\
        \implies &f(x)-f(y)\leq d(x,y)+\epsilon
    \end{align*}
    Thus $f(x)-f(y)\leq d(x,y)$, As required.\newline ii.\newline   This case is similar, and the proof is nearly the same.
    \end{proof}
\end{itemize}
\section{Section 4.1}
\begin{itemize}
    \item Problem 3: Show that the finite complementary topology is a topology for any set $X$.\begin{proof}\begin{definition}[Topology for $X$] Let $\T$ be a family of subsets of $X$, $\T$ is called a topology for $X$ if the following conditions are met:\begin{enumerate}
        \item The set $X$ and the empty set $\emptyset$ belong to $\T$.
        \item The union of any family of members of $\T$ is a member of $\T$.
        \item The intersection of any finite family of members of $\T$ is a member of $\T$.
    \end{enumerate}\end{definition}
    Recall that the finite complementary topology is the set containing the following:\begin{itemize}
        \item The empty set, $\emptyset$
        \item All subsets $\O$ of $X$ for which $X\backslash\O$ is a finite set.
    \end{itemize}
    Call this set $\T'$. We must show the three conditions from definition 1 hold for any set $X$.\newline There are 2 cases:\begin{enumerate}[label=\roman*.]
        \item $X$ is finite.
        \item $X$ is infinite.
    \end{enumerate}
    Case i. is trivial, because if $X$ is finite, then the finite complementary topology coincides with the discrete topology, which is clearly a topology.\newline Case ii. requires a bit more work.\begin{enumerate}
        \item $\emptyset$ is contained in the definition. The compliment $X\backslash X=\emptyset$ is finite, and therefore $X$ is contained in $\T$, so condition 1 is met.
        \item Let $A=\bigcup_{i\in I}U_i$ be an arbitrary union of sets whose complements are finite. Then, by De Morgan's law, $$A=X-(\bigcap_{i\in I}V_i)\quad\text{for finite sets }V_i$$ By definition of intersection, $\bigcap_{i\in I}V_i\subseteq V_i$ for all $i\in I$. $V_i$ is finite, therefore so is any of its subsets. Therefore the complement $X\backslash A$ is finite, so $A$ is an open set, thus condition 2 is met.
        \item Consider the intersection of any two finite sets $U_k\cap U_j$. Again using De Morgan's law, this is equivalent to $X\backslash(V_k\cup V_j)$. Since the intersection of two finite sets is finite, $U_k\cap U_j=X\backslash(V_k\cup V_j)$ is contained within $\T$. Since an intersection will \emph{never} increase when you add a set, confirming with 2 sets is sufficient to prove it for any finite number of sets. Therefore condition 3 is met.
    \end{enumerate}
    \end{proof}
\end{itemize}
\section{Section 4.2}
\begin{itemize}
    \item 1, part (b): Let $A,B$ be subsets of a space $X$. Show that $\overline{A\cap B}\subset\bar{A}\cap\bar{B}$.\begin{proof}We know that $A\cap B\subset A$, by definition, and $A\cap B\subset B$, also by definition. Also, $A\subset\bar{A}$ and $B\subset\bar{B}$. (Since $\bar{H}$ is the smallest closed set containing $H$). Thus we can conclude that $A\cap B$ is a subset of each of $\bar{A}$ and $\bar{B}$. Therefore, $\overline{A\cap B}\subset\bar{A}\cap\bar{B}$.
    \end{proof}
\end{itemize}
\section{Section 4.5}
\begin{itemize}
    \item Problem 4: Let $f:X\to Y$ be a continuous function on the indicated spaces and $A$ a subspace of $X$. Prove that the restriction $f|_A:A\to Y$ of $f$ to $A$ is continuous.\begin{proof} Let $(X,\T)$ be one space, and $(Y,\mathcal{J})$ the other. Consider some $H\in\mathcal{J}$, then $f|_A(H)=f^{-1}(H)\cap A$ where $f^{-1}(H)$ is open. Therefore $f^{-1}(H)\cap A$ is open in the subspace. Thus $f|_A:X\to Y$ is continuous.
    \end{proof}
    \item Problem 9: Prove that a finite subset $A$ of a Hausdorff space $X$ has no limit points. Conclude that $A$ must be closed.\begin{proof} Let $A=\{x_1,x_2,\dots,x_n\}$ and let $p$ be a limit point of $A$. This means for every open set $U$ containing $p$, it must also contain at least one other element of $A$. In mathematical terms: $$U\cap A\backslash\{p\}\neq\emptyset$$ Since $X$ is Hausdorff, there are open sets $U_i$ containing $p$ and open sets $V_i$ containing $x_i$ such that $U_i\cap V_i=\emptyset$ by definition. Let $U=\bigcap_{i=1}^nU_i$ and let $V=\bigcup_{i=1}^nV_i$. Then the following are true:\begin{enumerate}
        \item $p\in U$
        \item $A\subset V$
        \item $U\cap V=\emptyset$
    \end{enumerate}
    Therefore we have found a neighborhood of $p$ which doesn't intersect $A$. Thus, $p$ cannot be a limit point, and there are no limit points of $A$.\newline The second part follows, because $A'=\emptyset$ since $A$ has no limit points, and $\bar{A}=A\cup A'=A$, so $A$ is closed.
    \end{proof}
\end{itemize}
\end{document}