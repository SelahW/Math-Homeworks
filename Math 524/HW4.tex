\documentclass[hidelinks,12pt]{article}
\usepackage[utf8]{inputenc}
\usepackage{microtype}
\usepackage{mathtools}
\usepackage{amsthm}
\usepackage{amsmath}
\usepackage{amsfonts}
\usepackage{amssymb}
\usepackage{centernot}
\usepackage{marvosym}
\usepackage{enumitem}
\usepackage{hyperref}
\setcounter{tocdepth}{1}
\let\marvosymLightning\Lightning
\renewcommand{\geq}{\geqslant}
\renewcommand{\leq}{\leqslant}
\newtheorem{theorem}{Theorem}
\newtheorem{corollary}{Corollary}[theorem]
\newtheorem*{remark}{Remark}
\renewcommand\qedsymbol{QED}
\newcommand{\R}{\mathbb{R}}
\newcommand{\N}{\mathbb{N}}
\newcommand{\Z}{\mathbb{Z}}
\newcommand{\Q}{\mathbb{Q}}
\newcommand{\divby}{%
    \mathrel{\text{\vbox{\baselineskip.65ex\lineskiplimit0pt\hbox{.}\hbox{.}\hbox{.}}}}%
}
\newcommand{\notdivby}{\centernot\divby}
\title{\scalebox{2}{Math 524 Homework 3}}
\author{\scalebox{1.5}{Theo Koss}}
\date{March 2024}

\begin{document}

\maketitle
\begin{enumerate}
    \item Let $f$ be any function on $[a,b]$ with no upper bound. Prove that the upper sum $U(f,P)$ is infinite for every partition $P$ of $[a,b]$. Conclude that $f$ is not integrable.
        \begin{proof}
            \[M_k(f)=sup\left\{f(x)\mid x\in[x_{k-1},x_k]\right\}\]
            Upper sum for partition $P=\{x_0,x_1,x_2,\dots,x_n\}$:
            \[U(P,f)=\sum_{k=1}^nM_k\Delta x_k\]
            However since $f$ has no upper bound on $[a,b]$, we have that for each partition $P$, $M_k$ is infinite, therefore the sum $\sum_{k=1}^nM_k\Delta x_k$ is infinite. Therefore $f$ is not integrable.
        \end{proof}
    \item \begin{enumerate}
            \item Let $f$ be a bounded function on a set $A$, and consider:
                \[M=sup\left\{f(x)\mid x\in A\right\}\quad m=inf\left\{f(x)\mid x\in A\right\}\]
                \[M'=sup\left\{|f(x)|\mid x\in A\right\}\quad m'=inf\left\{|f(x)|\mid x\in A\right\}\]
                Show that $M-m\geq M'-m'$.
                \[M-m=sup(|f(x)-f(y)|\mid x,y\in A)\]
                and
                \[M'-m'=sup(||f(x)|-|f(y)||)\mid x,y\in A\]
                By the triangle inequality, we have:
                \[|f(x)-f(y)|\geq||f(x)-|f(y)||\]
                So,
                \[M-m\geq M'-m'\]
            \item Show that if $f$ is integrable on $[a,b]$ then $|f|$ is also integrable on $[a,b]$.
                Let $f$ be integrable on $[a,b]$, then we have that
                \[\underline{\int_a^b} f(x) d x=\overline{\int_a^b} f(x) d x=B\tag{Where B is a real number}\]
                Then, we have:
                \[U(P,f)=\sum_{k=1}^nM\Delta x_k\quad L(P,f)=\sum_{k=1}^nm\Delta x_k\]
                Subtracting, we get:
                \[U(P,f)-L(P,f)=\sum_{k=1}^n(M-m)\Delta x_k\]
                By above we get:
                \[U(P,f)-L(P,f)\geq U(P,|f|)-L(P,|f|)\]
                And since $f$ is Riemann integrable, we have $U(P,f)-L(P,f)=0$, so
                \[0\geq U(P,|f|)-L(P,|f|)\]
                And all $U(P,|f|)$ are greater than or equal to all $L(P,|f|)$, so we have $U(P,|f|)=L(P,|f|)$ therefore $|f|$ is Riemann integrble.
        \end{enumerate}
    \item Let $f(x)=|x|$ and define $F(x)=\int_{-1}^xf$. Find a piecewise algebraic formula for $F(x)$ for all $x$. Where is $F$ continuous? Where is $F$ differentiable?
        \[f(-2)=|-2|=2\quad F(-2)=\int_{-1}^{-2}f=\]
        Piecewise formula:
        \[F(x)=\begin{cases}
            \int_{-1}^{x}(-f)=\frac{1}{2}(1-x^2) & \text{if }x<0\\
            F(0)+\int_{-1}^{x}(f)=\frac{1}{2}(1+x^2) & \text{if }x>0\\
        \end{cases}\]
        $F$ is continuous everywhere, and it is differentiable everywhere (everywhere except 0 is obvious, at $F(0)$ we get $F'(0)=\lim_{h\to0}\frac{F(x)-F(0)}{h-0}$, and  $|F(x)-F(0)|\leq|x|^2$) Therefore $F$ is differentiable at 0 as well. 
    \item Let $L(x)=\int_1^x\frac{1}tdt$ for all $x>0$.
        \begin{enumerate}
            \item Evalute $L(1)$. Explain why $L$ is differentiable and find $L'(x)$.
                \[L(1)=\int_1^1\frac{1}tdt=0\]
                $L$ is differentiable by the fundamental theorem of calculus, we have $\frac{d}{dx}\int_1^x\frac{1}tdt=\frac{1}{x}$.
            \item If $E:\R\to(0,\infty)$ is given by $E(x)=e^x$, it is known that $E'(x)=e^x$. Let $ln:(0,\infty)\to\R$ denote the inverse function of $E$. Use the inverse function theorem to prove that the derivative of $ln(x)$ equals $\frac{1}x$.
                \begin{proof}
                    By inverse function theorem, we have that for a point $b=f(a)$:
                    \[(f^{-1})'(b)=\frac{1}{f'(a)}=\frac{1}{f'(f^{-1}(b))}\]
                    So,
                    \[\frac{d}{dx}ln(b)=\frac{1}{e^(a)}=\frac{1}{b}\]
                    as required.
                \end{proof}
            \item Explain why $L(x)=ln(x)$ for all $x>0$.\\
                The two functions both have the same derivative (by (a) and (b)), and $L(1)=ln(1)=0$, so they are ``aligned". It breaks for $x\leq0$ because that is not in the domain of $ln:(0,\infty)\to\R$.
            \item Let
                \[\gamma_n=(1+\frac{1}2+\frac{1}3+\dots+\frac{1}{n})-L(n)\]
                Prove $(\gamma_n)$ converges.
                \begin{proof}
                    By the integral test we have $(\gamma_n)$ converges iff $\int_1^{\infty}f(x)dx$ converges, where $f(x)=\gamma_x$.\\
                    Suppose $\int_1^{\infty}f(x)dx$ converges, then:
                    \[S_{k+1}-\gamma_1\leq\int_1^{k+1}f(x)dx\leq\int_1^{\infty}f(x)dx\]
                    So $S_{k+1}$ is bounded, and we have by induction that $(\gamma_k)$ is increasing, therefore $(\gamma_k)$ converges.
                \end{proof}
            \item Show how consideration of the sequence $(\gamma_{2n}-\gamma_n)$ leads to a new proof of the identity:
                \[L(2)=1-\frac{1}{2}+\frac{1}{3}-\frac{1}{4}+\dots\]
                RHS is equal to \[\sum_{n=1}^{2m}(-1)^{n+1}\frac{1}{n}\]
                Splitting into even and odd terms gives:
                \[=\sum_{n=1}^{m}\frac{1}{2n-1}-\sum_{n=1}^m\frac{1}{2n}\]
                Which is
                \[\gamma_{2m}-2\gamma_m+L(2)\]
                Which limits to $L(2)$.
        \end{enumerate}
    \item If $\vec{u}$ and $\vec{v}$ are vectors in $\R^3$, use the triangle inequality to prove that:
        \begin{enumerate}
            \item $\|\vec{u}\|-\|\vec{v}\|\leq\|\vec{u}-\vec{v}\|$.\\
                By theorem 9.2.2 in the text, we have that $\|\vec{u}+\vec{v}\|\leq\|\vec{u}\|+\|\vec{v}\|$.
                \[\|\vec{u}\|=\|u-v+v\|\leq\|u-v\|+\|v\|\tag{By triangle inequality}\]
                Subtract $\|\vec{v}\|$ from both sides:
                \[\|\vec{u}\|-\|\vec{v}\|\leq\|\vec{u}-\vec{v}\|\]
            \item $|\|\vec{u}\|-\|\vec{v}\||\leq\|\vec{u}-\vec{v}\|$
                \[\|\vec{v}\|=\|v-u+u\|\leq\|v-u\|+\|u\|\tag{By triangle inequality}\]
                Subtracting the two above equations gives:
                \[\|\vec{u}\|-\|\vec{v}\|\leq\|u-v\|+\|v\|-(\|v-u\|+\|u\|)\]
                \[\implies |\|\vec{u}\|-\|\vec{v}\||\leq\|\vec{u}-\vec{v}\|\]
                (Taking abs value forces $\|u-v\|=\|v-u\|$)
        \end{enumerate}
    \item Suppose $f:[a,b]\to\R$ is differentiable and $f'$ is continuous. The \emph{graph} of $f$ is the curve $y=f(x)$ in $\R^2$. Show that the arclength $L$ of the graph of $f$ for $a\leq x\leq b$ is given by:
        \[L=\int_a^b\sqrt{1+(dy/dx)^2}dx\]
        \begin{proof}
            The \emph{graph} of $f$ is the same as a smooth parameterization of $f$, which looks like $y=(t,f(t))$. Then, we have that the length of $y$ is defined by:
            \[L(y)=sup\left\{L_Q\mid Q\text{ is a partition of }[a,b]\right\}\]
            Using this smooth parameterization, we can invoke Theorem 9.7.2, it states that the length of a smooth parameterization $\gamma(t)=(f(t),g(t))$for $t\in[a,b]$ is finite and equal to:
            \[L(\gamma)=\int_a^b\sqrt{[f'(t)]^2+[g'(t)]^2}dt\]
            In our case, we have $y=(t,f(t))$, so the arc length is given by $t'=1$, and $f'(t)=dy/dx$:
            \[L=\int_a^b\sqrt{1+(dy/dx)^2}dx\]
        \end{proof}

\end{enumerate}
\end{document}
