\documentclass[hidelinks,12pt]{article}
\usepackage[utf8]{inputenc}
\usepackage{microtype}
\usepackage{mathtools}
\usepackage{amsthm}
\usepackage{amsmath}
\usepackage{amsfonts}
\usepackage{amssymb}
\usepackage{centernot}
\usepackage{marvosym}
\usepackage{enumitem}
\usepackage{hyperref}
\setcounter{tocdepth}{1}
\let\marvosymLightning\Lightning
\renewcommand{\geq}{\geqslant}
\renewcommand{\leq}{\leqslant}
\newtheorem{theorem}{Theorem}
\newtheorem{corollary}{Corollary}[theorem]
\newtheorem*{remark}{Remark}
\renewcommand\qedsymbol{QED}
\newcommand{\R}{\mathbb{R}}
\newcommand{\N}{\mathbb{N}}
\newcommand{\Z}{\mathbb{Z}}
\newcommand{\Q}{\mathbb{Q}}
\newcommand{\divby}{%
  \mathrel{\text{\vbox{\baselineskip.65ex\lineskiplimit0pt\hbox{.}\hbox{.}\hbox{.}}}}%
  }
\newcommand{\notdivby}{\centernot\divby}
\title{\scalebox{2}{Math 524 Homework 2}}
\author{\scalebox{1.5}{Theo Koss}}
\date{Feburary 2024}

\begin{document}

\maketitle
\begin{enumerate}
\item Give another proof of the Comparison Test using the Monotone Convergence Theorem. In other words, assume that $0<a_n \leq b_n$ for all $n$ and that $\sum b_n$ converges. Show that the sequence $\left(s_n\right.$ ) of partial sums of the series $\sum a_n$ is monotone and bounded above.
    \begin{proof}[Proof]
        Let $\sum a_n$ and $\sum b_n$ be series' such that $0<a_n\leq b_n$ for all $n$ and $\sum b_n$ converges. Let $\{s_k\}$ be the sequence of partial sums of $a_n$, and let $L=\sum_{n=1}^{\infty} b_n$. We must check 2 things: \begin{enumerate}
            \item $\{s_k\}$ is monotone increasing. Since we are given $a_n>0$ (strictly), $s_{k+1}>s_k$ for all $k$. Thus $\{s_k\}$ is monotone increasing.
            \item $\{s_k\}$ is bounded above. We are given $a_n\leq b_n$ for all $n$, so \[s_k=\sum_{n=1}^k a_n\leq\sum_{n=1}^k b_n\leq\sum_{n=1}^{\infty} b_n=L\]
                So $\{s_k\}$ is bounded above.
        \end{enumerate}
        Therefore $\{s_k\}$ converges and thus so does $\sum a_n$ 
    \end{proof}
\item Give another proof of the Alternating Series Test using Cantor's Nested Intervals Theorem: If $\left[a_1, b_1\right] \supset\left[a_2, b_2\right] \supset \cdots$ is a nested sequence of closed, bounded intervals and $\lim _{n \rightarrow \infty}\left|b_n-a_n\right|=$ 0 then the intersection $\bigcap\left[a_n, b_n\right]$ consists of a single point.
\begin{proof}[Proof]
    Let $\sum a_n$ where $a_n=(-1)^nb_n$ and $b_n\geq0\ \forall n$. Such that $\lim_{n\to\infty}b_n=0$ and $\{b_n\}$ is a decreasing sequence. We must show that $\sum a_n$ is convergent.
    \par\null\par
    To construct a set of nested closed, bounded intervals, we will look at the even and odd partial sums of $a_n$. 
    \[\{s_{2k}=s_2,s_4,s_6,\dots,s_{2k}\}\]
    \[\{s_{2k-1}=s_1,s_3,s_5,\dots,s_{2k-1}\}\]
    Each $s$ looks like:
    \[s_2=a_1-a_2,\ s_4=(a_1-a_2)+(a_3-a_4),\dots\]
    \[s_1=a_1,\ s_3=a_1-(a_2-a_3),\dots\]
    Observe that $s_{2k}$ (even partial sums) are monotone increasing. (Also, since these partial sums are never infinite, we never run into the grouping terms problem which was shown in class). And similarly, $s_{2k+1}$ are monotone decreasing.\\
    Now setting up our intervals, we have 
    \[[0,s_1]\supset[s_2,s_3]\supset[s_4,s_5]\supset\dots\supset[s_{2k},s_{2k+1}]\]
    Which are nested, closed, and bounded, and we were given that\\ $\lim_{n\to\infty}b_n=0$. Therefore we have $\bigcap[s_{2k},s_{2k+1}]\ni x$ for some $x$. Therefore $\sum a_n$ converges.
\end{proof}
\item In this problem, you will examine a specific rearrangement of a conditionally convergent series for which you will see that the value of the sum has definitely changed!
Let $S$ be the sum of the alternating harmonic series: $S=1-\frac{1}{2}+\frac{1}{3}-\frac{1}{4}+\cdots$. Multiplying each term by $1 / 2$ gives a new series that converges to $S / 2$. Add these two series together to get a third series. Show that this third series is a rearrangement of the alternating harmonic series, but that its sum is now $(3 / 2) S$.
\begin{align*}    
    S&=1-\frac{1}{2}+\frac{1}{3}-\frac{1}{4}+\dots\\
    \frac{S}{2}&=\frac{1}{2}-\frac{1}{4}+\frac{1}{6}+\dots\\
    \frac{3}{2}S&=1+\frac{1}{3}-\frac{2}{4}+\frac{1}{5}+\frac{1}{7}-\frac{2}{8}+\dots\\
                &=1-\frac{1}{2}+\frac{1}{3}-\frac{1}{4}+\dots\\
\end{align*}
The even terms will cancel out if mutliple of 2 (but not 4), and will add together if multiple of 4. (Giving the one which was cancelled out always)
\item Consider the infinite series $\sum_{n=0}^{\infty} \frac{1}{2 n+1}$. Determine whether this series converges or diverges. Explain.\\
    Using the limit comparison test with the harmonic series:
    \[\lim_{n\to\infty}\frac{\frac{1}{2n+1}}{\frac{1}{n}}=\lim_{n\to\infty}\frac{n}{2n+1}=\frac{1}{2}\] Which is finite and nonzero, and since the harmonic series diverges, so does this series.
\item For each power series, find the set of all values of $x$ for which the series converges or diverges. [Hint: The Ratio Test is often useful for finding an interval of convergence, but there may be points at which it gives an inconclusive result.]
    \begin{enumerate}
\item $\sin x=x-\frac{x^3}{3 !}+\frac{x^5}{5 !}-\frac{x^7}{7 !}+\cdots$
\item $\tanh ^{-1} x=x+\frac{x^3}{3}+\frac{x^5}{5}+\frac{x^7}{7}+\cdots$
    \end{enumerate}
    \end{enumerate}
\end{document}
