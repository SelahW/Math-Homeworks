\documentclass[a4paper,17pt]{extarticle}
\usepackage{geometry}
\usepackage[utf8]{inputenc}
\usepackage[table,xcdraw]{xcolor}
\usepackage{mathtools}
\usepackage{amsthm}
\usepackage{amsmath}
\usepackage{amsfonts}
\usepackage{amssymb}
\usepackage{centernot}
\usepackage{marvosym}
\usepackage{enumitem}
\usepackage{hyperref}
\setcounter{tocdepth}{1}
\theoremstyle{definition}
\newtheorem{definition}{Definition}
\let\marvosymLightning\Lightning
\renewcommand{\skip}{\par\null\par}
\newcommand{\T}{\mathcal T}
\renewcommand{\O}{\mathcal{O}}
\renewcommand{\Lightning}{\scalebox{1.5}{\marvosymLightning}}
\newtheorem{theorem}{Theorem}
\newtheorem{corollary}{Corollary}[theorem]
\newtheorem*{remark}{Remark}
\renewcommand\qedsymbol{QED}
\newcommand{\R}{\mathbb{R}}
\newcommand{\N}{\mathbb{N}}
\newcommand{\Z}{\mathbb{Z}}
\newcommand{\divby}{%
  \mathrel{\text{\vbox{\baselineskip.65ex\lineskiplimit0pt\hbox{.}\hbox{.}\hbox{.}}}}%
  }
\newcommand{\notdivby}{\centernot\divby}
\title{\scalebox{2}{Math 553 Homework 1}}
\author{\scalebox{1.5}{Theo Koss}}
\date{February 2022}
\begin{document}
\maketitle
\section{Section 1.3}
\begin{itemize}
    \item Problem 1: Find a parameterized curve $\alpha(t)$ whose trace is the circle $x^2+y^2=1$ such that $\alpha(t)$ runs clockwise around the circle and $\alpha(0)=(0,1)$. $$\alpha(t)=(\sin(t),\cos(t))$$
    \item Problem 2: Let $\alpha(t)$ be a parameterized curve which does not pass through the origin. If $\alpha(t_0)$ is the point of the trace of $\alpha$ closest to the origin and $\alpha'(t_0)\neq0$. Show that the position vector $\alpha(t_0)$ is orthogonal to $\alpha'(t_0)$. \begin{proof}N2S: $\alpha(t_0)\cdot\alpha'(t_0)=0$.\newline We know from the problem that $$|\alpha(t_0)|<\alpha(t)$$ for all $t$ in the domain except $t_0$. This implies $$|\alpha(t_0)|^2<|\alpha(t)|^2$$ Again for all $t$ except $t_0$. This inequality implies that $\alpha(t)\cdot\alpha(t)$ is minimized at $t_0$. Therefore the derivative equals 0: $$\frac{d}{dt}[\alpha(t_0)\cdot\alpha(t_0)]=2(\alpha'(t_0)\cdot\alpha(t_0))=0\implies \alpha'(t_0)\cdot\alpha(t_0)=0$$ As required.
    \end{proof}
    \item Problem 3: A parameterized curve $\alpha(t)$ has the property that its second derivative $\alpha''(t)$ is identically 0. What can be said about the curve?\skip $\alpha(t)$ must be a straight line, since the first derivative will be constant, then the second derivative must be 0.
    \item Problem 5: Let $\alpha:I\to\R^3$ be a parameterized curve, with $\alpha'(t)\neq0$ for all $t\in I$. Show that $|\alpha(t)|$ is a nonzero constant iff $\alpha(t)$ is orthogonal to $\alpha'(t)$ for all $t\in I$. \begin{proof}$(\Longrightarrow):$ Assume $|\alpha(t)|$ is a nonzero constant, then $$|\alpha(t)|=C>0$$ then, $$\alpha(t)\cdot\alpha(t)=|\alpha(t)|^2=C^2$$ Is also constant, and $$\alpha'(t)\cdot\alpha(t)=\frac{1}{2}\frac{d}{dt}[\alpha(t)\cdot\alpha(t)]=\frac{1}{2}\frac{d}{dt}C^2=0$$ Therefore $\alpha'(t)\cdot\alpha(t)=0$ as required.\skip $(\Longleftarrow):$ Assume $\alpha'(t)\cdot\alpha(t)=0$, then: $$\alpha'(t)\cdot\alpha(t)=\frac{1}{2}\frac{d}{dt}[\alpha(t)\cdot\alpha(t)]=0$$ implies $$\alpha(t)\cdot\alpha(t)=C^2$$ For some real $C$. Then $C>0$ because $C=0\implies\alpha'(t)=0$ contradiction to what the problem says. Then we have showed that $$|\alpha(t)|^2=C^2>0\implies \alpha(t)=C>0$$ As required.
    \end{proof}
\end{itemize}
\section{Section 1.3}
\begin{itemize}
    \item Problem 1: Show that the tangent lines with the regular parameterized curve $\alpha(t)=(3t,2t^2,2t^3)$ make a constant angle with the line $y=0, z=x$ (equivalent to $(x,0,x)$).\begin{proof}The direction of the given line is $u=(1,0,1)$. Now we must show that $\theta=\arccos\left(\frac{(u\cdot v)}{|u||v|}\right)$ is constant, where $v$ is the tangent line, $v=(3,4t,6t^2)$. $$\theta=\arccos\left(\frac{(u\cdot v)}{|u||v|}\right)=\arccos\left(\frac{3+6t^2}{\sqrt{18+32t^2+72t^4}}\right)$$ Is not constant?? Although I notice that it \emph{is} constant if instead $\alpha(t)=(3t,3t^2,2t^3)$, then $\theta=\arccos\left(\frac{3+6t^2}{\sqrt{18+72t^2+72t^4}}\right)=\frac{\pi}{4}$
    \end{proof}
    \item Problem 2: A circular disk of radius 1 in the $xy$ plane rolls without slipping along the $x$ axis.\begin{enumerate}[label=\alph*.]
        \item Obtain a parameterized curve $\alpha:\R\to\R^2$ the trace of which is the cycloid, and determine its singular points. $$\alpha(t)=(t-sint,1-cost)\implies \alpha'(t)=(1-cost,sint)$$ To find singular points, set $\alpha'(t)=0$. $$1-cost=0\quad sint=0\implies t=2\pi k,k\in\Z$$
        \item Compute the arc length of the cycloid corresponding to a complete rotation of the disk. $$L|_0^{2\pi}=\int_{0}^{2\pi}|\alpha'(t)|dt$$ $$=\int_{0}^{2\pi}\sqrt{(1-cost)^2+(sint)^2}dt$$ $$=\int_{0}^{2\pi}\sqrt{1-2cost+cos^2t+sin^2t}dt$$ $$=\int_{0}^{2\pi}\sqrt{2-2cost}dt$$ $$=2\int_{0}^{2\pi}\sqrt{\frac{1-cost}{2}}dt$$ $$=2\int_{0}^{2\pi}sin\left(\frac{t}{2}\right)dt$$ $$=-4cos\left(\frac{t}{2}\right)|_{0}^{2\pi}$$ $$=-4(-1-1)=8$$
    \end{enumerate}
    \item Problem 10: Let $\alpha:I\to\R^3$ be a parameterized curve. Let $[a,b]\subset I$ and set $\alpha(a)=p,\alpha(b)=q$.\begin{enumerate}[label=\alph*.]
        \item Show that for any constant vector $v$, $|v|=1$, $$(q-p)\cdot v=\int_{a}^b\alpha'(t)\cdot v\ dt\leq\int_{a}^b|\alpha'(t)|dt$$\begin{proof}By the Cauchy-Schwarz inequality: $$\alpha'(t)\cdot v\leq|\alpha'(t)||v|=|\alpha'(t)|$$ Therefore $$\int_{a}^b\alpha'(t)\cdot v\ dt\leq\int_{a}^b|\alpha'(t)|dt$$ As required\end{proof}
        \item Set $v=\frac{q-p}{|q-p|}$ and show that $|\alpha(b)-\alpha(a)|\leq\int_a^b|\alpha'(t)|dt$. That is, the curve of shortest length between $\alpha(a)$ and $\alpha(b)$ is the straight line joining these two points.\begin{proof}Inserting $v=\frac{q-p}{|q-p|}$, $p=\alpha(a)$ and $q=\alpha(b)$ into the inequality from a. gives us:$$\int_a^b|\alpha'(t)|dt\geq(q-p)\cdot v$$ $$=(\alpha(b)-\alpha(a))\cdot\frac{q-p}{|q-p|}$$ $$=\frac{|\alpha(b)-\alpha(a)|^2}{|\alpha(b)-\alpha(a)|}$$ $$=|\alpha(b)-\alpha(a)|$$ As required.\end{proof}
    \end{enumerate}
\end{itemize}
\end{document}