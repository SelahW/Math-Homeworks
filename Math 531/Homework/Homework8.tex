\documentclass[hidelinks,12pt]{article}
\usepackage[utf8]{inputenc}
\usepackage[table,xcdraw]{xcolor}
\usepackage{mathtools}
\usepackage{amsthm}
\usepackage{amsmath}
\usepackage{amsfonts}
\usepackage{amssymb}
\usepackage{centernot}
\usepackage{marvosym}
\usepackage{enumitem}
\usepackage{hyperref}
\setcounter{tocdepth}{1}
\usepackage{chngcntr}
\counterwithin*{equation}{section}
\counterwithin*{equation}{subsection}
\let\marvosymLightning\Lightning
\newtheorem{theorem}{Theorem}
\newtheorem{corollary}{Corollary}[theorem]
\newtheorem*{remark}{Remark}
\renewcommand\qedsymbol{QED}
\newcommand{\C}{\mathbb{C}}
\newcommand{\R}{\mathbb{R}}
\newcommand{\N}{\mathbb{N}}
\newcommand{\Z}{\mathbb{Z}}
\newcommand{\Q}{\mathbb{Q}}
\newcommand{\divby}{%
  \mathrel{\text{\vbox{\baselineskip.65ex\lineskiplimit0pt\hbox{.}\hbox{.}\hbox{.}}}}%
  }
\newcommand{\notdivby}{\centernot\divby}
\title{\scalebox{2}{Math 531 Homework 8}}
\author{\scalebox{1.5}{Theo Koss}}
\date{April 2021}
\begin{document}
\maketitle
\section{Section 4.1}
\begin{itemize}
    \item Problem 1: Let $f\left(x\right),g\left(x\right),h\left(x\right)\in F[x]$. Show that the following hold:\begin{enumerate}[label=(\alph*)]
        \item If $g\left(x\right)|f\left(x\right)$, and $h\left(x\right)|g\left(x\right)$, then $h\left(x\right)|f\left(x\right)$.\newline Since $g\left(x\right)|f\left(x\right)$, it follows that $f\left(x\right)=g\left(x\right)a\left(x\right)$ for some $a\left(x\right)\in F\left(x\right)$. Similarly, $g\left(x\right)=h\left(x\right)b\left(x\right)$, for some $b\left(x\right)\in F\left(x\right)$. Thus, $f\left(x\right)=h\left(x\right)b\left(x\right)a\left(x\right)$, and since $b\left(x\right)a\left(x\right)\in F\left(x\right)$, this shows $h\left(x\right)|f\left(x\right)$.
        \item If $h\left(x\right)|f\left(x\right)$, and $h\left(x\right)|g\left(x\right)$, then $h\left(x\right)|\left(f\left(x\right)\pm g\left(x\right)\right)$.\newline Since $h\left(x\right)|f\left(x\right)$ and $h\left(x\right)|g\left(x\right)$, $f\left(x\right)=h\left(x\right)a\left(x\right)$ and $g\left(x\right)=h\left(x\right)b\left(x\right)$. Then $\left(f\left(x\right)+g\left(x\right)\right)=h\left(x\right)\left(a\left(x\right)+b\left(x\right)\right)$, thus $h\left(x\right)|\left(f\left(x\right)+ g\left(x\right)\right)$. And $\left(f\left(x\right)-g\left(x\right)\right)=h\left(x\right)\left(a\left(x\right)-b\left(x\right)\right)$ so $h\left(x\right)|\left(f\left(x\right)-g\left(x\right)\right)$, therefore $h\left(x\right)|\left(f\left(x\right)\pm g\left(x\right)\right)$.
        \item If If $g\left(x\right)|f\left(x\right)$, then $g\left(x\right)\cdot h\left(x\right)|f\left(x\right)\cdot h\left(x\right)$.\newline Since $g\left(x\right)|f\left(x\right)$, $f\left(x\right)=g\left(x\right)a\left(x\right)$. Then $f\left(x\right)\cdot h\left(x\right)=g\left(x\right)a\left(x\right)h\left(x\right)$, thus, $g\left(x\right)\cdot h\left(x\right)|f\left(x\right)\cdot h\left(x\right)$.
        \item If $g\left(x\right)|f\left(x\right)$ and $f\left(x\right)|g\left(x\right)$, then $f\left(x\right)=kg\left(x\right)$ for some $k\in F$.\newline Since $g\left(x\right)|f\left(x\right)$, $f\left(x\right)=g\left(x\right)a\left(x\right)$, and since $f\left(x\right)|g\left(x\right)$, $g\left(x\right)=f\left(x\right)b\left(x\right)$.
    \end{enumerate}
    \item Problem 5: Over the given field $F$, write $f\left(x\right)=q\left(x\right)\left(x-c\right)+f\left(c\right)$ for:\begin{enumerate}[label=(\alph*)]
        \item $f\left(x\right)=2x^3+x^2-4x+3;c=1;F=\Q$.$$f\left(x\right)-f\left(1\right)=\left(2x^3+x^2-4x+3\right)-\left(2+1-4+3\right)$$ $$=\left(2x^3+x^2-3\right)+\left(4x-4\right)$$ $$=\left(x-1\right)\left(2x^2+3x+3\right)+4\left(x-1\right)$$ $$\left(2x^2+3x+7\right)\left(x-1\right)$$ Thus $f\left(x\right)=\left(2x^2+3x+7\right)\left(x-1\right)+2$.
        \item $f\left(x\right)=x^3-5x^2+6x+5;c=2;F=\Q$.$$f\left(x\right)-f\left(2\right)=\left(x^3-5x^2+6x+5\right)-\left(8-20+12+5\right)$$ $$=\left(x^3-5x^2+12\right)+\left(6x-12\right)$$ $$=\left(x-2\right)\left(x^2-3x-6\right)+6\left(x-2\right)$$ $$=\left(x^2-3x\right)\left(x-2\right)$$ Thus $f\left(x\right)=\left(x^2-3x\right)\left(x-2\right)+5$.
        \item $f\left(x\right)=x^3+1;c=1;F=\Z_3$. $$f\left(x\right)-f\left(1\right)=\left(x^3+1\right)-\left(1+1\right)$$ $$=x^3-1=\left(x-1\right)\left(x^2+x+1\right)$$ Thus $f\left(x\right)=\left(x^2+x+1\right)\left(x-1\right)+2$
        \item $f\left(x\right)=x^3+2x+3;c=2;F=\Z_5$.$$f\left(x\right)-f\left(2\right)=\left(x^3+2x+3\right)-\underbrace{\left(8+4+3\right)}_{\equiv0\mod{5}}$$ $$=\left(x^2+2x+6\right)\left(x-2\right)$$ Thus $f\left(x\right)=\left(x^2+2x+6\right)\left(x-2\right)$
    \end{enumerate}
    \item Problem 15: Show that the set of matrices of the form $\begin{bmatrix}
a & b\\
-b & a
\end{bmatrix}$ is a field under the operations of matrix addition and multiplication.\begin{proof}
Call this set $S$, and consider the elements of this set, $\alpha=\begin{bmatrix}
a & b\\
-b & a
\end{bmatrix}$, $\beta=\begin{bmatrix}
c & d\\
-d & c
\end{bmatrix}$, and $\gamma=\begin{bmatrix}
e & f\\
-f & e
\end{bmatrix}$ To show this is a field, we N2S:\begin{enumerate}
    \item Closure under addition and multiplication.
    \item Associativity of matrix addition and multiplication.
    \item Commutativity of matrix addition.
    \end{enumerate} Numbers 1-3 are clearly true with this set. Therefore we must only show \begin{enumerate}
        \item Commutativity of matrix mulitplication.\newline $\alpha\beta=\begin{bmatrix}
 ac-bd & ad+bc\\
-bc-ad & -bd+ac
\end{bmatrix}\in S$, $\beta\alpha=\begin{bmatrix}
 ca-db & da+cb\\
-cb-da & -db+ca
\end{bmatrix}\in S$, since $\alpha\beta=\beta\alpha \quad\forall \alpha,\beta\in S$, matrix multiplication on this set is commutative.
        \item Distributivity of multiplication over addition.\newline N2S, $\alpha\left(\beta+\gamma\right)=\alpha\beta+\alpha\gamma$, $\forall \alpha,\beta, \gamma\in S$. $$\alpha\left(\beta+\gamma\right)=\begin{bmatrix}
a & b\\
-b & a
\end{bmatrix}\left(\begin{bmatrix}
c+e & d+f\\
-d-f & c+e
\end{bmatrix}\right)$$$$=\begin{bmatrix}
a\left(c+e\right)+b\left(-d-f\right) & a\left(d+f\right)+b\left(c+e\right)\\
-b\left(c+e\right)+a\left(-d-f\right) & -b\left(d+f\right)+a\left(c+e\right)
\end{bmatrix}$$ $$\alpha\beta+\alpha\gamma=\left(\begin{bmatrix}
 ac-bd & ad+bc\\
-bc-ad & -bd+ac
\end{bmatrix}\right)+\left(\begin{bmatrix}
ae-bf & af+be\\
-be-af & -bf+ae
\end{bmatrix}\right)$$$$=\begin{bmatrix}
a\left(c+e\right)+b\left(-d-f\right) & a\left(d+f\right)+b\left(c+e\right)\\
-b\left(c+e\right)+a\left(-d-f\right) & -b\left(d+f\right)+a\left(c+e\right)
\end{bmatrix}$$ Therefore $\alpha\left(\beta+\gamma\right)=\alpha\beta+\alpha\gamma$, $\forall \alpha,\beta, \gamma\in S$. As required.
        \item Existence of identity elements for addition and multiplication.\newline Additive identity, $``0"\in S=\begin{bmatrix}
0 & 0\\
0 & 0
\end{bmatrix}$. Multiplicative identity, $``1"\in S=\begin{bmatrix}
1 & 0\\
0 & 1
\end{bmatrix}$.
        \item Existence of additive inverses.\newline $\forall \alpha=\begin{bmatrix}
a & b\\
-b & a
\end{bmatrix}\in S$, $\exists -\alpha=\begin{bmatrix}
-a & -b\\
b & -a
\end{bmatrix}\in S$.
        \item Existence of multiplicative inverses.\newline $\forall \alpha=\begin{bmatrix}
a & b\\
-b & a
\end{bmatrix}\in S$, $\exists \alpha^{-1}=\frac{1}{det\left(\alpha\right)}\begin{bmatrix}
a & -b\\
b & a
\end{bmatrix}\in S$, and $det\left(\alpha\right)=0$ iff $a,b=0$, which is the additive identity, therefore this is always defined.
\end{enumerate}
\end{proof}
\end{itemize}
\section{Section 4.2}
\begin{itemize}
    \item Problem 3: Find the greatest common divisor of $f\left(x\right)$ and $f'\left(x\right)$, over $\Q$.\begin{enumerate}[label=(\alph*)]
        \item $f\left(x\right)=x^4-x^3-x+1=\left(x-1\right)^2\left(x+1\right)$. \newline $f'\left(x\right)=4x^3-3x^2-1=\left(x-1\right)\left(2x-1\right)\left(2x+1\right)$. \newline $\gcd\left(f\left(x\right),f'\left(x\right)=\left(x-1\right)\right)$.
        \item $f\left(x\right)=x^3-3x-2=\left(x+1\right)\left(x+1\right)\left(x-2\right)$.\newline$f'\left(x\right)=3x^2-3=3\left(x+1\right)\left(x-1\right)$.\newline$\gcd\left(f\left(x\right),f'\left(x\right)\right)=\left(x+1\right)$.
        \item $f\left(x\right)=x^3+2x^2-x-2=\left(x+2\right)\left(x+1\right)\left(x-1\right)$.\newline$f'\left(x\right)=\underbrace{3x^2+4x-1}_{\text{irreducible}}$\newline$\gcd\left(f\left(x\right),f'\left(x\right)\right)=1$. ?
        \item $f\left(x\right)=x^4+2x^3+3x^2+2x+1$ is irreducible, therefore $\gcd\left(f\left(x\right),f'\left(x\right)\right)=1$.
    \end{enumerate}
    \item Problem 11: Find the irreducible factors of $x^6-1$ over $\R$. \newline Useful equations:\begin{equation}
        a^2-b^2=\left(a+b\right)\left(a-b\right)
    \end{equation}\begin{equation}
        a^3-b^3=\left(a-b\right)\left(a^2+b^2+ab\right)
    \end{equation}\begin{equation}
        a^3+b^3=\left(a+b\right)\left(a^2+b^2-ab\right)
    \end{equation} Let $f\left(x\right)=x^6-1$, $$f\left(x\right)=x^6-1$$ $$=\left(x^3\right)^2-\left(1\right)^2$$ $$=\left(x^3+1\right)\left(x^3-1\right)$$ $$=\left(x^3+1^3\right)\left(x^3-1^3\right)$$ $$=\left(x+1\right)\left(x^2+1-x\right)\left(x-1\right)\left(x^2+1+x\right)$$ $$=\left(x+1\right)\left(x-1\right)\left(x^2-x+1\right)\left(x^2+x+1\right)$$ This cannot be factorized further with real coefficients.
    \item Problem 17: Show that for any real number $a\neq0$, the polynomial $x^n-a$ has no multiple roots in $\R$.\begin{proof}
    Assume, for sake of contradiction, that $f$ has a multiple root, and let $\beta\in\R$ be that root. Then $f\left(\beta\right)=0$, $f'\left(\beta\right)=0$. Plugging in, we see:\setcounter{equation}{0}\begin{equation}f\left(\beta\right)=0\implies \beta^n-a=0\implies \beta^n=a\end{equation} And \begin{equation}f'\left(\beta\right)=0\implies n\beta^{n-1}=0\end{equation} and since $n\neq0$, this means $\beta=0$, and by (1), $\beta^n=a$ and therefore $a=0$. {\Large\Lightning}
    \end{proof}
\end{itemize}
\end{document}