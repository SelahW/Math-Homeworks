\documentclass[hidelinks,12pt]{article}
\usepackage[utf8]{inputenc}
\usepackage[table,xcdraw]{xcolor}
\usepackage{mathtools}
\usepackage{amsthm}
\usepackage{amsmath}
\usepackage{amsfonts}
\usepackage{amssymb}
\usepackage{centernot}
\usepackage{marvosym}
\usepackage{enumitem}
\usepackage{hyperref}
\setcounter{tocdepth}{1}
\usepackage{chngcntr}
\counterwithin*{equation}{section}
\counterwithin*{equation}{subsection}
\let\marvosymLightning\Lightning
\newtheorem{theorem}{Theorem}
\newtheorem{corollary}{Corollary}[theorem]
\newtheorem*{remark}{Remark}
\renewcommand\qedsymbol{QED}
\newcommand{\C}{\mathbb{C}}
\newcommand{\R}{\mathbb{R}}
\newcommand{\N}{\mathbb{N}}
\newcommand{\Z}{\mathbb{Z}}
\newcommand{\Q}{\mathbb{Q}}
\newcommand{\divby}{%
  \mathrel{\text{\vbox{\baselineskip.65ex\lineskiplimit0pt\hbox{.}\hbox{.}\hbox{.}}}}%
  }
\newcommand{\notdivby}{\centernot\divby}
\title{\scalebox{2}{Math 531 Homework 9}}
\author{\scalebox{1.5}{Theo Koss}}
\date{April 2021}
\begin{document}
\maketitle
\section{Section 4.3}
\begin{itemize}
    \item Problem 1: Let $F$ be a field. Given $p(x)\in F[x]$, prove that congruence modulo $p(x)$ defines an equivalence relation on $F[x]$.\begin{proof}We N2S 3 things,\begin{enumerate}[label=(\arabic*)]
        \item Reflexivity: For any $f(x)\in F[x]$, $$f(x)-f(x)=0\equiv0\mod{p(x)}$$ $$\implies f(x)\equiv f(x)\mod{p(x)}$$ Therefore this relation is reflexive.
        \item Symmetry: Let $f(x),g(x)\in F[x]$.\newline Assume that $f(x)\equiv g(x)\mod{p(x)}$. $$\implies p(x)|f(x)-g(x)$$ $$\implies p(x)|-[f(x)-g(x)]$$ $$\implies p(x)|g(x)-f(x)$$ $$g(x)=f(x)\mod{p(x)}$$Therefore congruence modulo $p(x)$ is symmetric.
        \item Transitivity: Let $f(x),g(x),h(x)\in F[x]$.\newline Assume that $f(x)\equiv g(x)\mod{p(x)}$ and $g(x)\equiv h(x)\mod{p(x)}$. $$\implies p(x)|f(x)-g(x), \text{ and }p(x)|g(x)-h(x)$$ $$\implies p(x)|[f(x)-g(x)]+[g(x)-h(x)]$$ $$\implies p(x)|f(x)-h(x)$$ $$\implies f(x)\equiv h(x)\mod{p(x)}$$ Therefore congruence modulo $p(x)$ is transitive.
    \end{enumerate}
    Since 1,2,3 are true, congruence modulo $p(x)$ is an equivalence relation.
    \end{proof}
    \item Problem 3: Let $E$ be a field, and $F$ a subfield of $E$. Prove that the multiplicative identity of $F$ must be the same as that of $E$.\begin{proof}
    Call the multiplicative identity of $F$, $1_F$, and call that of $E$, $1_E$. $$1_E\cdot 1_F=1_F$$ Since $1_F$ belongs to $E$, and $F$ is a subfield of $E$. Also, $$1_E\cdot 1_F=1_E$$ Because $1_F$ is the identity of $F$, by definition, this is true. Therefore, $$1_E\cdot 1_F=1_F=1_E$$As required.
    \end{proof}
    \item Problem 11: Let $F$ be any field. Prove that the field of $n\times n$ scalar matrices over $F$ is isomorphic to $F$.\begin{proof}
    Let $F$ be any field, then let $F'$ be the field of $n\times n$ scalar matrices with inputs from $F$. $$F'=\left\{\begin{bmatrix}
    a & 0 & \dots & 0 \\
    0 & a \\
    \vdots && \ddots \\
    0 &&& a
    \end{bmatrix}|a\in F\right\}$$
    Define a map $\phi:F'\to F$. $$\phi(a)=\phi(A)=\phi\begin{bmatrix}
    a & 0 & \dots & 0 \\
    0 & a \\
    \vdots && \ddots \\
    0 &&& a
    \end{bmatrix}\longrightarrow a$$
    N2S:\begin{enumerate}
        \item $\phi$ is a homomorphism:\newline Let $A,B\in F'$, where $A=(a)$ and $B=(b)$. Then $$\phi(A+B)=\phi((a)+(b))=a+b=\phi(a)+\phi(b)=\phi(A)+\phi(B)$$ And $$\phi(AB)=\phi(ab)=ab=\phi(a)\phi(b)=\phi(A)\phi(B)$$ Thus $\phi$ is a homomorphism.
        \item $\phi$ is 1-1:\newline Let $$\phi(A)=\phi(B)$$ $$\implies\phi(a)=\phi(b)$$ $$\implies a=b$$ Since $A,B\in F'$, we can write them like so: $$A=\begin{bmatrix}
    a & 0 & \dots & 0 \\
    0 & a \\
    \vdots && \ddots \\
    0 &&& a
    \end{bmatrix}, B=\begin{bmatrix}
    b & 0 & \dots & 0 \\
    0 & b \\
    \vdots && \ddots \\
    0 &&& b
    \end{bmatrix}$$ Since $a=b$ from above, this shows $A=B$. And thus $\phi(A)=\phi(B)$ implies $A=B$.
        \item $\phi$ is onto: Let $a\in F$, then $A=\begin{bmatrix}
    a & 0 & \dots & 0 \\
    0 & a \\
    \vdots && \ddots \\
    0 &&& a
    \end{bmatrix}$, and $\phi(A)=a$. Therefore $\forall a\in F$, $\exists A\in F'$ s.t. $\phi(A)=a$.
    \end{enumerate}
    Now since $\phi:F'\to F$ is a homomorphism, 1-1 and onto, it is an isomorphism.
    \end{proof}
\end{itemize}
\section{Section 4.4}
\begin{itemize}
    \item Problem 3: Find all integer roots of the following equations.\begin{enumerate}[label=(\alph*)]
        \item $x^3+8x^2+13x+6=0$. $x=-1,-6$
        \item $x^3-5x^2-2x+24=0$. $x=-2,3,4$
        \item $x^3-10x^2+27x-18=0$. $x=1,3,6$
        \item $x^4+4x^3+8x+32=0$. $x=-4,-2$
        \item $x^7+2x^5+4x^4-8x^2-32=0$. No integer solutions.
    \end{enumerate}
    \item Problem 13: Verify each of the following, for complex numbers $z$ and $w$.\begin{enumerate}[label=(\alph*)]
        \item $\overline{zw}=\overline{z}\cdot\overline{w}$\newline Let $z=x+iy$, $w=u+iv$. Then $$zw=(x+iy)(u+iv)=xu-yv+i(xv+yu)$$ $$\overline{zw}=(xu-yv)-i(xv+yu)$$ And $$\overline{z}=x-iy,\overline{w}=u-iv$$ $$\overline{z}\cdot\overline{w}=(x-iy)(u-iv)=xu+i(xv+uy)-yv=(xu-yv)-i(xv+yu)$$ Thus $\overline{zw}=\overline{z}\cdot\overline{w}$.
        \item $|zw|=|z||w|$\newline Let $z=x+iy$, $w=u+iv$. Then $$zw=(x+iy)(u+iv)=(xu-yv)+i(xv+yu)$$ $$|zw|=\sqrt{(xu-yv)^2+(xv+yu)^2}=\sqrt{x^2u^2+y^2v^2+x^2v^2+y^2u^2}$$ and $$|z|=\sqrt{x^2+y^2},|w|=\sqrt{u^2+v^2}$$ $$|z||w|=\sqrt{x^2+y^2}\cdot\sqrt{u^2+v^2}=\sqrt{(x^2+y^2)(u^2+v^2)}=\sqrt{x^2u^2+y^2v^2+x^2v^2+y^2u^2}$$ Thus $|zw|=|z||w|$
    \end{enumerate}
\end{itemize}
\end{document}