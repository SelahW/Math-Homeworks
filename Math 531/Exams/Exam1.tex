\documentclass[hidelinks,12pt]{article}
\usepackage[utf8]{inputenc}
\usepackage[table,xcdraw]{xcolor}
\usepackage{mathtools}
\usepackage{amsthm}
\usepackage{amsmath}
\usepackage{amsfonts}
\usepackage{amssymb}
\usepackage{centernot}
\usepackage{marvosym}
\usepackage{enumitem}
\usepackage{hyperref}
\usepackage{graphicx}
\setcounter{tocdepth}{1}
\let\marvosymLightning\Lightning
\newtheorem{theorem}{Theorem}
\newtheorem{corollary}{Corollary}[theorem]
\newtheorem*{remark}{Remark}
\renewcommand\qedsymbol{QED}
\newcommand{\C}{\mathbb{C}}
\newcommand{\R}{\mathbb{R}}
\newcommand{\N}{\mathbb{N}}
\newcommand{\Z}{\mathbb{Z}}
\newcommand{\divby}{%
  \mathrel{\text{\vbox{\baselineskip.65ex\lineskiplimit0pt\hbox{.}\hbox{.}\hbox{.}}}}%
  }
\newcommand{\notdivby}{\centernot\divby}
\title{\scalebox{2}{Math 531 Exam 1}}
\author{\scalebox{1.5}{Theo Koss}}
\date{March 2021}
\begin{document}
\begin{titlepage} % Suppresses displaying the page number on the title page and the subsequent page counts as page 1
	\newcommand{\HRule}{\rule{\linewidth}{0.5mm}} % Defines a new command for horizontal lines, change thickness here
	
	\center % Centre everything on the page
	
	%------------------------------------------------
	%	Headings
	%------------------------------------------------
	
	\textsc{\LARGE University of Wisconsin-Milwaukee}\\[1.5cm] % Main heading such as the name of your university/college
	
	\textsc{\Large Modern Algebra}\\[0.5cm] % Major heading such as course name
	
	\textsc{\large Math 531}\\[0.5cm] % Minor heading such as course title
	
	%------------------------------------------------
	%	Title
	%------------------------------------------------
	
	\HRule\\[0.4cm]
	
	{\huge\bfseries Exam 1}\\[0.4cm] % Title of your document
	
	\HRule\\[1.5cm]
	
	%------------------------------------------------
	%	Author(s)
	%------------------------------------------------
	
	\begin{minipage}{0.4\textwidth}
		\begin{flushleft}
			\large
			\textit{Author}\\
			Theodore \textsc{Koss} % Your name
		\end{flushleft}
	\end{minipage}
	~
	\begin{minipage}{0.4\textwidth}
		\begin{flushright}
			\large
			\textit{Supervisor}\\
			Dr. Burns \textsc{Healy} % Supervisor's name
		\end{flushright}
	\end{minipage}
	
	% If you don't want a supervisor, uncomment the two lines below and comment the code above
	%{\large\textit{Author}}\\
	%John \textsc{Smith} % Your name
	
	%------------------------------------------------
	%	Date
	%------------------------------------------------
	
	\vfill\vfill\vfill % Position the date 3/4 down the remaining page
	
	{\large\today} % Date, change the \today to a set date if you want to be precise
	
	%------------------------------------------------
	%	Logo
	%------------------------------------------------
	
	%\vfill\vfill
	%\includegraphics[width=0.2\textwidth]{placeholder.jpg}\\[1cm] % Include a department/university logo - this will require the graphicx package
	 
	%----------------------------------------------------------------------------------------
	
	\vfill % Push the date up 1/4 of the remaining page
	
\end{titlepage}
\section{Problem 1}
If $a,b$ are coprime and $b,c$ are coprime, then must $a,c$ be coprime? If so, prove. If not, provide a counterexample. What can you conclude about whether ``is coprime to'' is an equivalence relation among positive integers. If it is, prove it; if not, declare which axiom it fails.\begin{enumerate}[label=(\alph*)]
    \item Consider $a=2,b=5,c=6$. Clearly $\gcd(2,5)=1$ and $\gcd(5,6)=1$, however, $\gcd(2,6)=2$. Thus, $a,c$ are not necessarily coprime when $a,b$ and $b,c$ are.
    \item We can use the above result to prove ``is coprime to" is not an equivalence relation over $\Z^+$, because it fails the axiom of transitivity.
\end{enumerate}
\section{Problem 2}
Let $\Z_k\backslash \{[0]_k\}$ be the set $\Z_k$ without the zero element. What condition on the integer $k$ makes ($\Z_k\backslash \{[0]_k\},\cdot$) a group? Prove this condition is both sufficient and necessary.\begin{proof}$k$ must be a prime number.\begin{itemize}
    \item Necessity: $A\implies B$. \newline Assume $k$ is prime, then the set $\Z_k\backslash \{[0]_k\}$, with the binary operation $\cdot$, has:\begin{enumerate}
        \item Closure: As $\forall a,b\in\Z_k\backslash \{[0]_k\}$, with prime $k$, it is impossible to multiply two elements to be equivalent to $[0]_k$.
        \item Identity: The element $[1]_k=e$. Of course.
        \item Inverses: $\forall a\in\Z_k\backslash \{[0]_k\}$, by \href{https://en.wikipedia.org/wiki/B%C3%A9zout%27s_identity}{\color{cyan}Bezout's Identity} since $\gcd(a,k)=1$, $\exists x,y\in\Z$ such that $ax+ky=1$, reduce modulo $k$ to achieve: $ax=1$. Thus $x\in\Z_k\backslash \{[0]_k\}$ is the inverse of $a$.
        \item Associativity: Since multiplication over the integers mod $k$ is well defined, $\cdot$ is associative.
        \end{enumerate}
        Therefore ($\Z_k\backslash \{[0]_k\},\cdot$) is a group.
    \item Sufficiency: $B\implies A$, or $\neg A\implies \neg B$. \newline To the contrary, assume $k$ is composite. Then $k=pq$, for some $p,q\neq1\in\Z_k\backslash \{[0]_k\}$. This shows that there exists some $a,b\in\Z_k\backslash \{[0]_k\}$ such that $ab=k\equiv[0]_k\notin\Z_k\backslash \{[0]_k\}$. Thus $(\Z_k\backslash \{[0]_k\},\cdot)$ is not closed, and is therefore not a group. As required.
\end{itemize}
\end{proof}
\section{Problem 3}
Prove that, for an arbitrary integer $n\geq2$, any integer $M$ can be written as $m=an+r$, where $a\in\Z$ and $2n\leq r <3n$.\begin{proof}To show existence, we consider some set $$S=\{m-an=r|a\in\Z\},m-an\geq0$$ If we can prove this set is nonempty, by the well ordering principle, there will be a least element. There are two cases for $r$.\begin{enumerate}[label=(\roman*)]
    \item $m\geq0$, in this case, we set $a=0$ and achieve the following: $r=m-0n=m\in S$.
    \item $m<0$, then we can set $a=m$. Then $r=m-an=m-mn=m(1-n)$. And since $m<0$ and $n\geq2$, $a(1-d)$ is, of course, an element of $S$.
\end{enumerate}Thus $S$ is nonempty, and therefore has a least element $r=m-an$, rearranging this we get our original equation must be true: $m=an+r$.
\newline However this does not show uniqueness of $a$ and $r$.
To prove this, consider some elements $b,s$ (haha, get it?) satisfy $m=bn+s$. Then, we may assume $s\geq r$, and thus, $0\leq r-s<n$. Since $m=bn+s=an+r$, the following holds:$$r-s=n(b-a)$$Which, by definition, means $n$ divides $r-s$, which implies either $r-s\geq d$ or $r-s=0$. But, since we know $0\leq r-s<n$, $r-s=0$ and therefore $r=s$. This, of course, implies $b=a$, therefore $r$ and $a$ are unique.
\end{proof}
\section{Problem 4}
Let $k,n$ be arbitrary positive integers. Find a matrix $M_k$ that has order $k$ as an element of the group $GL_n(\C)$.\begin{proof}$GL_n(\C)=\{A=[a_{ij}]_{n\times n}\}$, such that $|A|\neq0$ and $a_{ij}\in\C$. This is the group of matrices of $n\times n$ order, with nonzero determinants.\newline Thus, upper or lower triangular matrix will have determinant $|A|=\underbrace{a_{11}\cdot a_{22}\cdot...\cdot a_{nn}}_{\text{Product of diagonal elements.}}$, which will of course be nonzero, since all of $a_{11}\cdot a_{22}\cdot...\cdot a_{nn}$ are nonzero.
\end{proof}
\section{Problem 5}Let $P,Q$ be regular polygons. Let $G_P, G_Q$ be the group of rigid motions of $P,Q$, respectively. Show that if there is an isomorphism between $G_P,G_Q$, then the polygons are similar. Do they also have to be congruent?\begin{proof}Consider the regular polygons $P,Q$, and consider the polygon to be a $n\text{-gon}$, and $m\text{-gon}$ respectively. Then $G_P=D_n$, and $G_Q=D_m$. Therefore the order of the rigid motion groups $G_P, G_Q$ are $2n$, and $2m$, respectively (This is proven in problem \color{magenta}{\ref{7}}\color{black}). Since $|G_P|, |G_Q|\neq\infty$, in order for there to be an isomorphism $\phi:G_P\to G_Q$, $|G_P|=|G_Q|$. Therefore, $2m=2n$, this necessarily implies $m=n$. And by \href{https://en.wikipedia.org/wiki/Regular_polygon#Regular_convex_polygons}{\color{cyan}this result} in geometry, all regular simple polygons with the same number of sides are similar. Thus, $P$ must be similar to $Q$ if there exists an isomorphism between their respective groups. 
However, this result has nothing to do with the congruence of these two polygons. \\

Counterexample: consider $P$ to be some regular 4-gon with area $4$, whose rigid motion group is isomorphic to some other rigid motion group of a polygon, $Q$, with area $25$. Then, by our result above, $Q$ must also be a square (regular 4-gon). These two rigid motion groups are isomorphic, however clearly, $4\neq25$ and therefore they are not congruent.\\

So if there exists an isomorphism between rigid motion groups $G_P$ and $G_Q$ for regular polygons $P$ and $Q$, then $P$ and $Q$ must be similar, however they do not necessarily have to be congruent.
\end{proof}
\section{Problem 6} 
Define $\C_r$ to be the set \{$a+rbi|a,b\in\R$\} for each $r\in\R$.\begin{itemize}
    \item Prove that this set is a group under addition.\begin{proof}To be a group, the following must be true:\begin{enumerate}[label=(\alph*)]
        \item Closure: This holds because $\forall n,m\in\C_r$, where $n=a_n+rb_ni$, and $m=a_m+rb_mi$, their sum, $n+m=\underbrace{a_n+a_m}_{\text{Real part}}+\underbrace{ri(b_n+b_m)}_{\text{Imaginary part}}\in\C_r$.
        \item Existence of identity: The identity element is $0$, $n+0=n$.
        \item Existence of inverses: For each element $n=a_n+rb_ni$, the inverse is $n^{-1}=-a_n-rb_ni\in\C_r$. Therefore every element has inverses.
        \item Associativity of addition, this holds because $\forall n,m,l\in\C_r$, $n+(m+l)=(n+m)+l$.
    \end{enumerate}Thus, $(\C_r,+)$ is a group.
    \end{proof}
    \item For what values of $r$ are the groups $(\C,+)$ and $(\C_r,+)$ isomorphic?\begin{proof}All values $r\in\R$. This is the case because we can set up a bijective homomorphism $\phi:(\C_r,+)\to (\C,+)$ defined by $\phi(n\in\C_r)=m\in\C$. $\forall n=a_n+rb_ni\in\C_r$, the homomorphism $\phi:(\C_r,+)\to (\C,+)$ is rather trivial, as to define a unique element $m=a_m+b_mi\in\C$, simply let $a_m=a_n$ and $b_m=rb_n$. To show $\phi$ is a bijection, we N2S the following:\begin{enumerate}[label=(\alph*)]
        \item $\phi$ is injective. This is the case because if we choose some value $k$ for which $\phi(k)=\phi(n)=m$, this means $k=a_k+rb_ki=m\in\C$, but so does $n=a_n+rb_ki$. Which implies that $a_m=a_k=a_n$, and $b_m=rb_k=rb_n$ This shows if there did exist some $k,n$ for which $\phi(k)=\phi(n)$, it would imply that $k=n$.
        \item $\phi$ is surjective. This is also true because $\forall m=a_m+b_mi\in\C$, $\exists n\in\C_r$ s.t. $\phi(n)=m$, specifically defined by $n=a_n+rb_ni$ where $a_m=a_n$ and $b_m=rb_n$.
    \end{enumerate}
    \end{proof}
\end{itemize}
\section{Problem 7}\label{7}
Define the order of a group. Let $D_n$ be the dihedral group and let $Sym_n$ be the symmetric group on $n$ letters. State and prove a relationship between $|D_n|$ and $|Sym_n|$.\newline The order of a group is the cardinality (or ``size") of the group. The relationship between the orders of $D_n$ and $Sym_n$ is \scalebox{1.5}{$\frac{|Sym_n|}{|D_n|}=\frac{n!}{2n}$}.\begin{proof}For $Sym_n$, the permutation group is a bijection from a set of $n$ elements to itself. Therefore, if you choose some $a\in Sym_n$, it has $n$ choices to be sent to, then the next element $b\in Sym_n$ has $n-1$ choices to be sent to. Continue this for all elements of $Sym_n$, and the result is $|Sym_n|=n!$.\newline For $D_n$, this is the group of symmetries of a regular $n\text{-gon}$. WLOG, consider the example $n=3$. This is the group of symmetries of an equilateral triangle. By inspection, it is easy to see that a rotation by $\frac{360}{3}^{\circ}$ is a symmetry, in fact, a symmetry for each rotation up to $360^{\circ}=e$, in this case 3. We can generalize this to any $n\text{-gon}$ to get the first $n$ symmetries. The next $n$ symmetries come from drawing a line through one of the $n$ vertices, then reflecting the shape over this line. Do this for each vertex to get $n$ more symmetries. The final result is $n+n=2n$ symmetries of a regular $n\text{-gon}$, and therefore $|D_n|=2n$.
\end{proof}
\section{Problem 8}
Prove that differentiable, bijective functions from $\R\to\R$ form a group under composition.\begin{proof}
Let $G=\{\text{Bijections } \phi:\R\to\R\}$. N2S: $(G,\circ)$ is a group, where $\circ$ denotes function composition.\begin{enumerate}
    \item Closure: A bijection composed with a bijection is necessarily another bijection, therefore $G$ is closed under composition.
    \item Identity: The element $e=\phi$ where $\phi(x)=x$ is the identity function, and $e\in G$.
    \item Inverses: For each element $\phi\in G$, $\phi$ defines some bijection from $\R\to\R$, then there must exist another bijection $\theta$, where $\theta$ defines a bijection from $\R\to\R$. WLOG, as an example, consider the finite sets $X=\{1,2,3\}$ and $Y=\{-1,-2,-3\}$. Of course, a bijection $\phi$ exists, namely $\phi:X\to Y$ defined by $\phi(x)=-x$, $\forall x\in X$. There also exists a bijection $\theta:Y\to X$, defined by $\theta(y)=-y$, $\forall y\in Y$. This $\theta$ is the inverse of $\phi$, this also means $\phi \circ \theta=e$. We can see this because if we do the bijection $\phi$, it's the mapping $1\to -1,2\to -2, 3\to -3$, then $\theta$ is the mapping $-1\to 1,-2\to 2, -3\to 3$. Therefore composing the two is the same as doing nothing. This case can be generalized to the infinite set $\R$.
    \item Associativity: Function composition is associative.
\end{enumerate}
\end{proof}
\end{document}