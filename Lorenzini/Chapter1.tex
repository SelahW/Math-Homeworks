\documentclass[hidelinks,12pt]{article}
\usepackage[utf8]{inputenc}
\usepackage{mathtools}
\usepackage{amsthm}
\usepackage{amsmath}
\usepackage{amsfonts}
\usepackage{amssymb}
\usepackage{centernot}
\usepackage{marvosym}
\usepackage{enumitem}
\usepackage{hyperref}
\usepackage{graphicx}
\graphicspath{{/home/theo/Documents/GitHub/Math-Homeworks/Images/}}
\setcounter{tocdepth}{1}
\let\marvosymLightning\Lightning
\renewcommand{\geq}{\geqslant}
\renewcommand{\leq}{\leqslant}
\newtheorem{theorem}{Theorem}
\newtheorem{corollary}{Corollary}[theorem]
\newtheorem*{remark}{Remark}
\newcommand{\R}{\mathbb{R}}
\newcommand{\N}{\mathbb{N}}
\newcommand{\Z}{\mathbb{Z}}
\newcommand{\Q}{\mathbb{Q}}
\newcommand{\F}{\mathbb{F}}
\newcommand{\gothp}{\mathfrak{P}}
\newcommand{\gothq}{\mathfrak{Q}}
\newcommand{\divby}{%
  \mathrel{\text{\vbox{\baselineskip.65ex\lineskiplimit0pt\hbox{.}\hbox{.}\hbox{.}}}}%
  }
\newcommand{\notdivby}{\centernot\divby}
\title{\scalebox{1.45}{Arithmetic Geometry Problems}}
\author{\scalebox{1.5}{Theo Koss}}
\date{October 2024}

\begin{document}

\maketitle
\section{Chapter 1}
\begin{enumerate}
    \item Let \(d\in\Q\setminus\Z\), prove \(\Z[\sqrt{d}]\) is not a finitely generated abelian group.
        \begin{proof}
            Let \(d=\frac{p}{q}\) with \(p\neq q\in\Z,\ q\neq0,1\)  and \(\gcd(p,q)=1\). Note that subgroups of finitely generated \emph{abelian} groups are themselves finitely generated. So consider \(\Z[d]<\Z[\sqrt{d}]\). Assume BWOC that \(\Z[d]\) is finitely generated, say \(n\) generators. Then we can write any element of \(\Z[d]\) as a linear combination of these elements. Consider 
            \begin{align*}
                \frac{1}{q^{n+1}}&=b_{0}+b_{1}d+b_{2}d^{2}+\dots+b_{n}d^{n}\tag{For integers \(b_{i}\) }\\
                                 &=b_{0}+b_{1}\frac{p}{q}+\dots+b_{n}\frac{p^{n}}{q^{n}}\\
                \implies 1&=b_{0}\cdot q^{n+1}+b_{1}p\cdot q^{n}+\dots+b_{n}p^{n}\cdot q\\
                          &=q\underbrace{(b_{0}\cdot q^{n}+b_{1}p\cdot q^{n-1}+\dots+b_{n}p^{n})}_{\in\Z}\\
                \implies \frac{1}{q}&\in\Z\\
            \end{align*}
            Contradiction because we have \(q\neq1\).
        \end{proof}
        \begin{proof}[Alternate Proof:]
            Due to a theorem (not in the book :/ ), the ring \(\Z[x]\) is finitely generated iff \(x\) is algebraic over \(\Z\). We have \[
                m_{\sqrt{d},\Z}(x)=x^{2}-d=qx^{2}-p
            \]
            Which is not monic in \(\Z\) because we have \(q\neq1\) and \(q\) does not divide \(p\).
        \end{proof}
    \item Prove \(\Z[\frac{2+i}{5}]\cap\Q=\Z\) and \(\Z[\frac{2-i}{5}]\cap\Q=\Z\).
        \begin{proof}
            Assume, BWOC, that we have some element \(k\in\Z[\frac{2+i}{5}]\) such that \(k\in\Q\setminus\Z\). Then \(k=\frac{p}{q}\) with \(p,q\in\Z\), \(q\neq0,1\) and \(\gcd(p,q)=1\). We also have
            \[
                k=a+b\cdot \frac{2+i}{5}=a+\frac{2b}{5}+\frac{bi}{5}
            \]
            for some \(a,b\in\Z\). Since \(k=\frac{p}{q}\) is strictly real, we must have
            \[
                \frac{bi}{5}=0\implies b=0
            \]
             But then \(k=a+0\in\Z\) contradiction.\\
            Similarly, write \(k=a+b\cdot \frac{2-i}{5}=a+\frac{2b}{5}-\frac{bi}{5}\) so \(\frac{bi}{5}=0\implies b=0\) so \(k\in\Z\).
        \end{proof}
    \item Let \(A\) be a ring, and let \(I,J\) be two coprime ideals of \(A\). Show that, \(\forall a,b\in\N,\ I^{a}\) is coprime to \(J^{b}\).
        \begin{proof}
            Since \(I\) and \(J\) are coprime, by definition we have \(I+J=A\). Base case: \(I^{1}+J^{1}=A\) obviously. Fix some \(b\in\N\), assume \(I^{k}\) is coprime to \(J^{n}\), for some \(a\in\N\). Then
            \[
                I^{a}+J^{b}=A
            \]
            Multiply both sides by \(I\) (on the left), \[
            I^{a+1}+J^{b}=IA=A
            \]
            Thus \(I^{a+1}\) is coprime to \(J^{b}\). Therefore the statement is true for all pairs \(a,b\in\N\).
        \end{proof}
    \item Show that in the ring \(\Z[\sqrt{-5}]\), the elements \(2,3,1+\sqrt{-5},1-\sqrt{-5}\) are irreducible, and that they are not associates.
    \item Let \(p\) be a prime number. Let \(\bar{g}(x)\in(\Z/p\Z)[x]\) be any irreducible polynomial. Let \(g(x)\in\Z[x]\) be such that its image under the natural reduction map \(\Z[x]\to(\Z/p\Z)[x]\) is \(\bar{g}(x)\). Show that the ideal \((p,g(x))\) is a maximal ideal of \(\Z[x]\).
        \begin{proof}
            We have that \[
                \Z[x]/(p,g(x))\cong(\Z[x]/p)/(g(x))\cong(\Z/p\Z)[x]/(g(x))
            \]
            Then, consider the natural reduction map
            \[
                \pi:\Z[x]\to(\Z/p\Z)[x]
            \]
            And we have \(\pi(g(x))=\bar{g}(x)\), so \((\Z/p\Z)[x]/(g(x))\cong(\Z/p\Z)[x]/(\bar{g}(x))\). We know \(\bar{g}(x)\) is irreducible, and the ring \((\Z/p\Z)[x]\) is a PID, so the ideal \((\bar{g}(x))\) is maximal, and therefore \((\Z/p\Z)[x]/(\bar{g}(x))\)is a field, (finite field \(\F_{p^{\deg(\bar{g}(x))}}\)). So the ideal \((p,g(x))\) is maximal in \(\Z[x]\)\\
            (Because modding by it gave a field.)
        \end{proof}
    \item Show that a prinicpal ideal domain has the property of unique factorization of ideals.
        \begin{proof}
        Let \(A\) be a PID, then obviously \(A\) is Noetherian. Let \((g)\) be a nontrivial ideal of \(A\). If \((g)\) is itself prime, we have \((g)=\gothp\) is a unique factorization.
        \end{proof}
\end{enumerate}
\end{document}
