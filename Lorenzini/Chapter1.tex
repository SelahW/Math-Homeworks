\documentclass[hidelinks,12pt]{article}
\usepackage[utf8]{inputenc}
\usepackage{mathtools}
\usepackage{amsthm}
\usepackage{amsmath}
\usepackage{amsfonts}
\usepackage{amssymb}
\usepackage{centernot}
\usepackage{marvosym}
\usepackage{enumitem}
\usepackage{hyperref}
\usepackage{graphicx}
\graphicspath{{/home/theo/Documents/GitHub/Math-Homeworks/Images/}}
\setcounter{tocdepth}{1}
\let\marvosymLightning\Lightning
\renewcommand{\geq}{\geqslant}
\renewcommand{\leq}{\leqslant}
\newtheorem{theorem}{Theorem}
\newtheorem{corollary}{Corollary}[theorem]
\newtheorem*{remark}{Remark}
\newcommand{\R}{\mathbb{R}}
\newcommand{\N}{\mathbb{N}}
\newcommand{\Z}{\mathbb{Z}}
\newcommand{\Q}{\mathbb{Q}}
\newcommand{\F}{\mathbb{F}}
\renewcommand{\O}{\mathcal{O}}
\newcommand{\gothp}{\mathfrak{P}}
\newcommand{\gothq}{\mathfrak{Q}}
\newcommand{\divby}{%
  \mathrel{\text{\vbox{\baselineskip.65ex\lineskiplimit0pt\hbox{.}\hbox{.}\hbox{.}}}}%
  }
\newcommand{\notdivby}{\centernot\divby}
\title{\scalebox{1.45}{Arithmetic Geometry Problems}}
\author{\scalebox{1.5}{Theo Koss}}
\date{October 2024}

\begin{document}

\maketitle
\section{Chapter 1}
\begin{enumerate}
    \item Let \(d\in\Q\setminus\Z\), prove \(\Z[\sqrt{d}]\) is not a finitely generated abelian group.
        \begin{proof}
            Let \(d=\frac{p}{q}\) with \(p\neq q\in\Z,\ q\neq0,1\)  and \(\gcd(p,q)=1\). Note that subgroups of finitely generated \emph{abelian} groups are themselves finitely generated. So consider \(\Z[d]<\Z[\sqrt{d}]\). Assume BWOC that \(\Z[d]\) is finitely generated, say \(n\) generators. Then we can write any element of \(\Z[d]\) as a linear combination of these elements. Consider 
            \begin{align*}
                \frac{1}{q^{n+1}}&=b_{0}+b_{1}d+b_{2}d^{2}+\dots+b_{n}d^{n}\tag{For integers \(b_{i}\) }\\
                                 &=b_{0}+b_{1}\frac{p}{q}+\dots+b_{n}\frac{p^{n}}{q^{n}}\\
                \implies 1&=b_{0}\cdot q^{n+1}+b_{1}p\cdot q^{n}+\dots+b_{n}p^{n}\cdot q\\
                          &=q\underbrace{(b_{0}\cdot q^{n}+b_{1}p\cdot q^{n-1}+\dots+b_{n}p^{n})}_{\in\Z}\\
                \implies \frac{1}{q}&\in\Z\\
            \end{align*}
            Contradiction because we have \(q\neq1\).
        \end{proof}
        \begin{proof}[Alternate Proof:]
            Due to a theorem (not in the book :/ ), the ring \(\Z[x]\) is finitely generated iff \(x\) is algebraic over \(\Z\). We have \[
                m_{\sqrt{d},\Z}(x)=x^{2}-d=qx^{2}-p
            \]
            Which is not monic in \(\Z\) because we have \(q\neq1\) and \(q\) does not divide \(p\).
        \end{proof}
    \item Prove \(\Z[\frac{2+i}{5}]\cap\Q=\Z\) and \(\Z[\frac{2-i}{5}]\cap\Q=\Z\).
        \begin{proof}
            Assume, BWOC, that we have some element \(k\in\Z[\frac{2+i}{5}]\) such that \(k\in\Q\setminus\Z\). Then \(k=\frac{p}{q}\) with \(p,q\in\Z\), \(q\neq0,1\) and \(\gcd(p,q)=1\). We also have
            \[
                k=a+b\cdot \frac{2+i}{5}=a+\frac{2b}{5}+\frac{bi}{5}
            \]
            for some \(a,b\in\Z\). Since \(k=\frac{p}{q}\) is strictly real, we must have
            \[
                \frac{bi}{5}=0\implies b=0
            \]
             But then \(k=a+0\in\Z\) contradiction.\\
            Similarly, write \(k=a+b\cdot \frac{2-i}{5}=a+\frac{2b}{5}-\frac{bi}{5}\) so \(\frac{bi}{5}=0\implies b=0\) so \(k\in\Z\).
        \end{proof}
    \item Let \(A\) be a ring, and let \(I,J\) be two coprime ideals of \(A\). Show that, \(\forall a,b\in\N,\ I^{a}\) is coprime to \(J^{b}\).
        \begin{proof}
            Since \(I\) and \(J\) are coprime, by definition we have \(I+J=A\). Base case: \(I^{1}+J^{1}=A\) obviously. Fix some \(b\in\N\), assume \(I^{k}\) is coprime to \(J^{n}\), for some \(a\in\N\). Then
            \[
                I^{a}+J^{b}=A
            \]
            Multiply both sides by \(I\) (on the left), \[
            I^{a+1}+J^{b}=IA=A
            \]
            Thus \(I^{a+1}\) is coprime to \(J^{b}\). Therefore the statement is true for all pairs \(a,b\in\N\).
        \end{proof}
    \item Show that in the ring \(\Z[\sqrt{-5}]\), the elements \(2,3,1+\sqrt{-5},1-\sqrt{-5}\) are irreducible, and that they are not associates.
    \item Let \(p\) be a prime number. Let \(\bar{g}(x)\in(\Z/p\Z)[x]\) be any irreducible polynomial. Let \(g(x)\in\Z[x]\) be such that its image under the natural reduction map \(\Z[x]\to(\Z/p\Z)[x]\) is \(\bar{g}(x)\). Show that the ideal \((p,g(x))\) is a maximal ideal of \(\Z[x]\).
        \begin{proof}
            We have that \[
                \Z[x]/(p,g(x))\cong(\Z[x]/p)/(g(x))\cong(\Z/p\Z)[x]/(g(x))
            \]
            Then, consider the natural reduction map
            \[
                \pi:\Z[x]\to(\Z/p\Z)[x]
            \]
            And we have \(\pi(g(x))=\bar{g}(x)\), so \((\Z/p\Z)[x]/(g(x))\cong(\Z/p\Z)[x]/(\bar{g}(x))\). We know \(\bar{g}(x)\) is irreducible, and the ring \((\Z/p\Z)[x]\) is a PID, so the ideal \((\bar{g}(x))\) is maximal, and therefore \((\Z/p\Z)[x]/(\bar{g}(x))\)is a field, (finite field \(\F_{p^{\deg(\bar{g}(x))}}\)). So the ideal \((p,g(x))\) is maximal in \(\Z[x]\)
        \end{proof}
    \item Show that a prinicpal ideal domain has the property of unique factorization of ideals.
        \begin{proof}
        Let \(A\) be a PID, then it is also a UFD. Consider an arbitrary ideal \(I=(a)\subset A\), then, by UFD, \(a\) can be written uniquely as a product of irreducibles, \(a=p_{1}\cdots p_{n}\). But, since every ideal is principal, and every element is contained in the ideal generated by it, we have \[
            I=(a)=(p_{1})\cdots(p_{n})
        \]
        And in a PID, ideals generated by irreducibles are maximal, and maximal = prime. So we have a unique factorization of the ideal \(I\) into prime ideals. 
        \end{proof}
    \item Let \(A\) be a commutative ring and \(I\subset A\) be an ideal.
        \begin{enumerate}[label=(\alph*).]
            \item Let \(a_{1},\dots,a_{s}\in A\) and let \(J\) denote the ideal of \(A/I\) generated by the images of \(a_{1},\dots,a_{s}\) under the map \(A\to A/I\). Show that \[
                    (A/I)/J\overset{\sim}{\to} A/(I,a_{1},\dots,a_{s})
            \]
            \begin{proof}
           We have the natural homomorphism \(\pi:A\to A/I\), and we have another homomorphism \(\psi:A/I\to (A/I)/J\) which has \(\ker(\psi)=J=(\pi(a_{1}),\dots,\pi(a_{s}))\). COME BACK!
            \end{proof}
            \item Let \(J\) be any ideal of \(A\). Show that \[
                    (A/I)/(J+I/I)\cong(A/J)/(I+J/J)
            \]
           \begin{proof}
           
           \end{proof} 
            \item 
        \end{enumerate}
    \item \begin{enumerate}[label=(\alph*).]
        \item Let \(k\) be any field. Let \(A\coloneqq k[x_{1},\dots,x_{n},\dots]\) be the polynomial ring in countably many variables. Show that \(A\) is not Noetherian.
        \begin{proof}
            By way of contradiction, assume \(A\) is Noetherian, so every increasing chain of ideals \[
            I_{0}\subset I_{1}\subset I_{2}\subset\dots\subset I_{n}=I_{n+1}
            \]
            Stabilizes at some point. We have the ideals
            \[
                (x_{1})\subset(x_{1},x_{2})\subset\dots(x_{1},\dots,x_{n})=(x_{1},\dots,x_{n},x_{n+1})
            \]
            That gives \(x_{n+1}\in I_{n}=(x_{1},\dots,x_{n})\). So we can write \(x_{n+1}\) as a linear combination of the elements of that ideal,
            \[
                x_{n+1}=\sum_{i=1}^{n} c_{i}x_{i}
            \]
            But, consider the evaluation mapping \(\phi:A\to k\) by evaluating \(x_{1},\dots,x_{n}=0\) and \(x_{n+1}=1\). Applying this evaluation mapping to above gives \[
            1=\phi(x_{n+1})=\phi(\sum_{i=1}^{n} c_{i}x_{i})=\sum c_{i}\phi(x_{i})=\sum c_{i}\cdot0=0
            \]
           Contradiction. 
        \end{proof}
    \item Let \(\bar{\Q}\) denote an algebraic closure of \(\Q\). Let \(\O\) denote the integral closure of \(\Z\) in \(\bar{\Q}\). Show that \(\O\) is not a Noetherian ring. (Hint: find a nonstationary sequence of ideals in \(\O\) by taking successive roots of an integer.)
        \begin{proof}
            Let \(\O\) be the integral closure of \(\Z\) in \(\bar{\Q}\). Then in particular, \(\Z\subset\O\), so consider \(2\in\O\). Let \(I=(2)\). Then consider the increasing chain of ideals:
            \[
                I=(2)\subset(\sqrt{2})\subset(\sqrt[3]{2})\subset\dots\subset\sqrt{I}
            \]
            Each ideal strictly contains the next, so this is a nonstationary increasing chain of ideals. (Equivalently the radical is not finitely generated)
        \end{proof}
    \end{enumerate}
\item Let \(k\) be a field, and \(A\coloneqq k[x_{1},\dots,x_{n}]\). Let \(\bar{k}\) denote an algebraic closure of \(k\), and let \(B\coloneqq \bar{k}[x_{1},\dots,x_{n}]\). Show that the extension \(B/A\) is integral. Note that in general, \(B\) is not a finitely generated \(A\)-module.
    \begin{proof}
        \(B/A\) is integral iff every element of \(B\) is integral over \(A\) (is the root of a monic polynomial in \(A[y]\)). So, choose an arbitrary element \(\alpha\in B\). There are two cases:
        \begin{enumerate}
            \item If \(\alpha\in A\), then we are done because \(y-\alpha\) is a polynomial in \(A[y]\) that is satisfied by \(\alpha\).
            \item So, assume \(\alpha\in B-A\). Then, since \(\bar{k}\) is an algebraic closure of \(k\), for all \(\beta\in\bar{k}\), we have \(f(\beta)=0\) where \(f\) is a monic polynomial with coefficients in \(k\). So, let \(\alpha\in B\), then \[
                    \alpha=\sum_{i=1}^{n} c_{i}x_{i}^{a_{i}}
            \]
            Where \(c_{i}\in\bar{k}\). Since \(\bar{k}\) is an algebraic closure, for each coefficient \(c_{i}\), there exists a monic polynomial \(p_{i}\in A\) such that \(p_{i}(c_{i})=0\). Then, the product of all of these \(p_{i}\) kills each coefficient, so let \[
            A\ni P(x_{1},\dots,x_{n})=\prod_{i=1}^{n}p_{i}(x_{i})
            \]
            Then
             \[
                 P\left(\alpha\right)=P\left(\sum_{i=1}^{n}c_{i}x_{i}^{a_{i}}\right)=\prod_{i=1}^{n}p_{i}\left(\sum_{i=1}^{n}c_{i}x_{i}^{a_{i}}\right)=0
            \]
           And, a product of monic polynomials is monic, so \(P(x_{1},\dots,x_{n})\) is a monic polynomial in \(A\) which is satisfied by \(\alpha\) as required.
        \end{enumerate}
    \end{proof}
\item Let \(B/A\) be an integral extension. Show that \(B\) is a field iff \(A\) is a field.
    \begin{proof}
        \((\implies)\) Let \(B/A\) be an integral extension and \(B\) be a field. Then we have \(\forall \beta\in B\), \(\exists f_{\beta}\in A\) with \(f_{\beta}(\beta)=0\) and \(f_{\beta}\) is monic. We also have that \(\forall \beta\in B,\ \beta^{-1}\in B\). So the minimal polynomial of both \(\beta\) and \(\beta^{-1}\) are in the ring \(A[y]\). COME BACK!
    \end{proof}
\end{enumerate}
\end{document}
