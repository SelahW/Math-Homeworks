\documentclass[hidelinks,12pt]{article}
\usepackage[utf8]{inputenc}
\usepackage{mathtools}
\usepackage{amsthm}
\usepackage{amsmath}
\usepackage{amsfonts}
\usepackage{amssymb}
\usepackage{centernot}
\usepackage{marvosym}
\usepackage{enumitem}
\usepackage{hyperref}
\usepackage{graphicx}
\usepackage{tikz-cd}
\graphicspath{{/home/theo/Documents/GitHub/Math-Homeworks/Images/}}
\setcounter{tocdepth}{1}
\let\marvosymLightning\Lightning
\renewcommand{\geq}{\geqslant}
\renewcommand{\leq}{\leqslant}
\newtheorem{theorem}{Theorem}
\newtheorem{corollary}{Corollary}[theorem]
\newtheorem*{remark}{Remark}
\newcommand{\R}{\mathbb{R}}
\newcommand{\N}{\mathbb{N}}
\newcommand{\Z}{\mathbb{Z}}
\newcommand{\Q}{\mathbb{Q}}
\newcommand{\F}{\mathbb{F}}
\renewcommand{\O}{\mathcal{O}}
\newcommand{\gothp}{\mathfrak{P}}
\newcommand{\gothq}{\mathfrak{Q}}
\newcommand{\rk}{\operatorname{rank}}
\title{\scalebox{1.45}{Arithmetic Geometry Problems}}
\author{\scalebox{1.5}{Theo Koss}}
\date{October 2024}

\begin{document}

\maketitle
\section{Chapter 1}
\begin{enumerate}
    \item Let \(d\in\Q\setminus\Z\), prove \(\Z[\sqrt{d}]\) is not a finitely generated abelian group.
        \begin{proof}
            Let \(d=\frac{p}{q}\) with \(p\neq q\in\Z,\ q\neq0,1\)  and \(\gcd(p,q)=1\). Note that subgroups of finitely generated \emph{abelian} groups are themselves finitely generated. So consider \(\Z[d]<\Z[\sqrt{d}]\). Assume BWOC that \(\Z[d]\) is finitely generated, say \(n\) generators. Then we can write any element of \(\Z[d]\) as a linear combination of these elements. Consider 
            \begin{align*}
                \frac{1}{q^{n+1}}&=b_{0}+b_{1}d+b_{2}d^{2}+\dots+b_{n}d^{n}\tag{For integers \(b_{i}\) }\\
                                 &=b_{0}+b_{1}\frac{p}{q}+\dots+b_{n}\frac{p^{n}}{q^{n}}\\
                \implies 1&=b_{0}\cdot q^{n+1}+b_{1}p\cdot q^{n}+\dots+b_{n}p^{n}\cdot q\\
                          &=q\underbrace{(b_{0}\cdot q^{n}+b_{1}p\cdot q^{n-1}+\dots+b_{n}p^{n})}_{\in\Z}\\
                \implies \frac{1}{q}&\in\Z\\
            \end{align*}
            Contradiction because we have \(q\neq1\).
        \end{proof}
        \begin{proof}[Alternate Proof:]
            Proposition 2.10 in the book, the \(\Z\)-module \(\Z[x]\) is finitely generated iff \(x\) is algebraic over \(\Z\). We have minimal polynomial: \[
                m_{\sqrt{d},\Z}(x)=x^{2}-d=qx^{2}-p
            \]
            Which is not monic in \(\Z\) because we have \(q\neq1\) and \(q\) does not divide \(p\).
        \end{proof}
    \item Prove \(\Z[\frac{2+i}{5}]\cap\Q=\Z\) and \(\Z[\frac{2-i}{5}]\cap\Q=\Z\).
        \begin{proof}
            Assume, BWOC, that we have some element \(k\in\Z[\frac{2+i}{5}]\) such that \(k\in\Q\setminus\Z\). Then \(k=\frac{p}{q}\) with \(p,q\in\Z\), \(q\neq0,1\) and \(\gcd(p,q)=1\). We also have
            \[
                k=a+b\cdot \frac{2+i}{5}=a+\frac{2b}{5}+\frac{bi}{5}
            \]
            for some \(a,b\in\Z\). Since \(k=\frac{p}{q}\) is strictly real, we must have
            \[
                \frac{bi}{5}=0\implies b=0
            \]
            But then \(k=a+0\in\Z\) contradiction.\\
            Similarly, write \(k=a+b\cdot \frac{2-i}{5}=a+\frac{2b}{5}-\frac{bi}{5}\) so \(\frac{bi}{5}=0\implies b=0\) so \(k\in\Z\).
        \end{proof}
    \item Let \(A\) be a ring, and let \(I,J\) be two coprime ideals of \(A\). Show that, \(\forall a,b\in\N,\ I^{a}\) is coprime to \(J^{b}\).
        \begin{proof}
            Since \(I\) and \(J\) are coprime, by definition we have \(I+J=A\). Base case: \(I^{1}+J^{1}=A\) obviously. Fix some \(b\in\N\), assume \(I^{k}\) is coprime to \(J^{n}\), for some \(a\in\N\). Then
            \[
                I^{a}+J^{b}=A
            \]
            Multiply both sides by \(I\) (on the left), \[
                I^{a+1}+J^{b}=IA=A
            \]
            Thus \(I^{a+1}\) is coprime to \(J^{b}\). Therefore the statement is true for all pairs \(a,b\in\N\).
        \end{proof}
    \item Show that in the ring \(\Z[\sqrt{-5}]\), the elements \(2,3,1+\sqrt{-5},1-\sqrt{-5}\) are irreducible, and that they are not associates.
    \item Let \(p\) be a prime number. Let \(\bar{g}(x)\in(\Z/p\Z)[x]\) be any irreducible polynomial. Let \(g(x)\in\Z[x]\) be such that its image under the natural reduction map \(\Z[x]\to(\Z/p\Z)[x]\) is \(\bar{g}(x)\). Show that the ideal \((p,g(x))\) is a maximal ideal of \(\Z[x]\).
        \begin{proof}
            We have that \[
                \Z[x]/(p,g(x))\cong(\Z[x]/p)/(g(x))\cong(\Z/p\Z)[x]/(g(x))
            \]
            Then, consider the natural reduction map
            \[
                \pi:\Z[x]\to(\Z/p\Z)[x]
            \]
            And we have \(\pi(g(x))=\bar{g}(x)\), so \((\Z/p\Z)[x]/(g(x))\cong(\Z/p\Z)[x]/(\bar{g}(x))\). We know \(\bar{g}(x)\) is irreducible, and the ring \((\Z/p\Z)[x]\) is a PID, so the ideal \((\bar{g}(x))\) is maximal, and therefore \((\Z/p\Z)[x]/(\bar{g}(x))\)is a field, (finite field \(\F_{p^{\deg(\bar{g}(x))}}\)). So the ideal \((p,g(x))\) is maximal in \(\Z[x]\)
        \end{proof}
    \item Show that a prinicpal ideal domain has the property of unique factorization of ideals.
        \begin{proof}
            Let \(A\) be a PID, then it is also a UFD. Consider an arbitrary ideal \(I=(a)\subset A\), then, by UFD, \(a\) can be written uniquely as a product of irreducibles, \(a=p_{1}\cdots p_{n}\). But, since every ideal is principal, and every element is contained in the ideal generated by it, we have \[
                I=(a)=(p_{1})\cdots(p_{n})
            \]
            And in a PID, ideals generated by irreducibles are maximal, and maximal = prime. So we have a unique factorization of the ideal \(I\) into prime ideals. 
        \end{proof}
    \item Let \(A\) be a commutative ring and \(I\subset A\) be an ideal.
        \begin{enumerate}[label=(\alph*).]
            \item Let \(a_{1},\dots,a_{s}\in A\) and let \(J\) denote the ideal of \(A/I\) generated by the images of \(a_{1},\dots,a_{s}\) under the map \(A\to A/I\). Show that \[
                    (A/I)/J\overset{\sim}{\to} A/(I,a_{1},\dots,a_{s})
                \]
                \begin{proof}
                    We have the natural homomorphism \(\pi:A\to A/I\), and we have another homomorphism \(\psi:A/I\to (A/I)/J\) which has \(\ker(\psi)=J=(\pi(a_{1}),\dots,\pi(a_{s}))\). COME BACK!
                \end{proof}
            \item Let \(J\) be any ideal of \(A\). Show that \[
                    (A/I)/(J+I/I)\cong(A/J)/(I+J/J)
                \]
                \begin{proof}

                \end{proof} 
            \item 
        \end{enumerate}
    \item \begin{enumerate}[label=(\alph*).]
            \item Let \(k\) be any field. Let \(A\coloneqq k[x_{1},\dots,x_{n},\dots]\) be the polynomial ring in countably many variables. Show that \(A\) is not Noetherian.
                \begin{proof}
                    By way of contradiction, assume \(A\) is Noetherian, so every increasing chain of ideals \[
                        I_{0}\subset I_{1}\subset I_{2}\subset\dots\subset I_{n}=I_{n+1}
                    \]
                    Stabilizes at some point. We have the ideals
                    \[
                        (x_{1})\subset(x_{1},x_{2})\subset\dots(x_{1},\dots,x_{n})=(x_{1},\dots,x_{n},x_{n+1})
                    \]
                    That gives \(x_{n+1}\in I_{n}=(x_{1},\dots,x_{n})\). So we can write \(x_{n+1}\) as a linear combination of the elements of that ideal,
                    \[
                        x_{n+1}=\sum_{i=1}^{n} c_{i}x_{i}
                    \]
                    But, consider the evaluation mapping \(\phi:A\to k\) by evaluating \(x_{1},\dots,x_{n}=0\) and \(x_{n+1}=1\). Applying this evaluation mapping to above gives \[
                        1=\phi(x_{n+1})=\phi(\sum_{i=1}^{n} c_{i}x_{i})=\sum c_{i}\phi(x_{i})=\sum c_{i}\cdot0=0
                    \]
                    Contradiction. 
                \end{proof}
            \item Let \(\bar{\Q}\) denote an algebraic closure of \(\Q\). Let \(\O\) denote the integral closure of \(\Z\) in \(\bar{\Q}\). Show that \(\O\) is not a Noetherian ring. (Hint: find a nonstationary sequence of ideals in \(\O\) by taking successive roots of an integer.)
                \begin{proof}
                    Let \(\O\) be the integral closure of \(\Z\) in \(\bar{\Q}\). Then in particular, \(\Z\subset\O\), so consider \(2\in\O\). Let \(I=(2)\). Then consider the increasing chain of ideals:
                    \[
                        I=(2)\subset(\sqrt{2})\subset(\sqrt[3]{2})\subset\dots\subset\sqrt{I}
                    \]
                    Each ideal strictly contains the next, so this is a nonstationary increasing chain of ideals. (Equivalently the radical is not finitely generated)
                \end{proof}
        \end{enumerate}
    \item Let \(k\) be a field, and \(A\coloneqq k[x_{1},\dots,x_{n}]\). Let \(\bar{k}\) denote an algebraic closure of \(k\), and let \(B\coloneqq \bar{k}[x_{1},\dots,x_{n}]\). Show that the extension \(B/A\) is integral. Note that in general, \(B\) is not a finitely generated \(A\)-module.
        \begin{proof}
            \(B/A\) is integral iff every element of \(B\) is integral over \(A\) (is the root of a monic polynomial in \(A[y]\)). So, choose an arbitrary element \(\alpha\in B\). There are two cases:
            \begin{enumerate}
                \item If \(\alpha\in A\), then we are done because \(y-\alpha\) is a polynomial in \(A[y]\) that is satisfied by \(\alpha\).
                \item So, assume \(\alpha\in B-A\). Then, since \(\bar{k}\) is an algebraic closure of \(k\), for all \(\beta\in\bar{k}\), we have \(f(\beta)=0\) where \(f\) is a monic polynomial with coefficients in \(k\). So, let \(\alpha\in B\), then \[
                        \alpha=\sum_{i=1}^{n} c_{i}x_{i}^{a_{i}}
                    \]
                    Where \(c_{i}\in\bar{k}\). Since \(\bar{k}\) is an algebraic closure, for each coefficient \(c_{i}\), there exists a monic polynomial \(p_{i}\in A\) such that \(p_{i}(c_{i})=0\). Then, the product of all of these \(p_{i}\) kills each coefficient, so let \[
                        A\ni P(x_{1},\dots,x_{n})=\prod_{i=1}^{n}p_{i}(x_{i})
                    \]
                    Then
                    \[
                        P\left(\alpha\right)=P\left(\sum_{i=1}^{n}c_{i}x_{i}^{a_{i}}\right)=\prod_{i=1}^{n}p_{i}\left(\sum_{i=1}^{n}c_{i}x_{i}^{a_{i}}\right)=0
                    \]
                    And, a product of monic polynomials is monic, so \(P(x_{1},\dots,x_{n})\) is a monic polynomial in \(A\) which is satisfied by \(\alpha\) as required.
            \end{enumerate}
        \end{proof}
    \item Let \(B/A\) be an integral extension. Show that \(B\) is a field iff \(A\) is a field.
        \begin{proof}
            \((\implies)\) Let \(B/A\) be an integral extension and \(B\) be a field. Let \(a\in A-\{0\}\), then by definition of extension, \(a\in B\) and \(a^{-1}\in B\) since \(B\) is a field. Since \(B\) is integral over \(A\), \(\exists g(y)\in A[y]\) with \(g(a^{-1})=0\), and \(g(y)\) monic. Multiply \[g(a^{-1})=(a^{-1})^{n}+a_{n-1}(a^{-1})^{n-1}+\dots+a_{1}(a^{-1})+a_{0}=0\] by \(a^{n}\), so \[1+a_{n-1}a+\dots+a_{1}a^{n-1}+a_{0}a^{n}=0\] which gives \[a_{n-1}a+\dots+a_{1}a^{n-1}+a_{0}a^{n}=-1\] as a polynomial in \(A[a]\). Since we have a linear combination in \(A[a]\) which equals \(-1\), we also have a linear combination in \(A[a]\) which equals \(1\), and so \(a^{-1}\in A\) so \(A\) is a field.
            \par\null\par
            \((\impliedby)\) Let \(B/A\) be an integral extension and \(A\) a field. Then \(B/A\) is a field extension, and so \(B\) is a field.
        \end{proof}
    \item Let \(B/A\) be an integral extension. Let \(M\subset B\) be a prime ideal. Let \(P\coloneqq M\cap A\). Show that \(M\) is maximal in \(B\) iff \(P\) is maximal in \(A\).
        \begin{proof}
            \((\implies)\) Let \(M\) be maximal in \(B\). Then \(B/M\) is a field and \((B/M)/(A/P)\) an integral extension obtained by restricting the extension \(B/A\). So by the problem above, \(A/P\) is a field, so that \(P\subset A\) is a maximal ideal of \(A\).\par\null\par
            \((\impliedby)\) Let \(P=M\cap A\) be a maximal ideal of \(A\). Then \(A/P\) is a field and then by above, \(B/M\) is also a field so \(M\subset B\) is a maximal ideal.
        \end{proof}
    \item[13.] Let \(A\) be a PID with field of fractions \(K\). Let \(L/K\) be an extension of degree 2. Assume that the integral closure \(B\) of \(A\) in \(L\) is a finitely generated \(A\)-module. Show that there exists \(b\in B\) such that \(\{1,b\}\) is a basis for \(B\) over \(A\).
        \begin{proof}
            Consider an element \(z\in B\). We want to show that there exists a \(b\in B\) so that \(z=1a_{0}+ba_{1}\) for some \(a_{i}\in A\). Since \(B\) is the integral closure of \(A\), \(z\) satisfies a monic polynomial \(g(y)\in A[y]\). And since \(L/K\) is a degree 2 extension, \(g\) has degree at most 2. If the degree of \(g\) is 1, then \(z\in A\) and so \(z=1\cdot z\) is its linear combination. So assume \(\deg(g)=2\),
            \[
                g(y)=y^{2}+c_{1}y+c_{0}
            \]
            With \(c_{i}\in A\). By definition, \[
                g(z)=0=z^{2}+c_{1}z+c_{0}
            \]
            Since \(z\in B\subset L\), we have \(z=\frac{r}{s}\), \(r,s\in B\) and \(s\neq 0\). \[
                0=\left(\frac{r}{s}\right)^{2}+\frac{c_{1}r}{s}+c_{0}
            \]
            Multiplying by \(s^{2}\):
            \[
                0=r^{2}+c_{1}rs+c_{0}s^{2}
            \]
            (COME BACK)
        \end{proof}
\end{enumerate}
\section{Chapter 2}
\begin{enumerate}
    \item Let \(A\) be a local ring with maximal ideal \(M\). Let \(m\in M\) and \(a\in A-M\). Show that \(a+m\) is a unit in \(A\).
        \begin{proof}
            Assume \(a+m\) is a non-unit, then by Zorn's lemma, \(a+m\) is contained in a maximal ideal, but there is only one maximal ideal, so \(a+m\in M\implies a\in M\) contradiction, so \(a+m\) must be a unit.
        \end{proof}
    \item Show that a ring \(A\) is a local ring iff the complement in \(A\) of the set of units \(A^{*}\) is an ideal of \(A\). 
        \begin{proof}
            \((\implies)\) Let \(A\) be a local ring, and \(M\) its unique maximal ideal. Then by Zorn's lemma, \(\forall x\in A-A^{*},\ x\in M\) so \(A-A^{*}=M\) is a (maximal) ideal of \(A\).
            \par\null\par
            \((\impliedby)\) Let \(A\) be a ring such that \(M\coloneqq A-A^{*}\) is an ideal of \(A\). Let \(I\) be an ideal, then \(I\) contains no units, otherwise \(I=A\), so then  \(I\subseteq A-A^{*}=M\). Then each ideal \(I\) is contained in \(M\), so \(M\) is the unique maximal ideal of \(A\), which means \(A\) is local.
        \end{proof}
    \item Let \(A\) be a local ring with maximal ideal \(\mathcal{M}\). If \(M\) is any \(A\)-module, let \(\mathcal{M}M\coloneqq\left\{\sum_{i=1}^{n}\mu_{i}m_{i}\mid \mu_{i}\in\mathcal{M},\ m_{i}\in M,\ n\in\N\right\}\). \(\mathcal{M}M\) is an \(A\)-submodule of \(M\). Assume now that \(M\) is a finitely generated \(A\)-module. Show that if \(\mathcal{M}M=M\), then \(M=(0)\). Hint: Let \(\{m_{1},\dots,m_{n}\}\) be a system of generators for \(M\). Express that \(m_{i}\in\mathcal{M}M\) and use problem 1.
        \begin{proof}
            Let \(M\) be a finitely generated \(A\)-module and assume \(\mathcal{M}M=M\), so \begin{align*}
                \mathcal{M}M&\coloneqq\left\{\sum_{i=1}^{n}\mu_{i}m_{i}\mid \mu_{i}\in\mathcal{M},\ m_{i}\in M,\ n\in\N\right\}\\ 
                            &=\left\{\sum_{j=1}^{n}a_{j}m_{j}\mid a_{j}\in A,\ m_{j}\in M,\ n\in N\right\}
            \end{align*}
            Let \(x\in M\), then \(x\) is a linear combination of the basis \(\{m_{1},\dots,m_{n}\}\) with coefficients from \(\mathcal{M}\) and with coefficients from \(A\):
            \begin{align*}
                x&=\mu_{1}m_{1}+\mu_{2}m_{2}+\dots+\mu_{n}m_{n}\\
            x&=a_{1}m_{1}+a_{2}m_{2}+\dots+a_{n}m_{n}\\
           \implies 0&=(a_{1}-\mu_{1})m_{1}+\dots+(a_{n}-\mu_{n})m_{n}\\
            \end{align*}
            Since \(\{m_{1},\dots,m_{n}\}\) is a basis, we have either \((a_{i}-\mu_{i})=0\) or \(m_{i}=0\) for all \(i\in\{1,\dots,n\}\). However, by problem 1, any element of the form \(a+m\) for \(a\in A\) and \(m\in\mathcal{M}\) is a unit, so \(a_{i}-\mu_{i}\neq0\) which means \(m_{i}=0\) for all \(i\), and so \(M=(0)\).
        \end{proof}
    \item Let \(A\) and \(B\) be two local principal ideal domains with the same field of fractions. Show that if \(A\subseteq B\), then \(A=B\). (Maybe have to assume \(B\) is not the field of fractions)
        \begin{proof}
           Let \(A\) be a local PID and \(M\subset A\) its unique maximal (therefore prime) ideal. Let \(K\) be the field of fractions of both \(A\) and \(B\). We have \(A\subseteq B\subseteq K\).\par \textbf{claim:} Every ring between a PID \(A\) and its field of fractions \(K\) is a localization of \(A\). \begin{proof}[Proof of claim]
          See proposition 6.4 (Universal property of rings of fractions)  
            \end{proof}
            So now because \(A\subseteq B\subseteq K\) and \(A\) is a PID, the choices for \(B\) are all the localizations of \(A\). If we choose \(T=A-\{0\}\), then \(T^{-1}A=B=K\) contradiction. So we must choose the only other multiplicative set, \(S=A-M\), then \(B=S^{-1}A\). The units in \(B\) are then just elements of the form \(b=1/s\) so \(s\in B\) is also a unit, but \(s\in A\) is a unit because it is in \(A-M\) (problem 1). So \(A=B\).
        \end{proof}
    \item Let \(A\) be a domain with field of fractions \(K=A_{(0)}\). Let \(M\) be any \(A\)-module. The rank of \(M\) over \(K\), denoted by \(\operatorname{rank}_{A}(M)\), is the dimension of the \(K\)-vector space \(M_{(0)}\).
        \begin{enumerate}[label=(\alph*).]
            \item Show that \(M\) is a torsion \(A\)-module iff \(\rk_{A}(M)=0\).
            \item Let \[
                    \begin{tikzcd}
                        0\arrow[r] & M' \arrow[r] & M\arrow[r] & M''\arrow[r] & 0
                    \end{tikzcd}
                \] be an exact sequence of \(A\)-modules. Show that \(\rk_{A}(M)<\infty\) iff \(\rk_{A}(M')<\infty\) and \(\rk_{A}(M'')<\infty\). Show that if \(\rk_{A}(M)<\infty\), then \(\rk_{A}(M)=\rk_{A}(M')+\rk_{A}(M'')\). In particular, show that \(\rk_{A}(M_{1}\oplus M_{2})=\rk_{A}(M_{1})+\rk_{A}(M_{2})\).
        \end{enumerate}
        \begin{proof}
        \begin{enumerate}[label=(\alph*).]
            \item \((\implies)\) Let \(M\) be a torsion \(A\)-module, so that \(\forall m\in M\), \(\exists a\neq0\in A\) with \(am=0\). Note that \(M_{(0)}\) is the \(K\)-vector space obtained by extending scalars from \(A\) to \(K\), so \(M_{(0)}=K\otimes_{A} M\) then consider \(\mathfrak{m}\in M\)
                \[
                    \mathfrak{m}=k\otimes m=\frac{1}{a}\otimes am=\frac{1}{a}\otimes 0=0
                \]
                So \(M_{(0)}=\{0\}\) and thus is a 0 dimensional \(K\)-vector space.
                \par\null\par
                \((\impliedby)\) Let \(M\) be an \(A\)-module with rank 0. So \[M_{(0)}=\{0\}=K\otimes_{A} M\] Therefore we have some nonzero \(k\in K\) which annihilates \(M\). Let \(k=\frac{a}{b}\) be such that \(\frac{a}{b}\otimes m=0\) for all \(m\in M\). \[
                0=\frac{a}{b}\otimes m=\frac{1}{b}\otimes am
                \]
                Clearly \(\frac{1}{b}\neq0\), so \(am=0\), we also have that \(k=\frac{a}{b}\) was nonzero so that \(a\) is not 0. Therefore we have a nonzero \(a\in A\) such that \(\forall m\in M\), \(am=0\). Thus \(M\) is a torsion \(A\)-module.
            \item Let \[
                    \begin{tikzcd}
                        0\arrow[r] & M' \arrow[r, "f"] & M\arrow[r, "g"] & M''\arrow[r] & 0
                    \end{tikzcd}
            \]
           Be an exact sequence of \(A\)-modules, then (since field of fractions is a flat module) \[
                    \begin{tikzcd}
                        0\arrow[r] & K\otimes_{A} M' \arrow[r, "1\otimes f"] & K\otimes_{A} M\arrow[r, "1\otimes g"] & K\otimes_{A} M''\arrow[r] & 0
                    \end{tikzcd}
           \]
           is exact. Let \(M_{(0)}\coloneqq K\otimes_{A} M\) and \(\bar{h}\coloneqq 1\otimes h\). By the first isomorphism theorem for modules (in this case \(K\)-vector spaces), \(M_{(0)}/\ker(\bar{g})\cong M_{(0)}''\). Since \(\ker(\bar{g})=\operatorname{im}(\bar{f})=M_{(0)}'\) by exactness, \[M_{(0)}/M_{(0)}'\cong M_{(0)}''\] Then \[
           \rk_{A}(M)=\rk_A(M')+\rk_A(M'')
           \]
            \begin{enumerate}[label=(\roman*)]
           \item \((\implies)\) Let \(\rk(M)\) be finite. Then neither \(\rk(M')\) nor \(\rk(M'')\) can be infinite.
           \item \((\impliedby)\) Let \(\rk(M'')\) and \(\rk(M'')\) be finite, then their sum is finite so \(\rk(M)<\infty\).
           \item Let \(M_1\) and \(M_2\) be \(A\)-modules. Then \(M_1\oplus M_2\) is an \(A\)-module and we have a natural short (split) exact sequence
               \[
                    \begin{tikzcd}
                        0\arrow[r] & M_1 \arrow[r, "\imath"] & M_1\oplus M_2\arrow[r, "\phi"] & M_2\arrow[r] & 0
                    \end{tikzcd}
               \]
               Where \(\imath:M_1\hookrightarrow M_1\oplus M_2\) is the inclusion map, and \(\phi:M_1\oplus M_2\to M_2\) is the projection. Clearly \(\operatorname{im}(\imath)=M_1\) as it is an injection, and \(\ker(\phi)=\{(m_1,0)\mid m_1\in M_1\}=M_1\). Then, we can tensor this sequence with the flat module \(K\) to see \[
               \rk(M_1\oplus M_2)=\rk(M_1)+\rk(M_2)
               \]
       \end{enumerate}
        \end{enumerate}
        \end{proof}
    \item Let \(A\) be a commutative domain. Let \(M\) be an \(A\)-module. Let \((0)=M_0\subseteq M_1\subseteq M_2\subseteq\dots\subseteq M_{s-1}\subseteq M_s=M\) be a chain of \(A\)-submodules. Assume that \(\rk(M_i/M_{i-1})\) is finite, \(\forall i=1,2,\dots,s\). Show that \[
            \rk(M)=\sum_{i=1}^s\rk(M_i/M_{i-1})
    \]
   \begin{proof}
       This chain of \(A\)-submodules gives natural injections \(\imath_{n}:M_n\to M_{n+1}\). Each injection defines a short exact sequence:
       \[
           \begin{tikzcd}
               0\arrow[r] & M_{n}\arrow[r, "\imath"] & M_{n+1}\arrow[r] & M_{n+1}/M_n\arrow[r]& 0
   \end{tikzcd}
       \]
       Tensoring each sequence with the field of fractions of \(A\). Gives exact sequences
\[
           \begin{tikzcd}
               0\arrow[r] & \mathcal{M}_{n}\arrow[r, "\imath"] & \mathcal{M}_{n+1}\arrow[r] & \mathcal{M}_{n+1}/\mathcal{M}_n\arrow[r]& 0
   \end{tikzcd}
       \]
       Then since this is an exact sequence of vector spaces, it splits and so for each \(n\), \(\mathcal{M}_{n+1}\cong\mathcal{M}_{n}\oplus\mathcal{M}_{n+1}/\mathcal{M}_{n}\). In particular,
       \begin{align*}
           \mathcal{M}_{s}&\cong\left[\mathcal{M}_{s-1}\oplus\mathcal{M}_{s}/\mathcal{M}_{s-1}\right]\\
           &\cong\left[\left(\mathcal{M}_{s-2}\oplus\mathcal{M}_{s-1}/\mathcal{M}_{s-2}\right)\oplus\mathcal{M}_{s}/\mathcal{M}_{s-1}\right]\\
           &\cong\left[\left(\mathcal{M}_{s-3}\oplus\mathcal{M}_{s-2}/\mathcal{M}_{s-3}\right)\oplus\left(\mathcal{M}_{s-1}/\mathcal{M}_{s-2}\right)\oplus\mathcal{M}_{s}/\mathcal{M}_{s-1}\right]\\
           &\vdots\\
           &\cong\bigoplus_{i=1}^{s}(\mathcal{M}_{i}/\mathcal{M}_{i-1})
       \end{align*}
       Each term has \(\dim_K(\mathcal{M}_i/\mathcal{M}_{i-1})=\rk_A(M_i/M_{i-1})\) so \[\rk_A(M)=\sum_{i=1}^s\rk_A(M_i/M_{i-1})\]
\end{proof} 
\end{enumerate}
\end{document}
