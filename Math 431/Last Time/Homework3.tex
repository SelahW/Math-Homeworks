\documentclass[12pt]{article}
\usepackage[utf8]{inputenc}
\usepackage{mathtools}
\usepackage{amsmath}
\usepackage{amsfonts}
\usepackage{amssymb}
\usepackage{amsthm}
\usepackage{enumitem}
\usepackage{centernot}
\newcommand{\subscript}[2]{$#1 _ #2$}
\newcommand{\N}{\mathbb{N}}
\newcommand{\Z}{\mathbb{Z}}
\newcommand{\R}{\mathbb{R}}
\renewcommand\qedsymbol{QED}
\newcommand{\divby}{%
  \mathrel{\text{\vbox{\baselineskip.65ex\lineskiplimit0pt\hbox{.}\hbox{.}\hbox{.}}}}%
  }
\newcommand{\notdivby}{\centernot\divby}
\newcommand\setItemnumber[1]{\setcounter{enumi}{\numexpr#1-1\relax}}
\title{\scalebox{2}{Math 431 Homework 1}}
\author{\scalebox{1.5}{Theo Koss}}
\date{September 2020}
\begin{document}
\maketitle
\section{Chapter 3 Problems}
\subsection{Problem 1}
\textbf{Find all subgroups of $\Z_{24}$.}
\newline Since a subset is only a subgroup if it is closed under addition, and any relatively prime number to 24 will ``span'' the whole set. In other words any subset generated by a relatively prime number to 24 will not be closed. Thus the only subgroups of $\Z_{24}$ are 1,2,3,4,6,8,12,24. For example the subgroup generated by 3 looks like:$\{3,6,9,12,15,18,21,0\}$ then it repeats.
\subsection{Problem 2}
\textbf{List the left and right cosets of the $<3>$ in $\Z_{24}$. Note that both are groups under addition.}
\newline Let $<3>=H$\begin{enumerate}
    \item Left cosets: \begin{enumerate}
        \item $0+H=\{0,3,6,9,12,15,18,21\}$
        \item $1+H=\{1,4,7,10,13,16,19,22\}$
        \item $2+H=\{2,5,8,11,14,17,20,23\}$
    \end{enumerate}
    \item Right cosets: \begin{enumerate}
        \item $H+0=\{0,3,6,9,12,15,18,21\}$
        \item $H+1=\{1,4,7,10,13,16,19,22\}$
        \item $H+2=\{2,5,8,11,14,17,20,23\}$
    \end{enumerate}
\end{enumerate}
\subsection{Problem 3}
\textbf{Let $G$ be a finite group with an element $g$ of order 7 and an element $h$ of order 13. Explain why $G$ must have at least 91 elements. That is, explain why $|G|\geq91$.}
\newline $g$ has order 7, which means the subgroup generated by $g$ is indeed a subgroup and it has 7 left cosets, namely $<g> =\{0,7,14,...91,...\}$. Since $g$ has order 7, it has 7 left cosets. They are:
\begin{enumerate}
    \item $0+g=\{0,7,14,...91,...\}$
    \item $1+g=\{1,8,15,...\}$
    \item $2+g=\{2,9,16,...\}$
    \newline\scalebox{1.5}{\qquad\vdots}
    \setItemnumber{7}
    \item $6+g=\{6,13,20,...\}$
\end{enumerate}
Thus $|G|$ is some $a\in\N,\text{ s.t. }|G|=7a$ \newline
Similarly, since $h$ has order 13, the subgroup it generates must also be a subgroup. Meaning it is closed, therefore $|G|= \text{some } 13b$, $b\in\N$.\newline Therefore the absolute lowest possible $|G|$ which satisfies both $|G|=7a=13b$ is $|G|=91$. So $|G|\geq91$.
\subsection{Problem 4}
\textbf{Suppose $G$ is a finite group of order 60. What are the possible orders of subgroups of $G$, and why? What subgroups of $G$ can we say definitely exist?}
\newline The possible subgroups of $G$ are generated by $g\in G$ s.t. $G|g$, that is to say that the only subgroups of $G$ are generated by factors of $G$. These are: $g\in\{1,2,3,4,5,6,10,12,15,20,30,60\}$. The order of these subgroups are the number of left cosets, or the amount of numbers (including 0) that are less that itself. So every $<g>$ has order $|<g>|=g$. We can say that every one of these definitely exists.
\subsection{Problem 5}
Let $G=<a>$ be a cyclic group generated by the element $a$. Let $H$ be a subgroup of $G$. Show that $H$ is also a cyclic group. (Hint: If H is non-trivial, let k be the smallest positive integer such that: $a^k\in H$).
\newline If $G$ is a cyclic group generated by $a$, then $a$ is a factor of $G$. If $H$ is a sub\underline{group} then it must be closed and so some $b\in H$ is a factor of $H$, and since any factor of a factor of $G$ is a factor of $G$ itself, the group $H$ is a cyclic group.
\end{document}