\documentclass[a4paper,17pt]{extarticle}
\usepackage{geometry}
\usepackage{microtype}
\usepackage[utf8]{inputenc}
\usepackage[table,xcdraw]{xcolor}
\usepackage{mathtools}
\usepackage{amsthm}
\usepackage{amsmath}
\usepackage{amsfonts}
\usepackage{amssymb}
\usepackage{centernot}
\usepackage{marvosym}
\usepackage{enumitem}
\usepackage{hyperref}
\setcounter{tocdepth}{1}
\theoremstyle{definition}
\newtheorem{definition}{Definition}
\let\marvosymLightning\Lightning
\renewcommand{\skip}{\par\null\par}
\newcommand{\T}{\mathcal T}
\renewcommand{\O}{\mathcal{O}}
\renewcommand{\Lightning}{\scalebox{1.5}{\marvosymLightning}}
\renewcommand{\L}{\mathcal{L}}
\renewcommand{\leq}{\leqslant}
\renewcommand{\geq}{\geqslant}
\newtheorem{theorem}{Theorem}
\newtheorem{corollary}{Corollary}[theorem]
\newtheorem*{remark}{Remark}
\newcommand{\N}{\mathbb{N}}
\newcommand{\Z}{\mathbb{Z}}
\newcommand{\R}{\mathbb{R}}
\newcommand{\Q}{\mathbb{Q}}
\renewcommand\qedsymbol{QED}
\title{\scalebox{1.3}{Math 431 Homework 2}}
\author{\scalebox{1.4}{Theo Koss}}
\date{September 2022}
\begin{document}
\maketitle
\section{Section 2.3}
\begin{itemize}
    \item Problem 8: Prove the Liebniz rule for $f^{(n)}(x)$, where $f^{(n)}$ is the $n$th derivative of $f$; that is, show that\begin{align}\label{(1)}
        (fg)^{(n)}(x)=\sum_{k=0}^n{n\choose k}f^{(k)}(x)g^{(n-k)}(x)
    \end{align}
    \begin{proof}
    \textbf{Base Case:} For $n=1$, \begin{align}\label{(2)}
        (fg)'(x)=f'(x)g(x)+f(x)g'(x) \end{align} Which is the derivative product rule.\skip\textbf{Inductive Case:} Assume \ref{(1)} is true for $n\geq1$. Then: \begin{align}
        (fg)^{(n+1)}(x)&=\sum_{k=0}^{n+1}{{n+1}\choose k}f^{(k)}(x)g^{((n+1)-k)}(x)\\
        &=\sum_{k=0}^n{n\choose k}f^{(k)}(x)g^{(n-k)}(x)\\&\cdot\sum_{k=0}^1{1\choose k}f^{(k)}(x)g^{(1-k)}(x)\\
        (fg)^n(x)\cdot(fg)'(x)&=(fg)^{(n)}(x)\cdot\sum_{k=0}^1{1\choose k}f^{(k)}(x)g^{(1-k)}(x)\\
    \end{align}
    Dividing both sides by $(fg)^{(n)}(x)$:\begin{align}
        (fg)'(x)&=\sum_{k=0}^1{1\choose k}f^{(k)}(x)g^{(1-k)}(x)\\
        &=f'(x)g(x)+f(x)g'(x)
    \end{align}Which is true, as shown in \ref{(2)}. Thus the rule holds for $1,n,$ and $n+1$.
    \end{proof}
    \item Problem 15e: For each of the following pairs of numbers $a$ and $b$, calculate $\gcd(a, b)$ and find integers $r$ and $s$ such that $\gcd(a, b) = ra + sb$.\begin{itemize}[label=(e)]
        \item 23771 and 19945\begin{align*}
            23771&=19945\cdot1+3826\\
            19945&=3826\cdot5+815\\
            3826&=815\cdot4+556\\
            815&=556\cdot1+259\\
            556&=249\cdot2+68\\
            249&=68\cdot3+45\\
            68&=45\cdot1+23\\
            45&=23\cdot1+22\\
            23&=22\cdot1+1\\
        \end{align*}
        Therefore the gcd is 1. Inverse Euclidean Algorithm:\begin{enumerate}[label=\roman*.]
            \item $1=23-1(22)$
            \item $22=45-1(23)$
            \item $23=68-1(45)$
            \item $45=249-3(68)$
            \item $68=556-2(249)$
            \item $249=815-1(556)$
            \item $556=3826-4(815)$
            \item $815=19945-5(3826)$
            \item $3826=23771-1(19945)$
        \end{enumerate}
        (Did on paper) $r=881$, $s=-1050$.\begin{align*}
            881\cdot23771-1050\cdot19945=1
        \end{align*}
    \end{itemize}
\end{itemize}
\end{document}