\documentclass[12pt]{article}
\usepackage[utf8]{inputenc}
\usepackage{mathtools}
\usepackage{amsmath}
\usepackage{amsfonts}
\usepackage{amssymb}
\usepackage{amsthm}
\usepackage{enumitem}
\usepackage{centernot}
\usepackage{marvosym}
\usepackage{lscape}
\let\marvosymLightning\Lightning
\newtheorem{theorem}{Theorem}
\newtheorem{corollary}{Corollary}[theorem]
\newtheorem*{remark}{Remark}
\newtheorem{innercustomthm}{Theorem}
\newenvironment{customthm}[1]
  {\renewcommand\theinnercustomthm{#1}\innercustomthm}
  {\endinnercustomthm}
\newcommand{\subscript}[2]{$#1 _ #2$}
\newcommand{\N}{\mathbb{N}}
\newcommand{\Z}{\mathbb{Z}}
\newcommand{\R}{\mathbb{R}}
\newcommand{\C}{\mathbb{C}}
\renewcommand\qedsymbol{QED}
\newcommand{\divby}{%
  \mathrel{\text{\vbox{\baselineskip.65ex\lineskiplimit0pt\hbox{.}\hbox{.}\hbox{.}}}}%
  }
\newcommand{\notdivby}{\centernot\divby}
\newcommand\setItemnumber[1]{\setcounter{enumi}{\numexpr#1-1\relax}}
\title{\scalebox{2}{Math 431 Homework 4}}
\author{\scalebox{1.5}{Theo Koss}}
\date{October 2020}
\begin{document}
\maketitle
\section{Problem 1}
Let $E(\Z_{11})$ be the elliptic curve over $\Z_{11}$ defined by $y^2=x^{3}+x+1$ together with the point at infinity.
\begin{enumerate}
    \item Find all elements in E.
    \newline \begin{table}[h]
\begin{tabular}{|l|l|l|l|l|l|}
\hline
$y$ & $y^2$ & $y^2\mod{11}$ & $x$ & $x^3+x+1$ & $x^3+x+1\mod{11}$ \\ \hline
0   & 0     & 0             & 0   & 1         & 1                 \\ \hline
1   & 1     & 1             & 1   & 3         & 3                 \\ \hline
2   & 4     & 4             & 2   & 11        & 0                 \\ \hline
3   & 9     & 9             & 3   & 31        & 9                 \\ \hline
4   & 16    & 5             & 4   & 69        & 3                 \\ \hline
5   & 25    & 3             & 5   & 131       & 10                \\ \hline
6   & 36    & 3             & 6   & 223       & 3                 \\ \hline
7   & 49    & 5             & 7   & 251       & 9                 \\ \hline
8   & 64    & 9             & 8   & 521       & 4                 \\ \hline
9   & 81    & 4             & 9   & 739       & 2                 \\ \hline
10  & 100   & 1             & 10  & 1011      & 10                \\ \hline
\end{tabular}
\end{table}
\newline So $E(\Z_{11})=\{O,(0,1),(0,10),(1,5),(1,6),(2,0),(3,3),(3,8),(4,5),(4,6),(6,5),\newline
(6,6),(7,3),(7,8),(8,2),(8,9)\}$
    \item Find $(3,8)+(4,6)$ in E.
    $m=\frac{6-8}{4-3}=-\frac{2}{1}=-2\equiv9\mod{11}$.
    \newline Point-slope form: $$y-8=9(x-3)$$ $$y=9x-19$$
    \newline Substituting in to $E$: $$(9x-19)^2=x^3+x+1$$ $$81x^2-342x+361\equiv4x^2-x+9\mod{11}$$ $$x^3-3x^2+2x-8=0$$ So $x=-8$. $y=-73-19=-92\equiv7\mod{11}$, so $(-P,Q)=(-8,7)$. Then $(P,Q)=(8,7)$.
    \item Find $(1,6)+(1,6)$ in E.
    \newline Implicit differentiation: $$\frac{d}{dx}(y^2)=\frac{d}{dx}(x^3+x+1)$$ $$2y\frac{dy}{dx}=3x^2+1$$ $$\frac{dy}{dx}=\frac{3x^2+1}{2y}$$ At $P=(1,6)$, $\frac{dy}{dx}=\frac{1}{3}$. $y-6=\frac{1}{3}(x-1)$. So $y=\frac{x}{3}+\frac{17}{3}$. Plugging into $E$: $$(\frac{x}{3}+\frac{17}{3})^2=x^3+x+1$$ $$\frac{x^2+34x+289}{9}=x^3+x+1$$Multiply by 9 on both sides and simplify mod 11:$$x^2+x+3=9x^3+9x+9$$ $$9x^3-x^2+8x+6=0\mod{11}$$ So $x=1$. \newline\scalebox{3}{I'm confused :(}
    \item Find $(1,6)+(1,5)$ in E.
    \item Is E cyclic? Explain why or why not.
\end{enumerate}
\end{document}