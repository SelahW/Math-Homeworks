\documentclass[a4paper,12pt]{extarticle}
\usepackage{geometry,microtype,mathtools,amsthm,amssymb,enumitem}
\usepackage[utf8]{inputenc}
\usepackage[font=small,labelfont=bf]{caption}
\theoremstyle{definition}
\newcommand{\R}{\mathbb{R}} \newcommand{\Q}{\mathbb{Q}} \newcommand{\Z}{\mathbb{Z}} \newcommand{\N}{\mathbb{N}} \newcommand{\myskip}{\par\null\par} \renewcommand\qedsymbol{QED} \renewcommand{\leq}{\leqslant}\renewcommand{\geq}{\geqslant}
\newtheorem*{theorem}{Theorem}
\title{Math 431 Midterm} 
\author{Theo Koss}
\date{October 2022} 
\begin{document}
    \maketitle
    \section*{Problem 1}\begin{enumerate}[label=(\alph*)]
        \item An equivalence relation is a relation (a set of ordered pairs) $\sim$ on a set $X$ that is: \begin{enumerate}[label=\roman*.]
            \item Reflexive: $\forall a\in X,\ a\sim a$.
            \item Symmetric: $\forall a,b\in X,\ a\sim b\implies b\sim a$.
            \item Transitive: $\forall a,b,c\in X,\ \text{if }a\sim b\text{ and }b\sim c,\text{ then }a\sim c$.
        \end{enumerate}
        \item The logical flaw in the argument is that for all $x$, there must exist a $y$ such that $x\sim y$. This is not true in general.
        \item The reflexive property is not redundant. Consider the set $X=\{x,y,z\}$, defined by the following relation:\begin{itemize}
            \item $x\sim x$
            \item $z\sim z$
            \item $x\sim z$
            \item $z\sim x$
        \end{itemize}
        \begin{enumerate}[label=\roman*.]
            \item This is not reflexive, by counterexample, $y\not\sim y$.
            \item This is symmetric, because for all elements $x,y\in X$ such that $x\sim y$, $y\sim x$.
            \item This relation is transitive, and we can check exhaustively:\begin{itemize}
                \item $x\sim z$ and $z\sim x$, we check $x\sim x$. \checkmark
                \item $z\sim x$ and $x\sim z$, we check $z\sim z$. \checkmark
            \end{itemize}
        \end{enumerate}
    \end{enumerate}
    \section*{Problem 2}\begin{enumerate}[label=(\alph*)]
        \item $gcd(1372,1050)$ using the Euclidean Algorithm:\begin{align*}
            1372&=1050\cdot1+322\\
            1050&=322\cdot3+84\\
            322&=84\cdot3+70\\
            84&=70\cdot1+14\\
            70&=14\cdot5+0
        \end{align*} Thus $gcd(1372,1050)=14$.
        \item $$1372\cdot-13+1050\cdot17=14$$
    \end{enumerate}
    \section*{Problem 3}\begin{enumerate}[label=(\alph*)]
        \item Rather than using theorem 4.10 in the text (probably what you intended with this problem), I'll use theorems  9.6, 9.7, and 9.8 from the text.\myskip
            \begin{theorem}[9.6]
            \textit{ Let $\phi:G\to H$ be an isomorphism of two groups. Then the following statements are true.} [Statements 1,2,3,5 I've left out as they're not important]\textit{\begin{enumerate}[label=4.]
            \item If $G$ is cyclic, then $H$ is cyclic.
            \end{enumerate}}
            \end{theorem}
        \myskip
        \begin{theorem}[9.7]\textit{ All cyclic groups of infinite order are isomorphic to $\Z$.}\end{theorem}
        \myskip\begin{theorem}[9.8]
        \textit{If $G$ is a cyclic group of order $n$, then $G$ is isomorphic to $\Z_n$.}
        \end{theorem}\begin{proof} Let $G$ be a cyclic group, then either:\begin{enumerate}[label=\arabic*)]
            \item $G$ is infinite, in which case thm. 9.8 shows that $G\cong\Z$, and thus for any subgroup $H$ of $G$, $H\cong n\Z,\text{ or }H\cong\Z_n,\ n\in\N$, both of which are clearly cyclic.
            \item $G$ is finite. WLOG, let $|G|=k$. Then by thm 9.8, $G\cong\Z_k$. Then let $H$ be a subgroup of $G$. We know $H\cong\Z_m$, where $0\leq m<k$, all such subgroups $H$ are cyclic, since the integers modulo anything is always a cyclic group.
        \end{enumerate} Therefore, any subgroup $H$ of a cyclic group $G$ is also cyclic.
        \end{proof}
        \item If $G$ is cyclic, of order $p$ (where $p$ prime), then $G$ has no proper subgroups as a result of theorem 4.13 in the text.\myskip\begin{theorem}[4.13]
        \textit{ Let $G$ be a cylic group of order $n$ and suppose that $a\in G$ is a generator of the group. If $b=a^k$, then the order of $b$ is $n/d$, where $d=gcd(k,n)$.}
        \end{theorem}\myskip Therefore, the order of any element $b=a^k\in\Z_p$ must be equal to $p/gcd(k,p)$. In other words, if $k$ and $p$ are coprime, then $b$ is also a generator of $G$ (and thus, any subgroup containing $b$ is not a proper subgroup). Since \emph{every} number $k<p$ is coprime to $p$, there are no proper subgroups of $G$.
    \end{enumerate}
    \section*{Problem 4}\begin{enumerate}[label=(\alph*)]
        \item $$A=\begin{pmatrix}a & b \\c & d\end{pmatrix}\quad B=\begin{pmatrix}p & q \\r & s\end{pmatrix}\quad C=\begin{pmatrix}w & x\\y & z\end{pmatrix}$$
        \begin{align*}
            A(BC)&=\begin{pmatrix}a & b \\c & d\end{pmatrix}\cdot\begin{pmatrix}pw+qy&px+qz\\ rw+sy&rx+sz\end{pmatrix}\\
        &=\begin{pmatrix}a\left(pw+qy\right)+b\left(rw+sy\right)&a\left(px+qz\right)+b\left(rx+sz\right)\\ c\left(pw+qy\right)+d\left(rw+sy\right)&c\left(px+qz\right)+d\left(rx+sz\right)\end{pmatrix}\\
        (AB)C&=\begin{pmatrix}ap+br&aq+bs\\ cp+dr&cq+ds\end{pmatrix}\cdot\begin{pmatrix}
            w & x\\
             y & z
        \end{pmatrix}\\
        &=\begin{pmatrix}\left(ap+br\right)w+\left(aq+bs\right)x&\left(ap+br\right)y+\left(aq+bs\right)z\\ \left(cp+dr\right)w+\left(cq+ds\right)x&\left(cp+dr\right)y+\left(cq+ds\right)z\end{pmatrix}\\
        &=A(BC)
        \end{align*} Therefore $A(BC)=(AB)C$ for 2$\times$2 matrices $A,B,C$.
        \item A 2x2 matrix $A=\begin{pmatrix}a & b \\c & d\end{pmatrix}$ is invertible iff $ad-bc\neq0$.\begin{proof}($\Longrightarrow):$ Consider an invertible $2\times 2$ matrix $A=\begin{pmatrix}a & b \\c & d\end{pmatrix}$ and, seeking contradiction, let $ad-bc=0$. From a result in linear algebra, \textbf{$A$ is invertible iff its reduced echelon form is exactly the identity matrix}.(As a ``proof'', not a real proof, invertibility is invariant under row operations, therefore if we do row operations to a matrix and the result is not invertible, the original matrix MUST not be). Doing some row operations on $A$:$$\begin{pmatrix}a & b \\c & d\end{pmatrix}\xrightarrow{cR_1}\begin{pmatrix}ac & bc \\c & d\end{pmatrix}\xrightarrow{aR_2}\begin{pmatrix}ac & bc \\ca & da\end{pmatrix}\xrightarrow{R_2=R_2-R_1}\begin{pmatrix}ac & bc \\0 & ad-bc\end{pmatrix}=\begin{pmatrix}ac & bc \\0 & 0\end{pmatrix}$$ This is clearly not equal to the identity, therefore $ad-bc$ must not be equal to 0. Contradiction.
        \myskip 
        $(\Longleftarrow):$ Let $A=\begin{pmatrix}a & b \\c & d\end{pmatrix}$ and $ad-bc\neq0$. Then $A^{-1}=\frac{1}{ad-bc}\begin{pmatrix}d & -b \\-c & a\end{pmatrix}$, and since $ad-bc\neq0$, this exists (and is unique). Then, $$AA^{-1}=\begin{pmatrix}1 & 0 \\0 & 1\end{pmatrix}=I$$ Therefore, $A$ is invertible.
        \end{proof}
        \item $GL_2(\R)$ is closed under matrix multiplication.\begin{proof}
        Consider $A,B\in GL_2(\R)$, that is, $A=\begin{pmatrix}a & b \\c & d\end{pmatrix}$ and $B=\begin{pmatrix}p & q \\r & s\end{pmatrix}$, both with inputs from $\R$, and such that $ad-bc\neq0$, and $ps-qr\neq0$. Seeking contradiction, let $AB\not\in GL_2(\R)$. Then at least one of the following must be true:\begin{enumerate}[label=\arabic*)]
            \item $AB$ has inputs that are not in $\R$. $$AB=\begin{pmatrix}ap+br&aq+bs\\ cp+dr&cq+ds\end{pmatrix}$$ Since $\R$ is closed under regular addition and multiplication, this is false.
            \item $AB$ has determinant 0. So $$AB=\begin{pmatrix}ap+br&aq+bs\\ cp+dr&cq+ds\end{pmatrix}\implies \Delta(AB)=\Delta(A)\cdot\Delta(B)=0$$ Therefore, one of $\Delta(A)$ and $\Delta(B)$ must be equal to 0. However this is untrue since $A,B\in GL_2(\R)$.
        \end{enumerate} Contradiction. Therefore $AB\in GL_2(\R)$. Similarly, $BA\in GL_2(\R)$.
        \end{proof}
    \end{enumerate}
    \section*{Problem 5}Write each of the following two elements of $S_9$ as a product of disjoint cycles.
        \begin{enumerate}[label=(\alph*)]
            \item $\left(\begin{array}{lllllllll}1 & 2 & 3 & 4 & 5 & 6 & 7 & 8 & 9 \\ 5 & 6 & 1 & 9 & 2 & 3 & 4 & 8     & 7\end{array}\right)=(1\ 5\ 2\ 6\ 3)(4\ 9\ 7)(8)$.
            \item 
                $(1\ 3\ 5\ 7\ 9)(1\ 4\ 7)(4\ 5\ 7\ 8):=\\(1\mapsto4)(2\mapsto2)(3\mapsto5)(4\mapsto7)(5\mapsto3)(6\mapsto6)(7\mapsto8)(8\mapsto9)(9\mapsto1)\\=(1\ 4\ 7\ 8\ 9)(2)(3\ 5)(6)$.
\end{enumerate}
\section*{Problem 6}Disjoint cycles in any $S_n$ will always commute because the two cycles have nothing to do with one another, therefore the order in which you do them will not matter. Informally it is akin to moving things around in the pantry, then moving things around in the freezer. It doesn't matter which one you do first, because they don't have any affect on one another.
\section*{Problem 7} The symmetry group of the octahedron consists of:\begin{itemize}
    \item $e$, the identity.
    \item The rotations through the vertices, these are rotations by $90^{\circ}$ (order 4), through 3 of the faces of the surrounding cube. This gives 9 such rotations.
    \item The rotations through the edges, these are rotations by $180^{\circ}$ (order 2), through the center of the 6 edges (both of the cube and of the octahedron). There are 6.
    \item The rotations through the center of the faces, these are rotations by $120^{\circ}$ (order 3), through the 4 vertices of the surrounding cube. There are 8.
\end{itemize}In total, there are 24 symmetries of the octahedron. The same as the cube.
\section*{Problem 8}\begin{enumerate}[label=(\alph*)]
    \item For any $a\in G$, $t_a$ is a bijection.\begin{proof} Consider some $a\in G$. Then $t_a(x)=ax$ for all $x\in G$. \begin{itemize}
        \item $t_a:G\to G$ is injective: Let $t_b(x)=t_c(x)$, then $bx=cx$ and therefore $b=c$.
        \item $t_a:G\to G$ is surjective: For each $c\in G$, we must show that there is some $b\in G$ such that $t_b(x)=cx$. \begin{align*}
            t_b(x)&=cx\\
            bx&=cx\\
            b&=cxx^{-1}\\
            b&=c\\
        \end{align*} There is indeed such a $b$, particularly, the $b\in G$ which maps to $c\in G$ is simply $c$ itself.
    \end{itemize} We know the identity element of the $S_G$ group is the identity function, $id(x)=x$. Now it is easy to see that the inverse element of any $t_a(x)$ is simply the $t_b(x)$ where $b$ is the inverse of $a$ in the group $G$. Thus, $$id(x)=t_a(x)\circ t_b(x)=t_a(x)\circ t_{a^{-1}}(x)=t_a(a^{-1}(x))=aa^{-1}x=x$$ Thus $t_a:G\to G$ is a bijetion for any choice of $a$.
    \end{proof}
    \item $H=\{t_a|a\in G\}$ is a subgroup of $S_G$.\begin{proof}By proposition 3.31 in the text:\myskip \textbf{Proposition 3.31} \textit{Let $H$ be a subset of a group $G$. Then $H$ is a subgroup of $G$ if and only if $H \neq \emptyset$, and whenever $g, h \in H$ then $g h^{-1}$ is in $H$.}\myskip Clearly, the set of $t_a$s is a subset of $S_G$, and, so long as $\exists a\in G$, $H\neq\emptyset$. It remains to show that whenever $g,h\in H$, $gh^{-1}\in H$. Consider two elements $g,h$ of $H$. These are two (not necessarily distinct) bijections $t_a:G\to G$. We will name them $t_g$ and $t_h$. As seen above, the inverse of $t_h$ is $t_{h^{-1}}$. Therefore $$gh^{-1}=t_g\circ t_{h^{-1}}=t_{gh^{-1}}$$ Now, is this element in $H$? Well, $g\in H$, from our assumption. Also from the assumption, $h\in H$, which necessarily means $h^{-1}\in H$. This element is just a bijection $t_{gh^{-1}}:G\to G$, which sends $t_{gh^{-1}}(x)$ to $gh^{-1}x$. Therefore it is in $H$, as required.
    \end{proof}
    \item Define a function $\phi:G\to H$ by $\phi(a)=t_a$ for every $a\in G$. Show that $\phi$ is an isomorphism.
        \begin{proof} To show $\phi$ is an isomorphism we must show the following:\begin{enumerate}[label=\roman*.]
            \item $\phi$ is injective: Consider $\phi(a)=\phi(b)$, then $t_a=t_b$, and, by above, this shows $a=b$.
            \item $\phi$ is surjective: Consider arbitrary $b\in H$, then $b=\phi(a)$, we must show $a\in G$ exists. Since $b\in H$, $b=t_b$, so $t_b=t_a$, and again the element which maps to $b\in H$ is $b\in G$.
            \item $\phi(ab)=\phi(a)\circ\phi(b)$: $$\phi(ab)=t_{ab}=t_a\circ t_b=\phi(a)\circ\phi(b)$$
        \end{enumerate}
        \end{proof}
    \item Therefore any group is isomorphic to a group of permutations, and in particular, any finite group is isomorphic to a subgroup of some $S_n$.\begin{proof}
        It remains to show that $t_a$ is a permutation of $G$. To show this we need to show $t_a$ is a bijection for all $a$, and sends elements of $G$ to elements of $G$. This is proven above in part (a). Now we are ready to complete the proof. Consider an arbitrary group $G$, with $a\in G$. Now define a function $t_a:G\to G$, this is a permutation of $G$. Now define a new group $H=\{t_a|a\in G\}$, this is all the permutations of $G$. As shown before $H$ is indeed a group. Finally, define $\phi:a\mapsto t_a$ from $G$ to $H$. This is an isomorphism between $G$ and $H$, showing that they are, in essence, ``the same group''. $$G\overset{t_a}{\longrightarrow}G\overset{\phi}{\longrightarrow}H$$
        Finally, consider some finite group $G$. Then this has finite order, WLOG, let $|G|=k$. Since $\phi$ isomorphic, we find $|H|=k$, particularly, this means there are exactly $k$ permutations of $G$. Thus, we can define a new isomorphism $\psi:H\to K$, where $K$ is a subgroup of $S_n$ with order $k$. This is an isomorphism because permutations in $H$ are simply relabeled to permutations in $K$. Then for finite $G$ we can expand our diagram: $$G\overset{t_a}{\longrightarrow}G\overset{\phi}{\longrightarrow}H\overset{\psi}{\longrightarrow}K$$
    \end{proof}
    \end{enumerate}
\end{document}