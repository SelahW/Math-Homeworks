\documentclass[hidelinks,12pt]{article}
\usepackage[utf8]{inputenc}
\usepackage{microtype}
\usepackage{mathtools}
\usepackage{amsthm}
\usepackage{amsmath}
\usepackage{amsfonts}
\usepackage{amssymb}
\usepackage{centernot}
\usepackage{marvosym}
\usepackage{enumitem}
\usepackage{hyperref}
\setcounter{tocdepth}{1}
\let\marvosymLightning\Lightning
\renewcommand{\geq}{\geqslant}
\renewcommand{\leq}{\leqslant}
\newtheorem{theorem}{Theorem}
\newtheorem{corollary}{Corollary}[theorem]
\newtheorem*{remark}{Remark}
\renewcommand\qedsymbol{QED}
\newcommand{\R}{\mathbb{R}}
\newcommand{\N}{\mathbb{N}}
\newcommand{\Z}{\mathbb{Z}}
\newcommand{\Q}{\mathbb{Q}}
\newcommand{\divby}{%
    \mathrel{\text{\vbox{\baselineskip.65ex\lineskiplimit0pt\hbox{.}\hbox{.}\hbox{.}}}}%
}
\newcommand{\notdivby}{\centernot\divby}
\title{\scalebox{2}{Math 524 Homework 5}}
\author{\scalebox{1.5}{Theo Koss}}
\date{May 2024}

\begin{document}

\maketitle
\section{Section 10.2}
\begin{enumerate}
    \item[2(a).] \[\lim_{(x,y)\to(0,0)}\frac{xy^2}{x^2+y^2}=\lim_{r\to0}(\frac{r\cos{\theta}(r\sin{\theta})^2}{(r\cos\theta)^2+(r\sin\theta)^2})=\lim_{r\to0}(r\sin^2\theta\cos\theta)=0\]
        This agrees with definition 10.2.1 because $\lim_{(x,y)\to(0,0)}\frac{xy^2}{x^2+y^2}=0$ iff for every $\varepsilon>0$ there exists a $\delta>0$ such that $|f(x,y)-0|<\varepsilon$ whenever $(x,y)\in D$ and $0<\sqrt{x^2+y^2}<\delta$.\begin{proof}
            Let $\varepsilon$ be given. Then we want $\exists\delta$ such that $f(x,y)=\frac{xy^2}{x^2+y^2}<\varepsilon$ when $0<\sqrt{x^2+y^2}<\delta$. So we need $0<|x|<\delta$ and $0<|y|<\delta$. \begin{align*}
                |f(x,y)|&=|\frac{xy^2}{x^2+y^2}|\\
                        &=|\frac{r\cos{\theta}(r\sin{\theta})^2}{(r\cos\theta)^2+(r\sin\theta)^2}|\\
                        &=|r\sin^2\theta\cos\theta|\\
                        &< |r|\\
                \text{Let }\delta&=\varepsilon\\
            \end{align*}
            Since $0<r<\delta$, $\delta=\varepsilon$ forces $|f(x,y)|<\varepsilon$. As required.
        \end{proof}
    \item[3(g).] Determine if the given limit is finite.
        \[\lim_{(x,y)\to(0,0)}\frac{x^2-y^4}{x^2+y^4}\]
        Approach from $x=0$, 
        \[\lim_{y\to0}\frac{-y^4}{y^4}=-1\]
        Approach from $x=y$,
        \[\lim_{y\to0}\frac{y^2-y^4}{y^2+y^4}=\frac{1-y^2}{1+y^2}=1\]
        Therefore the limit is infinite.
\end{enumerate}
\section{Section 10.4}
\begin{enumerate}
    \item[3(b).] Show that $f(x,y)=\sqrt{x^2+y^2}$ is not differentiable at the origin by showing $f_x(0,0)$ does not exist. $f(x,0)=\sqrt{x^2}=|x|$ which is not differentiable.
    \item[4.] $f(x,y)=\sqrt[3]{xy}$
        \begin{enumerate}
            \item Show that $f_x(0,0)=0=f_y(0,0)$.
                \[f(x,0)=\sqrt[3]{0}=0\ f(0,y)=\sqrt[3]{0}=0\]
            \item Find $\nabla F=(0,0)$.
            \item Show that $f$ is not differentiable at $(0,0)$.
                \begin{proof}
                    From definition 10.4.1, we must show $f(P+h)=f(P)+m\cdot h+\varepsilon\|h\|$ has \\$\varepsilon\not\to0$ as $h\to0$.
                \end{proof}
            \item Is $f$ continuous at $(0,0)$? Yes:
                \begin{proof}
                    Let $\varepsilon>0$ be given. Choose $\delta=\sqrt[3]{\varepsilon^2}$ and suppose that $0<|(x,y)|<\delta$.
                    \begin{align*}
                        0 &<\sqrt{x^2+y^2}<\delta\\
                        0 &<\sqrt{x^2+y^2}<\sqrt{\varepsilon^3}\\
                    \end{align*}
                    And, $x<\sqrt{x^2+y^2}$ $y<\sqrt{x^2+y^2}$
                    \begin{align*}
                        |f(x,y)|&=|\sqrt[3]{xy}|\\
                                &<|\sqrt[3]{(\sqrt{x^2+y^2})(\sqrt{x^2+y^2})}|\\
                                &<|\sqrt[3]{(\sqrt{\varepsilon^3})(\sqrt{\varepsilon^3})}|\\
                                &=|\sqrt[3]{\varepsilon^3}|\\
                                &=\varepsilon
                    \end{align*}
                \end{proof}
        \end{enumerate}
\end{enumerate}
\section{Section 10.5}
\begin{enumerate}
    \item[2.]\begin{enumerate}
            \item Show $D_if=-D_{-i}f$, provided $f_x$ exists.\\
                Suppose $f_x$ exists, so $D_if=f_x=\lim_{h\to0}\frac{f(P+hi)-f(P)}{h}$ exists.
                \[-D_{-i}f=-\left[\lim_{h\to0}\frac{f(P-hi)-f(P)}{h}\right]=\]\[=-\left[-\left[\lim_{h\to0}\frac{f(P+hi)-f(P)}{h}\right]\right]\]\[=-(-f_x)=f_x\]
            \item Show that if $f$ is differentiable at $(a,b)$  then for any unit vector $u$, $D_{-u}f(a,b)=-D_{u}f(a,b)$.
                \begin{proof}
                    By theorem 10.5.2, since $f$ is differentiable at $(a,b)$, we have $D_uf(a,b)$ exists in any direction $u$, and that \[D_uf(a,b)=\nabla f(a,b)\cdot u\]
                    So, \[D_{-u}f(a,b)=\nabla f(a,b)\cdot -u=-(\nabla f(a,b)\cdot u)=-D_u f(a,b)\]
                \end{proof}
            \item If $D_uf$ exists for a unit vector $u$, show that $D_{-u}f=-D_uf$.
                \begin{proof}
                    Let $D_uf$ exist for some unit vector $u$. That is, the limit \[\lim_{h\to0}\frac{f(P+hu)-f(P)}{h}\] exists.
                    \begin{align*}
                        -D_{-u}f&=-\left[\lim_{h\to0}\frac{f(P-hu)-f(P)}{h}\right]\\
                                &=-\left[-\left[\lim_{h\to0}\frac{f(P+hu)-f(P)}{h}\right]\right]\\
                                &=\lim_{h\to0}\frac{f(P+hu)-f(P)}{h}\\
                                &=D_uf\\
                    \end{align*}
                \end{proof}
        \end{enumerate}
\end{enumerate}
\section{Section 11.1}
\begin{enumerate}
    \item[1(b).]Prove part (b) of lemma 11.1.1: If $Q$ is a partition of $R$ and $P\subseteq Q$, then $L(P,f)\leq L(Q,f)$ and $U(Q,f)\leq U(P,f)$.
        \begin{proof}
            Let $Q$ be a partition of $R$ and $P\subseteq Q$. If $P=Q$, we are done because $L(P,f)=L(Q,f)$ and $U(Q,f)=U(P,f)$. So suppose $P=(x_0,x_1,\dots,x_n)$, and $Q$ contains all of $P$ and one extra point of $[a,b]$, say $c\in[x_{i-1},x_i]$ for some $i$ between 1 and $n$. Then let,
\begin{align*}
    m_i&=\inf\left\{f(x)\mid x\in[x_{i-1},x_i]\right\},\\
    r_1&=\inf\left\{f(x)\mid x\in[x_{i-1},c]\right\},\\
    r_2&=\inf\left\{f(x)\mid x\in[c,x_i]\right\}
\end{align*}
We have $m_i=\min(r_1,r_2)$, so now:
            \begin{align*}
                L(P,f)&=\sum_{k=1}^nm_k\Delta x_k\\
                      &=\sum_{k=1}^{i-1}m_k(x_k-x_{k-1})+m_i(x_i-x_{i-1})+\sum_{k=i+1}^nm_k\Delta x_k\\
                      &\leq\sum_{k=1}^{i-1}m_k(x_k-x_{k-1})+r_1(c-x_{i-1})+r_2(x_i-c)+\sum_{k=i+1}^nm_k\Delta x\\
                      &=L(Q,f)\\
            \end{align*}
            Then, for $U(Q,f)\leq U(P,f)$, we do the same argument except $M_i=\sup\{\dots\}$ above, and $R_1$ and $R_2$ are also supremum. Then $M_i=\max(R_1,R_2)$ and essentially the same argument follows.
        \end{proof}
    \item[1(c).]Prove Theorem 11.1.3: \begin{theorem}A bounded function $f(x,y)$ on a rectangle $R=[a,b]\times[c,d]$ is Riemann integrable iff for any $\varepsilon>0$, there exists a partition $P$ of $R$ such that $U(P,f)-L(P,f)<\varepsilon$.
        \end{theorem}
        \begin{proof}
            $(\implies):\ $ Let $f(x,y)$ on $R$ be Riemann integrable. Then by definition 11.1.2, \[\underline{\iint_R}f=I=\overline{\iint_R}f\]
            Therefore, \[\sup\{L(P,f)\mid P\text{ is a partition}\}=\inf\{U(P,f)\mid P\text{ is a partition}\}\] So there exists a partition $P$ with $U(P,f)-L(P,f)=0$ which is less than any $\varepsilon$.\\
            $(\impliedby):\ $ Suppose for any $\varepsilon>0$, there exists a partition $P$ of $R$ such that $U(P,f)-L(P,f)<\varepsilon$. As $\varepsilon\to0$, $\sup{L(P,f)}$ gets closer and closer to $\inf{U(P,f)}$. $\sup{L(P,f)}$ is bounded above by $\inf{U(P,f)}$ and $\inf{U(P,f)}$ is bounded below by $\sup{L(P,f)}$. So as $\varepsilon\to0$, $\sup{L(P,f)}\to\inf{U(P,f)}$. Therefore, we have \[\underline{\iint_R}f=\overline{\iint_R}f\]
            So $f(x,y)$ is Riemann integrable.
        \end{proof}
\end{enumerate}
\section{Section 11.2}
\begin{enumerate}
    \item[7(a).]
        Suppose that $f:[a,b]\to\mathfrak{R}$ and $g:[c,d]\to\mathfrak{R}$ are Riemann integrable, and there is a rectangle $R=[a,b]\times[c,d]$. Prove that \[\iint_R f(x)g(y)=\left[\int_a^bf\right]\left[\int_c^dg\right]\]
        \begin{proof}
            Let $f$ and $g$ be Riemann integrable. That is, their integrals $\int_a^bf$ and $\int_c^dg$ exist. Since $f$ only depends on $x$ and $g$ only depends on $y$, we have that $\int_a^bf(x)dx$ ``looks like'' a constant w.r.t. $g$, and vice versa. 
            Now since an integral times a constant is equal to the integral of the function times that constant:
            \[\int cf=c\int f\]
            We can now write
            \[\iint_R f(x)g(y)=\int_c^d\left(\int_a^bf(x)dx\right)g(y)dy=\int_a^bf(x)dx\cdot\int_c^dg(y)dy\]
        \end{proof}
\end{enumerate}
\end{document}
