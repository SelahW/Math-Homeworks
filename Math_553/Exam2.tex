\documentclass[a4paper,17pt]{extarticle}
\usepackage{geometry}
\usepackage[utf8]{inputenc}
\usepackage[table,xcdraw]{xcolor}
\usepackage{mathtools}
\usepackage{amsthm}
\usepackage{amsmath}
\usepackage{amsfonts}
\usepackage{amssymb}
\usepackage{centernot}
\usepackage{marvosym}
\usepackage{enumitem}
\usepackage{hyperref}
\setcounter{tocdepth}{1}
\theoremstyle{definition}
\newtheorem{definition}{Definition}
\let\marvosymLightning\Lightning
\renewcommand{\skip}{\par\null\par}
\newcommand{\T}{\mathcal T}
\renewcommand{\O}{\mathcal{O}}
\renewcommand{\Lightning}{\scalebox{1.5}{\marvosymLightning}}
\renewcommand{\L}{\mathcal{L}}
\renewcommand{\leq}{\leqslant}
\renewcommand{\geq}{\geqslant}
\newtheorem{theorem}{Theorem}
\newtheorem{corollary}{Corollary}[theorem]
\newtheorem*{remark}{Remark}
\renewcommand\qedsymbol{QED}
\newcommand{\R}{\mathbb{R}}
\newcommand{\N}{\mathbb{N}}
\newcommand{\Z}{\mathbb{Z}}
\newcommand{\divby}{%
  \mathrel{\text{\vbox{\baselineskip.65ex\lineskiplimit0pt\hbox{.}\hbox{.}\hbox{.}}}}%
  }
\newcommand{\notdivby}{\centernot\divby}
\title{\scalebox{2}{Math 553 Exam 2}}
\author{\scalebox{1.5}{Theo Koss}}
\date{April 2022}
\begin{document}
\maketitle
\begin{enumerate}
\item Let $L:\R^3\to\R^3$ be a linear map, and let $S\subset\R^3$ be a regular surface. Suppose that $L(S)\subset S$, and consider the restriction $L|_S:S\to S$. Show that $d(L|_S)_p(v)=L(v)$ for all $p\in S$ and $v\in T_pS$. \begin{proof} A tangent vector $v\in T_pS$ is defined as $v=\alpha'(0)$ for some curve $\alpha:(-\epsilon,\epsilon)\to S$ with $\alpha(0)=p$. By definition, the differential of this restriction, $$d(L|_S)_p(v)=\frac{d}{dt}|_{t=0}L(\alpha(t))=L(\alpha'(0))=L(v)$$ using the chain rule, and the fact that $L$ is  a homogeneous function (as it's a linear map). 
\end{proof}
\item Describe the image of the Gauss map for each of the following surfaces:
    \begin{enumerate}[label=(\alph*).]
    \item The cylinder $\{(x,y,z)\in\R^3:x^2+y^2=1\}$.
    $$X(u,v)=(\cos u,\sin u,v)$$ $$N(p)=\frac{X_u(u,v)\times X_v(u,v)}{||X_u(u,v)\times X_v(u,v)||}=$$ $$=\frac{(\cos u,\sin u,0)}{||(\cos u,\sin u,0)||}=(\cos u,\sin u,0)$$ So the Gauss map maps this surface to the unit circle.
    \item The graph of $f(x,y)=x^2+y^2$.  $$X(u,v)=(u,v,u^2+v^2)$$ $$N(p)=\frac{(-2u,-2v,1)}{\sqrt{4u^2+4v^2+1}}=$$ $$=\left(\frac{-u}{\sqrt{u^2+v^2+\frac{1}{4}}},\frac{-v}{\sqrt{u^2+v^2+\frac{1}{4}}},\frac{1}{2\sqrt{u^2+v^2+\frac{1}{4}}}\right)$$
    \item The hyperboloid $\{(x,y,z)\in\R^3:z^2=x^2+y^2+1,z>0\}$. $$X(u,v)=(u,v,u^2+v^2+1)$$ $$N(p)=\frac{(-2u,-2v,1)}{\sqrt{4u^2+4v^2+1}}=$$ Same as above!
    \end{enumerate}
\item Consider the hyperboloid $S=\{(x,y,z)\in\R^3:z^2=x^2+y^2+1,z>0\}$, and let $\phi:\R^2\to S$ be given by $\phi(u,v)=(u,v,\sqrt{1+u^2+v^2})$.\begin{enumerate}[label=(\alph*).]
    \item Show that $\phi$ is differentiable with continuous inverse, and that $d\phi_{(u,v)}$ is injective. Conclude that $S$ is a regular surface.\begin{proof} The component functions of $\phi$, namely, $$\phi_1(u,v)=u,\quad \phi_2=v,\quad\phi_3=\sqrt{1+u^2+v^2}$$ All have continuous partial derivatives of all orders. Thus, $\phi(u,v)$ is differentiable. Its inverse, $\phi^{-1}:S\to R^2$ is given by $\phi^{-1}(u,v,\sqrt{1+u^2+v^2})=(u,v)$. All of its components are continuous therefore $\phi^{-1}$ is continuous. Let $Q=(u,v)$, $$d\phi_Q:=\begin{pmatrix}X_uQ& X_vQ\\
    Y_uQ& Y_vQ\\
    Z_uQ& Z_vQ
    \end{pmatrix}$$  $$d\phi_Q=\begin{pmatrix}1 & 0\\
    0& 1\\
    \frac{u}{\sqrt{1+u^2+v^2}} & \frac{v}{\sqrt{1+u^2+v^2}}
    \end{pmatrix}$$ Since the two columns of this matrix are linearly independent, $d\phi_Q$ is an injection, and therefore we may conclude that $S$ is a regular surface.
    \end{proof}
    \item Compute the basis $\phi_u,\phi_v$ for $T_{(x,y,z)}S$. $$\phi_u=(1,0,\frac{u}{\sqrt{1+u^2+v^2}})$$ $$\phi_v=(0,1,\frac{v}{\sqrt{1+u^2+v^2}})$$
    \item Consider the bilinear form $\langle\cdot,\cdot\rangle_{2,1}$ on $\R^3$ defined by $\langle(v_1,v_2,v_3),(w_1,w_2,w_3)\rangle_{2,1}=v_1w_1+v_2w_2-v_3w_3$. Use the basis you made in part (b) to expand the restriction of $\langle\cdot,\cdot\rangle_{2,1}$ to $T_{(x,y,z)}S$, as follows: Find $E,F,$ and $G$, where $$\langle r\phi_u+s\phi_v,r\phi_u+s\phi_v\rangle_{2,1}=r^2E+2rsF+s^2G$$ \skip $$r\phi_u=(r,0,\frac{ru}{\sqrt{1+u^2+v^2}})$$ $$s\phi_v=(0,s,\frac{sv}{\sqrt{1+u^2+v^2}})$$ $$r\phi_u+s\phi_v=(r,s,\frac{ru+sv}{\sqrt{1+u^2+v^2}})$$ So $$\langle r\phi_u+s\phi_v,r\phi_u+s\phi_v\rangle_{2,1}=r^2+s^2-\frac{r^2u^2+2rsuv+s^2v^2}{1+u^2+v^2}$$ Thus, \begin{align*}
        E&=1-\frac{u^2}{1+u^2+v^2}\\ F&=-\frac{uv}{1+u^2+v^2}\\ G&=1-\frac{v^2}{1+u^2+v^2}
    \end{align*}
    \item Let $Q(x,y)=ax^2+2bxy+cy^2$ be a quadratic form on $\R^2$ with $a>0$. Recall that $Q$ is a positive-definite provided $Q(x,y)\geq0$, with equality iff $x=y=0$. Show that $Q$ is positive-definite iff $ac-b^2>0$.\begin{proof}$(\Longrightarrow):$ Assume $ac-b^2>0$, and $x\neq0$. The discriminant of this quadratic is given by: $$4b^2y^2-4acy^2=4y^2(b^2-ac)$$ Since $y^2$ must be positive, and per our assumption $b^2-ac<0$, it follows that the discriminant of the polynomial is negative, therefore it has no real roots. Since a quadratic polynomial with positive first coefficient ($a>0$) is always positive if it has no real roots, we conclude that $Q(x,y)$ is positive under this assumption.\skip Now if $x=0$, then $$Q(x,y)=\underbrace{c}_{>0}\cdot\underbrace{y^2}_{\geq0}$$ Thus $Q(x,y)\geq0$, with equality whence and only whence $x=y=0$.\skip $(\Longleftarrow):$ Assume $Q$ is positive-definite. Then it follows that $Q(x,y)\geq0$, with equality iff $x=y=0$. Thus, $$ax^2+2bxy+cy^2\geq0$$ Since $Q$ is positive definite: $$ax^2+2bxy+cy^2=0\Longleftrightarrow x=y=0$$ and $$ax^2+2bxy+cy^2>0$$ otherwise. If this is the case, then $Q$ must have no real roots. This is logically equivalent to the following statement: $$4b^2y^2-4acy^2=4y^2(b^2-ac)<0$$ This implies $$b^2-ac<0\implies ac>b^2$$ As required.
    \end{proof}
    \item Use part (d) to show that the restriction of $v\mapsto\langle v,v\rangle_{2,1}$ to $T_{(x,y,z)}S$ is positive-definite.
\end{enumerate}
\item Let $a,b,c\in\R$ with $a\geq b\geq c>0$.\begin{enumerate}[label=(\alph*).]
    \item Show that the ellipsoid $\frac{x^2}{a^2}+\frac{y^2}{b^2}+\frac{z^2}{c^2}=1$ is a regular surface.\begin{proof}Solving for $z$: $$z=\sqrt{c^2-\frac{c^2}{a^2}x^2-\frac{c^2}{b^2}y^2}$$ Which is defined by $$(x,y)\mapsto\left(x,y,\sqrt{c^2-\frac{c^2}{a^2}x^2-\frac{c^2}{b^2}y^2}\right)$$ Which is a smooth function. Now simply repeat, however solving for $y$ this time: $$y=\sqrt{b^2-\frac{a^2}{b^2}x^2-\frac{c^2}{b^2}z^2}$$ Gives a similar function: $$(x,z)\mapsto\left(x,\sqrt{c^2-\frac{c^2}{a^2}x^2-\frac{c^2}{b^2}y^2},z\right)$$ Which is again smooth. Do the same for $x$ and we have 3 smooth functions that cover the whole ellipse. Therefore we may conclude that the ellipse must be a regular surface.
    \end{proof}
    \item Show that the Gaussian curvature at $p=(x,y,z)$ is given by $$\frac{1}{a^2b^2c^2}\left(\frac{x^2}{a^2}+\frac{y^2}{b^2}+\frac{z^2}{c^2}\right)^{-2}$$
    Using the above parametrization, we can say $$X(u,v)=\left(u,v,\sqrt{c^2-\frac{c^2}{a^2}u^2-\frac{c^2}{b^2}v^2}\right)$$
    Gaussian curvature can be defined as the determinant of the differential of the Gauss map. Computing the Gauss map: $$N(p)=\frac{X_u(u,v)\times X_v(u,v)}{||X_u(u,v)\times X_v(u,v)||}=$$ $$=\frac{\begin{pmatrix}\frac{c^2x\sqrt{a^2}\sqrt{b^2}}{a^2\sqrt{c^2}\sqrt{a^2b^2-a^2y^2-b^2x^2}}&\frac{c^2y\sqrt{a^2}\sqrt{b^2}}{b^2\sqrt{c^2}\sqrt{a^2b^2-a^2y^2-b^2x^2}}&1\end{pmatrix}}{||\begin{pmatrix}\frac{c^2x\sqrt{a^2}\sqrt{b^2}}{a^2\sqrt{c^2}\sqrt{a^2b^2-a^2y^2-b^2x^2}}&\frac{c^2y\sqrt{a^2}\sqrt{b^2}}{b^2\sqrt{c^2}\sqrt{a^2b^2-a^2y^2-b^2x^2}}&1\end{pmatrix}||}=$$ 
    \text{Ok this is getting very messy I'm sure I messed up somewhere.}
\end{enumerate}
\item Rotate the graph $y=\frac{1}{x}$ in $\R^2$ about the $x$-axis to obtain a regular surface. Use the formulas derived in class to compute the Gaussian curvature of the regular surface. What are the max and min values of the Gaussian curvature? (This is Gabriel's Horn correct?) $$X(u,v)=(u,\frac{\cos v}{u},\frac{\sin{v}}{u})$$
$$K=-\frac{eg-f^2}{EG-F^2}$$
*Did calculations on scratch paper it would take a while to Tex them out.
\begin{align*}
    X_u&=\left(1,-\frac{\cos v}{u^2},-\frac{\sin v}{u^2}\right)\\
    X_v&=\left(0,-\frac{\sin v}{u},\frac{\cos v}{u}\right)\\
    E&=1+\frac{a^2}{u^4}\\
    F&=0\\
    G&=\frac{a^2}{u^2}\\
    e&=-\frac{2a}{u\sqrt{a^2+u^4}}\\
    f&=0\\
    g&=\frac{au}{\sqrt{a^2+u^4}}\\
    \text{Therefore }K&=\frac{eg}{EG}\\
    &=-\frac{\frac{2a^2}{a^2+u^4}}{\frac{a^2}{u^2}+\frac{a^4}{u^6}}\\
    &=\frac{2a^2}{a^2+u^4}\cdot(\frac{a^2}{u^2}+\frac{a^4}{u^6})\\
    &=-\frac{2u^6}{(a^2+u^4)^2}
\end{align*}
\end{enumerate}
\end{document}