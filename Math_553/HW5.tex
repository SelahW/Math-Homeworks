\documentclass[a4paper,17pt]{extarticle}
\usepackage{geometry}
\usepackage[utf8]{inputenc}
\usepackage[table,xcdraw]{xcolor}
\usepackage{mathtools}
\usepackage{amsthm}
\usepackage{amsmath}
\usepackage{amsfonts}
\usepackage{amssymb}
\usepackage{centernot}
\usepackage{marvosym}
\usepackage{enumitem}
\usepackage{hyperref}
\setcounter{tocdepth}{1}
\theoremstyle{definition}
\newtheorem{definition}{Definition}
\let\marvosymLightning\Lightning
\renewcommand{\skip}{\par\null\par}
\newcommand{\T}{\mathcal T}
\renewcommand{\O}{\mathcal{O}}
\renewcommand{\Lightning}{\scalebox{1.5}{\marvosymLightning}}
\newtheorem{theorem}{Theorem}
\newtheorem{corollary}{Corollary}[theorem]
\newtheorem*{remark}{Remark}
\renewcommand\qedsymbol{QED}
\newcommand{\R}{\mathbb{R}}
\newcommand{\N}{\mathbb{N}}
\newcommand{\Z}{\mathbb{Z}}
\newcommand{\divby}{%
  \mathrel{\text{\vbox{\baselineskip.65ex\lineskiplimit0pt\hbox{.}\hbox{.}\hbox{.}}}}%
  }
\newcommand{\notdivby}{\centernot\divby}
\title{\scalebox{1.5}{Math 553 Homework 5}}
\author{\scalebox{1.5}{Theo Koss}}
\date{April 2022}
\begin{document}
\maketitle
\section{Section 2.5}
\begin{itemize}
    \item Problem 1: Compute the first fundamental forms of the following parametrized surfaces where they are regular:\begin{enumerate}[label=\textbf{\alph*.}]
        \item $X(u,v)=(a\sin u\cos v,b\sin u\sin v,c\cos u)$; ellipsoid\skip $$X_u=(a\cos u\cos v,b\cos u\sin{v},-c\sin{u})$$ $$X_v=(-a\sin u\sin v,b\sin u\cos v,0)$$ $$E=X_u\cdot X_u=(a^2\cos^2v+b^2\sin^2v)\cos^2u+c^2\sin^2u$$ $$F=X_u\cdot X_v=\frac{1}{4}(b^2-a^2)\sin(2u)\sin(2v)$$ $$G=X_v\cdot X_v=(a^2\sin^2v+b^2\cos^2v)\sin^2u$$
        $$\text{First fundamental form is always:}$$ $$Edu^2+2Fdudv+Gdv^2$$ (I'll omit writing it out, just going to write down the coefficients)
        \item $X(u,v)=(au\cos v,bu\sin v,u^2);$ elliptic paraboloid. $$X_u=(a\cos v,b\sin v,2u),X_v=(-au\sin v,bu\cos v,0)$$ $$E=a^2\cos^2v+b^2\sin^2v+4u^2$$ $$F=\frac{u}{2}(b^2-a^2)\sin(2v)$$ $$G=(a^2\sin^2v+b^2\cos^2v)u^2$$
        \item $X(u,v)=(au\cosh{v},bu\sinh{v},u^2);$ hyperbolic paraboloid. $$X_u=(a\cosh{v},b\sinh{v},2u),X_v=(au\sinh{v},bu\cosh{v},0)$$ $$E=a^2\cosh^2v+b^2\sinh^2v+4u^2$$ $$F=\frac{u}{2}(b^2+a^2)\sinh{2v}$$ $$G=(a^2\sinh^{v}+b^2\cosh^2{v})u^2$$
        \item $X(u,v)=(a\sinh{u}\cos v,b\sinh{u}\sin v,c\cosh{u});$ hyperboloid of two sheets. $$X_u=(a\cosh{u}\cos v,b\cosh{u}\sin v,c\sinh{u})$$ $$X_v=(-a\sinh{u}\sin v,b\sinh{u}\cos v,0)$$ $$E=(a^2\cos^2v+b^2\sin^2v)\cosh^2u+c^2\sinh^2u$$ $$F=\frac{1}{4}(b^2-a^2)\sinh(2u)\sin(2v)$$ $$G=(a^2\sin^2v+b^2\cos^2v)\sinh^2u$$
    \end{enumerate}
    \item Problem 3: Obtain the first fundamental form of the sphere in the parametrization given by the stereographic projection. $$X(u,v)=\left(\frac{4u}{u^2+v^2+4},\frac{4v}{u^2+v^2+4},\frac{2(u^2+v^2)}{u^2+v^2+4}\right)$$ Therefore, $$X_u=\frac{4}{(u^2+v^2+4)^2}(-u^2+v^2+4,-2uv,2u)$$ and $$X_v=\frac{4}{(u^2+v^2+4)^2}(-2uv,u^2-v^2+4,2v)$$ Then: $$E=\frac{16}{(u^2+v^2+4)^4}[(u^2+v^2+4)^2-12u^2]$$ $$F=-\frac{192uv}{(u^2+v^2+4)^4}$$ $$G=\frac{16}{(u^2+v^2+4)^4}[(u^2+v^2+4)^2-12v^2]$$
    Let $\zeta=u^2+v^2+4$ then, $$E=\frac{16(\zeta^2-12u^2)}{\zeta^4},F=-\frac{192}{\zeta^4},G=\frac{16(\zeta^2-12v^2)}{\zeta^4}$$
    \item Problem 5: Show that the area $A$ of a bounded region $R$ of the surface $z=f(x,y)$ is $$A=\iint_Q\sqrt{1+f^2_x+f^2_y}dxdy$$ where $Q$ is the normal projection of $R$ onto the $xy$ plane.\begin{proof}If the surface is given by: $$z=f(x,y)$$ Then $R$ can be parametrized by $x:Q\to S$ given by: $$X(u,v)=(u,v,f(u,v))$$ Then, taking partial derivatives: $$X_u=(1,0,f_u(u,v))$$ $$X_v=(0,1,f_v(u,v))$$ Then, taking the cross product: $$X_u\times X_v=(-f_u(u,v),-f_v(u,v),1)$$ By definition, the area is then $$A=\iint_Q||X_u\times X_v||dudv=\iint_Q\sqrt{1+f^2_x+f^2_y}dxdy$$ As required.
    \end{proof}
    \item Problem 7: The coordinate curves of a parametrization $X(u,v)$ are a Tchebyshef net if the lengths of the opposite sides of any quadrilateral formed by them are equal. Show that a necessary and sufficient condition for this is $$\frac{\partial E}{\partial v}=\frac{\partial G}{\partial u}=0$$\begin{proof} $(\Longrightarrow)$: Suppose that the coordinate curves of $X$ form a Tschebyshef net. Let $X(u_1,v_1)$ and $X(u_2,v_2)$ be the opposite vertices of a quadrilateral. Then, clearly, the other two vertices are $X(u_1,v_2)$ and $X(u_2,v_1)$. Since the system is Tschebyshef, opposite sides have the same length, therefore: $$\int_{u_1}^{u_2}||X_u(u,v_1)||du=\int_{u_1}^{u_2}||X_u(u,v_2)||du$$
    and $$\int_{v_1}^{v_2}||X_v(u_1,v)||dv=\int_{v_1}^{v_2}||X_v(u_2,v)||dv$$ Then, if $u_1=u_0$ and $v_1=v_0$ are constants and $u_2=u$, $v_2=v$ we have: $$\int_{u_0}^u||X_u(t,v_0)||dt=\int_{u_0}^u||X_u(t,v)||dt$$ Since $E=X_u\cdot X_u$, $$\int_{u_0}^u\sqrt{E(t,v_0)}dt=\int_{u_0}^u\sqrt{E(t,v)}dt$$ Now differentiating both sides with respect to $v$, we achieve: $$\int_{u_0}^u\underbrace{\partial_v\sqrt{E(t,v_0)}}_{=0}dt=\int_{u_0}^u\partial_v\sqrt{E(t,v)}dt$$ $$\implies 0=\int_{u_0}^u\frac{1}{2\sqrt{E(t,v)}}\cdot\frac{\partial E}{\partial v}(t,v)dt$$ Since this holds for all $u$, this implies that $$\frac{\partial E}{\partial v}(t,v)=0$$ (Since $\frac{1}{2\sqrt{E(t,v)}}$ can't).\skip
    Similarly, using equation 2, we achieve: $$\frac{\partial G}{\partial u}=0$$
    \skip $(\Longleftarrow):$ Suppose that $$\frac{\partial E}{\partial v}=0$$ This necessarily implies $$E=E(u)$$ Then $$\int_{u_1}^{u_2}||X_u(t,v_1)||dt=\int_{u_1}^{u_2}\sqrt{E(t)}dt=\int_{u_1}^{u_2}||X_u(t,v_2)||dt$$
    Again, similarly, $$\frac{\partial G}{\partial u}=0$$ implies $$\int_{v_1}^{v_2}||X_v(u_1,t)||dt=\int_{u_1}^{u_2}\sqrt{G(t)}dt=\int_{u_1}^{u_2}||X_v(u_2,t)||dt$$
    \end{proof}
    \item Problem 10: Let $P=\{(x,y,z)\in\R^3;z=0\}$ be the $xy$ plane and let $X:U\to P$ be a parametrization of $P$ given by $$X(\rho,\theta)=(\rho\cos\theta,\rho\sin\theta)$$ where $$U=\{(\rho,\theta)\in\R^2;p>0,0<\theta<2\pi\}$$
    Compute the coefficients of the first fundamental form of $P$ in this parametrization. $$P=(\rho\cos\theta,\rho\sin\theta,0)$$ $$P_{\rho}=(\cos\theta,\sin\theta,0)$$ $$P_{\theta}=(-\rho\sin\theta,\rho\cos\theta,0)$$ $$E=1\quad F=0\quad G=\rho^2$$ $$\text{FFF}=dudu+\rho^2dvdv$$
\end{itemize}
\section{Section 2.6}
\begin{enumerate}
    \item  Let S be a regular surface covered by coordinate neighborhoods $V_1$ and $V_2$. Assume that $V_1\cap V_2$ has two connected components, $W_1$, $W_2$, and that the Jacobian of the change of coordinates is positive in $W_1$ and negative in $W_2$. Prove that S is nonorientable. 
    \skip *Not sure*
\end{enumerate}
\end{document}