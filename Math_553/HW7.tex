\documentclass[a4paper,17pt]{extarticle}
\usepackage{geometry}
\usepackage[utf8]{inputenc}
\usepackage[table,xcdraw]{xcolor}
\usepackage{mathtools}
\usepackage{amsthm}
\usepackage{amsmath}
\usepackage{amsfonts}
\usepackage{amssymb}
\usepackage{centernot}
\usepackage{marvosym}
\usepackage{enumitem}
\usepackage{hyperref}
\setcounter{tocdepth}{1}
\theoremstyle{definition}
\newtheorem{definition}{Definition}
\let\marvosymLightning\Lightning
\renewcommand{\skip}{\par\null\par}
\newcommand{\T}{\mathcal T}
\renewcommand{\O}{\mathcal{O}}
\renewcommand{\Lightning}{\scalebox{1.5}{\marvosymLightning}}
\newtheorem{theorem}{Theorem}
\newtheorem{corollary}{Corollary}[theorem]
\newtheorem*{remark}{Remark}
\renewcommand\qedsymbol{QED}
\newcommand{\R}{\mathbb{R}}
\newcommand{\N}{\mathbb{N}}
\newcommand{\Z}{\mathbb{Z}}
\newcommand{\divby}{%
  \mathrel{\text{\vbox{\baselineskip.65ex\lineskiplimit0pt\hbox{.}\hbox{.}\hbox{.}}}}%
  }
\newcommand{\notdivby}{\centernot\divby}
\title{\scalebox{1.5}{Math 553 Homework 7}}
\author{\scalebox{1.5}{Theo Koss}}
\date{May 2022}
\begin{document}
\maketitle
\section{Section 3.2}
\begin{itemize}
    \item Problem 3: Let $C\subset S$ be a regular curve on a surface $S$ with Gaussian curvature $K>0$. Show that the curvature $k$ of $C$ at $p$ satisfies $$|k|\geq\min(|k_1|,|k_2|)$$ Where $k_1$ and $k_2$ are the principal curvatures of $S$ at $p$.\begin{proof}By definition the normal curvature $k_n$ of $C$ at $p$ is, $$k_n=k\cos\theta$$ where $\theta=$ the angle between $n(p)$ and $N(p)$, the normal vectors of $C$ and $S$ at $p$, respectively. Suppose $|k_1|\leq|k_2|$, since $k_n$ lies between these 2, and $K>0$, we have that $k_1$ and $k_2$ are nonzero and have the same sign. And, \begin{align*}
           |k_1|&\leq|k_n|\leq|k_2|\\
           \implies|k_1|&\leq k|\cos\theta|\leq|k_2|
    \end{align*}
Since the maximum of $|\cos\theta|=1$, we have \begin{align*}
    k\geq k|\cos\theta|\geq|k_1|\implies k\geq\min(|k_1|,|k_2|)
\end{align*} 
    \end{proof}
    \item Problem 4: Assume that a surface $S$ has the property that $|k_1|\leq1,$ $|k_2|\leq1$ everywhere. Is it true that the curvature $k$ of a curve on $S$ also satisfies $|k|\leq1$?\begin{proof} Let $C$ be a curve on $S$ with curvature $k$. By definition $$k_n=k\cos\theta\implies|k_n|=|k\cos\theta|$$ Now, applying Euler's formula, \begin{align*}
        k_n&=k\cos\theta\\
        k_n&=k_1\cos^2\theta+k_2\sin^2\theta\\
        |k\cos\theta|&=|k_1\cos^2\theta+k_2\sin^2\theta|\\
        |k\cos\theta|&\leq|k_1\cos^2\theta|+|k_2\sin^2\theta|\\
        |k\cos\theta|&\leq\cos^2\theta+\sin^2\theta=1
    \end{align*}
    This implies $|k|\leq\frac{1}{|\cos\theta|}$, since $|\cos\theta|\leq1$, $|k|\leq1$ only when $\theta=n\pi$.
    \end{proof}
    \item Problem 6: Show that the sum of the normal curvatures for any pair of orthogonal directions, at a point $p\in S$, is constant.\begin{proof} According to Euler's formula, $$k_n(\theta)=k_1\cos^2\theta+k_2\sin^2\theta$$ Where $\theta$ is some direction on $S$ at $p$. Then for an orthogonal direction, we consider $\theta+\frac{\pi}{2}$. Call the vector in this direction $v$. Its normal curvature is then given by:\begin{align*}
        k_v(\theta)&=k_n\left(\theta+\frac{\pi}{2}\right)\\
        &=k_1\cos^2\left(\theta+\frac{\pi}{2}\right)+k_2\sin^2\left(\theta+\frac{\pi}{2}\right)\\
        &=k_1\left(\cos\left(\theta+\frac{\pi}{2}\right)\right)^2+k_2\left(\sin\left(\theta+\frac{\pi}{2}\right)\right)^2\\
        &=k_1((\cos\theta)(0)-(\sin\theta)(1))^2+k_2((\sin\theta)(0)+(\cos\theta)(1))^2\\
        &=k_1\sin^2\theta+k_2\cos^2\theta\\
    \end{align*}
    Therefore the sum of the normal curvatures for these two directions is:\begin{align*}
        k_n(\theta)+k_v(\theta)&=(k_1\cos^2\theta+k_2\sin^2\theta)+(k_1\sin^2\theta+k_2\cos^2\theta)\\
        &=k_1(\cos^2\theta+\sin^2\theta)+k_2(\sin^2\theta+\cos^2\theta)\\
        &=k_1+k_2\\
    \end{align*}
    Which is constant.
    \end{proof}
    \item Problem 17: Show that if $H\equiv0$ on $S$ and $S$ has no planar points, then the Gauss map $N:S\to S^2$ has the following property: $$\langle dN_p(w_1),dN_p(w_2)\rangle=-K(p)\langle w_1,w_2\rangle$$ for all $p\in S$ and all $w_1,w_2\in T_p(S)$. Show that the above condition implies that the angle of two intersecting curves on $S$ and the angle of their spherical images are equal up to a sign.\begin{proof}
    Let $e_1,e_2$ be the principal directions of $p\in S$. Then we have: $$w_1=a_1e_1+a_2e_2$$ and $$w_2=b_1e_1+b_2e_2$$ Then $$dN_p(w_1)=dN_p(a_1e_1+a_2e_2)=-a_1k_1e_1-a_2k_2e_2$$ And $$dN_p(w_2)=dN_p(b_1e_1+b_2e_2)=-b_1k_1e_1-b_2k_2e_2$$ Then the inner product $$\langle dN_p(w_1),dN_p(w_2)\rangle=a_1b_1k_1^2+a_2b_2k_2^2$$
    Then, since $H\equiv0$ on $S$, $k_1=-k_2\implies K(p)=-k_1^2=-k_2^2$. Since $S$ has no planar points, $K(p)$ is not 0. Thus, $$\langle dN_p(w_1),dN_p(w_2)\rangle=-K(p)\langle w_1,w_2\rangle$$
    \end{proof}
\end{itemize}
\end{document}