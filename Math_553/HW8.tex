\documentclass[a4paper,17pt]{extarticle}
\usepackage{geometry}
\usepackage[utf8]{inputenc}
\usepackage[table,xcdraw]{xcolor}
\usepackage{mathtools}
\usepackage{amsthm}
\usepackage{amsmath}
\usepackage{amsfonts}
\usepackage{amssymb}
\usepackage{centernot}
\usepackage{marvosym}
\usepackage{enumitem}
\usepackage{hyperref}
\setcounter{tocdepth}{1}
\theoremstyle{definition}
\newtheorem{definition}{Definition}
\let\marvosymLightning\Lightning
\renewcommand{\skip}{\par\null\par}
\newcommand{\T}{\mathcal T}
\renewcommand{\O}{\mathcal{O}}
\renewcommand{\Lightning}{\scalebox{1.5}{\marvosymLightning}}
\newtheorem{theorem}{Theorem}
\newtheorem{corollary}{Corollary}[theorem]
\newtheorem*{remark}{Remark}
\renewcommand\qedsymbol{QED}
\newcommand{\R}{\mathbb{R}}
\newcommand{\N}{\mathbb{N}}
\newcommand{\Z}{\mathbb{Z}}
\newcommand{\divby}{%
  \mathrel{\text{\vbox{\baselineskip.65ex\lineskiplimit0pt\hbox{.}\hbox{.}\hbox{.}}}}%
  }
\newcommand{\notdivby}{\centernot\divby}
\title{\scalebox{1.5}{Math 553 Homework 8}}
\author{\scalebox{1.5}{Theo Koss}}
\date{May 2022}
\begin{document}
\maketitle
\section{Section 4.3}
\begin{itemize}
    \item Problem 2: Show that if $x$ is an isothermal parametrization, that is, $E=G=\lambda(u,v)$ and $F=0$, then $$K=-\frac{1}{2\lambda}\Delta(\log{\lambda})$$ where $\Delta\varphi$ denotes the Laplacian $(\delta^2\varphi/\delta u^2)+(\delta^2\varphi/\delta v^2)$ of the function $\varphi$. Conclude that when $E=G=(u^2+v^2+c^2)^{-2}$ and $F=0$, then $K=\text{const.}=4c$.\begin{proof}By Exercise 1 in this section, if $x$ is an orthogonal parametrization, then $$K=-\frac{1}{2\sqrt{EG}}\left\{\left(\frac{E_v}{\sqrt{EG}}\right)_v+\left(\frac{G_u}{\sqrt{EG}}\right)_u\right\}.$$ Trivially, if $x$ is an isothermal parametrization, it must also be an orthogonal parametrization, so it remains to show that \begin{enumerate}
        \item $2\sqrt{EG}=2\lambda$
        \item $\Delta(\log\lambda)=\left(\frac{E_v}{\sqrt{EG}}\right)_v+\left(\frac{G_u}{\sqrt{EG}}\right)_u$
    \end{enumerate}Since $E=G=\lambda>0$, $\sqrt{EG}=\sqrt{\lambda^2}=\lambda$. Therefore $-\frac{1}{2\sqrt{EG}}=-\frac{1}{2\lambda}$.\skip Then,\begin{align*}
    \Delta(\log\lambda)&=\delta_u^2(\log\lambda)+\delta_v^2(\log\lambda)\\
    &=(\delta_v(\log\lambda))_v+(\delta_u(\log\lambda))_u\\
    &=\left(\frac{\lambda_v}{\lambda}\right)_v+\left(\frac{\lambda_u}{\lambda}\right)_u\\
    &=\left(\frac{E_v}{\sqrt{EG}}\right)_v+\left(\frac{G_u}{\sqrt{EG}}\right)_u\\
    \end{align*}
    As required.
    \end{proof}
    \skip Now, plugging in $E=G=\lambda=(u^2+v^2+c^2)^{-2}$. First calculating the Laplacian:\begin{align*}
        \Delta(\log\lambda)&=\Delta(\log(u^2+v^2+c^2)^{-2})\\
        &=\left(\frac{-\frac{4v}{\left(u^2+v^2+c^2\right)^3}}{(u^2+v^2+c^2)^{-2}}\right)_v+\left(\frac{-\frac{4u}{\left(u^2+v^2+c^2\right)^3}}{(u^2+v^2+c^2)^{-2}}\right)_u\\
        &=\left(-\frac{4v(u^2+v^2+c^2)^2}{\left(u^2+v^2+c^2\right)^3}\right)_v+\left(-\frac{4u(u^2+v^2+c^2)^2}{\left(u^2+v^2+c^2\right)^3}\right)_u\\
        &=\left(-\frac{4v}{\left(u^2+v^2+c^2\right)}\right)_v+\left(-\frac{4u}{\left(u^2+v^2+c^2\right)}\right)_u\\
        &=-\frac{4\left(-v^2+c^2+u^2\right)}{\left(v^2+u^2+c^2\right)^2}-\frac{4\left(v^2+c^2-u^2\right)}{\left(v^2+u^2+c^2\right)^2}\\
        &=-8c\lambda
    \end{align*}
    The Gaussian curvature is then,
    \begin{align*}
        K&=-\frac{1}{2\lambda}\Delta(\log{\lambda})\\
        &=-\frac{1}{2\lambda}\cdot-8c\lambda\\
        &=4c
    \end{align*}
    As required.
    \item \textbf{Problem.} Consider the unit disk $\mathbb{D}=\{(x,y)\in\R^2:x^2+y^2<1\}$ endowed with FFF $$\frac{dx^2+dy^2}{4(1-x^2-y^2)^2}$$ Use the previous problem to show that the Gaussian curvature of this surface satisfies $K(x,y)=-1$ for all $(x,y)\in\mathbb{D}$.\skip Since the first fundamental form is what it is, we know that $\lambda=E=G=\frac{1}{4(1-x^2-y^2)^2}$ and $F=0$. Thus it follows that $$K=-\frac{1}{2\lambda}\Delta(\log{\lambda})$$ And we must show that $$-\frac{1}{2\lambda}\Delta(\log{\lambda})=-1$$ Therefore it will suffice to show that $2\lambda=\Delta(\log\lambda)$. Computing the Laplacian:\begin{align*}
        \Delta(\log\lambda)&=\Delta\left(\log\left(\frac{1}{4(1-x^2-y^2)^2}\right)\right)\\
        &=\left(\frac{\frac{x}{2\left(1-x^2-y^2\right)^2}}{\frac{1}{4(1-x^2-y^2)^2}}\right)_x+\left(\frac{\frac{y}{2\left(1-x^2-y^2\right)^2}}{\frac{1}{4(1-x^2-y^2)^2}}\right)_y\\
        &=\left(\frac{2\lambda x}{\lambda}\right)_x+\left(\frac{2\lambda y}{\lambda}\right)_y\\
        &=\delta_x(\lambda x)+\delta_y(\lambda y)\\
        &=2\lambda
    \end{align*}
\end{itemize}
\end{document}