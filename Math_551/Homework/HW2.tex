\documentclass[hidelinks,12pt]{article}
\usepackage[utf8]{inputenc}
\usepackage[table,xcdraw]{xcolor}
\usepackage{mathtools}
\usepackage{amsthm}
\usepackage{amsmath}
\usepackage{amsfonts}
\usepackage{amssymb}
\usepackage{centernot}
\usepackage{marvosym}
\usepackage{enumitem}
\usepackage{hyperref}
\setcounter{tocdepth}{1}
\theoremstyle{definition}
\newtheorem{definition}{Definition}[section]
\let\marvosymLightning\Lightning
\renewcommand{\Lightning}{\scalebox{1.5}{\marvosymLightning}}
\newtheorem{theorem}{Theorem}
\newtheorem{corollary}{Corollary}[theorem]
\newtheorem*{remark}{Remark}
\renewcommand\qedsymbol{QED}
\newcommand{\R}{\mathbb{R}}
\newcommand{\N}{\mathbb{N}}
\newcommand{\Z}{\mathbb{Z}}
\newcommand{\divby}{%
  \mathrel{\text{\vbox{\baselineskip.65ex\lineskiplimit0pt\hbox{.}\hbox{.}\hbox{.}}}}%
  }
\newcommand{\notdivby}{\centernot\divby}
\title{\scalebox{2}{Math 551 Homework 2}}
\author{\scalebox{1.5}{Theo Koss}}
\date{October 2021}
\begin{document}
\maketitle
\section{Section 2.3}
\begin{itemize}
    \item 2: Let $x$ be a number and $A$ a subset of $\R$.\begin{enumerate}[label=(\alph*)]
        \item Prove that if $d(x,A)>0$, then $d(x,y)>0$ for all $y\in A$.\begin{proof}Let $x\in\R$ and $A\subset\R$ such that $A\neq\emptyset$. Assume $d(x,A)>0$. Thus $d(x,A)>0$ is the infimum of all distances $|x-y|$ for $y\in A$. Thus by definition of infimum, $d(x,A)\leq|x-y|=d(x,y)$ for all $y$ in $A$. Therefore $d(x,y)>0$.
        \end{proof}
        \item Give an example for which $d(x,y)>0$ for all $y\in A$, but $d(x,A)=0$.\newline Consider $A=(0,1)$ and $x=0$. Then $d(x,A)=0$ but $d(x,y)>0$ for all $y\in A$.
        \end{enumerate}
    \item 3: Prove that a subset of $\R$ is bounded if and only if it has both upper and lower bounds.\begin{proof}$(\Longrightarrow)$: Suppose $A$ has a lower bound $x$ and an upper bound $y$. Then $A\subseteq[x,y]$ and using the triangle inequality, $d(a,b)<y-x$. Therefore $A$ is bounded.
    \newline$(\Longleftarrow)$: Suppose that $A$ is bounded, then there is some $n$ such that $d(a,b)<n$ for all $a,b\in A$. Then $A\subseteq[a-n,a+n]$ for some $a\in A$. and therefore it has lower and upper bounds.
    \end{proof}
    \item 4: If $\{C_i\}^{n}_{i=1}$ is a finite family of closed sets, then $\cup_{i=1}^{n}C_i$ is closed.\begin{proof}We must show that $\R\backslash \{C_i\}^{n}_{i=1}$ is open.$$\cup_{i\in(1,n)}(\R\backslash C_i)=\R\backslash\cap_{i\in(1,n)}C_i$$ Is open by theorem 2.7. Therefore, by definition, $\cup_{i=1}^{n}C_i$ is closed.
    \end{proof}
    \item 8: Show that if $x$ is the limit of the sequence $\{a_n\}^{\infty}_{n=1}$ of real numbers and all the terms of the sequence are distinct, then $x$ is a limit point of the range of the sequence. Give an example to show that the limit of a sequence may not be a limit point of the range of the sequence if the terms of the sequence are not distinct.\newline Consider some $x$ where it is the limit of the sequence $\{a_n\}^{\infty}_{n=1}$ of real numbers. Then given $\epsilon>0$ there is a positive integer $N$ such that if $n\geq N$ then $|a_n-x|<\epsilon$. Therefore if we consider the range of the sequence, $\{a_0,a_n\}$, then we must show that $x$ is a limit point. Recall that $x$ is a limit point of this range iff every neighborhood of $x$ contains a separate point of the range.$$a_n\in(x-\epsilon,x+\epsilon)$$ Therefore every neighborhood of $x$ contains a separate point of the range.\newline If the terms are not distinct, then we can consider the sequence $\{b_n\}^{\infty}_{n=1}$ where $b_n=x$ for all $n\in\N$. Then this sequence certainly converges to $x$, however you can not find a point other than $x$ that is contained in the range.
    \item 9: Let $x\in\R$ and $A\subset\R$.\begin{enumerate}[label=(\alph*)]
        \item Prove that $x$ is a limit point of $A$ if and only if there is a sequence of distinct points of $A$ which converges to $x$. \begin{proof}See the above problem for one direction.\newline For the other direction, assume there is a sequence of distinct points of $A$ which converge to $x$. Then, by the question above, $x$ is a limit point of the range of $A$.\end{proof}
        \item Prove that $x$ is a limit point of $A$ if and only if every open set containing $x$ contains infinitely many points of $A$.\begin{proof}$(\Longrightarrow)$: Assume $x$ is a limit point of $A$, then for every open interval around $x$, there is an element of $A\neq x$ in that interval, therefore, by varying the size of the interval, you can get infinitely many points.\newline $(\Longleftarrow)$: Assume there are infinitely many points of $A$ such that every open set containing $x$ contains a distinct point of $A$ which is not $x$. Then, by definition, $x$ is a limit point.
        \end{proof}
    \end{enumerate}
\end{itemize}
\section{Section 2.4}
\begin{itemize}
    \item 2: Give an example of a nested sequence $\{[a_n,b_n]\}_{n=1}^{\infty}$ whose intersection is empty.\newline Let $a_n=0,\forall n\in\N$, and let $b_n=\frac{1}{n},\forall n\in\N$. Then their intersection is empty, since any element would be greater than 0, yet less than $\frac{1}{n}$ for all $n\in\N$. No such element exists.
    \item 3: Consider $[0,1]$ and the family of open intervals $O=\{(-0.001,0.001),(0.999,1.001)\}\cup\{\frac{1}{n},1\}_{n=1}^{\infty}$. Find a finite subcollection of $O$ whose union contains $[0,1]$. *I don't understand this problem
    \item 4: Prove the Bolzano-Weierstrass Theorem. Every bounded, infinite subset of $\R$ has a limit point. \begin{proof}Begin with a bounded sequence $(x_n)$ (Call it $[a,b]$): Using the bisection argument, we can show that $[a,b]$ is a sequence of nested intervals, since they are nested, the intersection of all of these intervals is nonempty, thus there is a number $x$ which is in each interval of $[a,b]$. This is a limit point of $(x_n)$.
    \end{proof}
    \item 8: Show that every uncountable subset of $\R$ has a limit point.\begin{proof} Suppose that $A\subset\R$ is uncountable. For $n\in\Z$, let $A_n=A\cap[n,n+1]$. Some $A_n$, call it $A_N$, is infinite. Since $A_N\subseteq[N,N+1]$, $A_N$ has a limit point (call it $x$) in $[N,N+1]$. This point is also a limit point of $A$ in $\R$, since if there exists a neighborhood of $x$ in $\R$, then $(\text{Said neighborhood}\cap[N,N+1])$ is a neighborhood of $x$ in $[N,N+1]$. Therefore it contains a point of $A_N$ other than $x$.
    \end{proof}
\end{itemize}
\section{Section 3.1}
\begin{enumerate}
    \item 
\end{enumerate}
\end{document}